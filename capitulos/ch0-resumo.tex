O processo de avaliação é uma etapa fundamental para a verificação de aprendizagem e manutenção do andamento do ensino conforme o currículo previsto. Dentro da avaliação de aprendizagem, as questões discursivas são comumente utilizadas para desenvolver o pensamento crítico e as habilidades de escrita. Porém, com mais alunos, torna-se necessário ao professor o desenvolvimento de seus métodos, sem tornar a avaliação um fator limitante. Alinhado a isso, ressaltamos a quantidade de material para avaliação, mesmo que individualmente as respostas representarem pequenas quantidades de texto produzido. Portanto, o professor precisa avaliar cautelosamente todos os alunos para identificar possíveis problemas no aprendizado. Além disso, a adesão de métodos de suporte educacional tende a melhorar a qualidade dos materiais e impactar diretamente no desenvolvimento do aluno. Deste modo, neste trabalho apresentamos um modelo avaliativo usando \textit{Active Learning}, ou seja, combinando os resultados de \textit{clusterização} e classificação para apoio ao tutor na avaliação de respostas discursivas curtas. Através do reconhecimento das estruturas textuais de forma gramatical, morfológica, semântica, sintática, estatística ou sequencial identificamos padrões textuais em cada conjunto de respostas por questão. Assim, com os modelos de resposta anotados, ajustamos o modelo avaliativo para se aproximar das expectativas de nota do professor. Deste modo, apresentamos a robustez do modelo avaliativo produzido pelo sistema através de diferentes \textit{datasets} da literatura. Em 255 questões, com um total de 65875 respostas, alcançamos \textit{Accuracy} média de 72,11\% e \textit{F1} ponderado de 70,63\% em relação aos avaliadores humanos.

\textbf{Palavras-chaves}: Avaliação Automática de Questões Discursivas. Aprendizado Ativo. Sistemas de Apoio ao Tutor. Processamento de Linguagem Natural. Classificação de Texto.