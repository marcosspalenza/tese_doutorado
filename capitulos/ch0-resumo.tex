O processo de avaliação é uma etapa fundamental para a verificação de aprendizagem e manutenção do andamento do ensino conforme o currículo previsto. Dentro da avaliação de aprendizagem, as questões discursivas são comumente utilizadas para desenvolver o pensamento crítico e as habilidades de escrita. Porém, com mais alunos, torna-se necessário ao professor o desenvolvimento de seus métodos, sem tornar a avaliação um fator limitante. Alinhado a isso, existe em sua totalidade uma grande quantidade de material, mesmo que a produção individual do estudante seja pequena. Assim, apesar da quantidade, o professor precisa analisar em detalhes cada uma das respostas dos estudantes para identificar \textit{gaps} na aprendizagem. Deste modo, a adoção de métodos de suporte educacional busca a melhoria da capacidade analítica deste professor, impactando diretamente no acompanhamento do aluno. Neste trabalho apresentamos um modelo de \textit{Active Learning} para classificação de documentos educacionais, em especial a avaliação de respostas discursivas curtas. Para isso, combinamos métodos de clusterização e classificação com enriquecimento textual de forma gramatical, morfológica, semântica, sintática, estatística e sequencial para identificação dos padrões de respostas. Enquanto o sistema detecta os padrões textuais que se aproximam do modelo de correção do professor, este têm menor esforço de correção e suporte para seu modelo avaliativo. Para teste deste modelo, utilizamos um total de 65875 respostas em 255 questões da literatura, alcançando em média \textit{accuracy} de 72\% e \textit{F1} ponderado de 70\% em relação aos avaliadores humanos.


\textbf{Palavras-chaves}: Avaliação Automática de Questões Discursivas. Aprendizado Ativo. Sistemas de Apoio ao Tutor. Processamento de Linguagem Natural. Classificação de Texto.