O processo de avaliação é uma etapa muito importante para a verificação de aprendizagem e manutenção do andamento do ensino conforme o currículo previsto. Dentro da avaliação de aprendizagem, as questões discursivas são comumente utilizadas para desenvolver o pensamento crítico e as habilidades de escrita. Conforme é ampliado o acesso à educação, é importante que os métodos avaliativos também sejam adequados para não representarem um fator limitante. Nesse aspecto é importante ressaltar que, apesar da pequena quantidade de texto produzido, é necessário que o professor avalie cautelosamente todos os alunos para identificar possíveis problemas no aprendizado. Além disso, o tempo concorrente entre a análise de desempenho dos alunos, planejamento das aulas e atualização dos materiais impossibilita o acompanhamento detalhado do aluno em classe. Portanto, a adesão de métodos de suporte educacional é fundamental para melhorar a qualidade dos materiais e impactar diretamente no desenvolvimento do aluno. Neste trabalho, apresentamos uma ferramenta de apoio ao tutor na análise, correção e produção de \textit{feedbacks} para o método avaliativo de respostas discursivas curtas. Através de técnicas de aprendizado semi-supervisionado em \textit{Machine Learning}, o sistema auxilia o tutor na identificação principais respostas para reduzir o esforço de correção. Com os modelos avaliativos em meio computacional, o professor audita os resultados produzidos pelo sistema e acompanha seu processo de decisão. Deste modo, apresentamos a robustez do modelo avaliativo produzido pelo sistema através de diferentes \textit{datasets} da literatura. Em 200 questões com um total de 9892 respostas, alcançamos \textit{Accuracy} média de 60,62\% e \textit{F1} ponderado de 57,84\% em relação aos avaliadores humanos.

\textbf{Palavras-chaves}: Avaliação Automática de Questões Discursivas. Aprendizado Semi-Supervisionado. Sistemas de Apoio ao Tutor. Processamento de Linguagem Natural. Classificação de Texto.