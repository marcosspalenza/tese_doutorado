% ==============================================================================
% TCC - Nome do Aluno
% Capítulo 3 - Considerações Finais
% ==============================================================================
\chapter{Considerações Finais}
\label{cap-conclusao}

O processo de avaliação de questões discursivas compreende um longo ciclo, da formulação das atividades até a análise de desempenho. Através das avaliações acontece o redimensionamento das práticas de ensino e aprendizagem. Portanto, a dinâmica avaliativa é fundamental para que os professores compreendam quais são os \textit{gaps} de aprendizagem. Além disso, a avaliação indica formas de refinar a disciplina conforme o desempenho de cada estudante e cada turma. Portanto, é fundamental a produção de técnicas que resultem na redução do esforço do professor para aplicação cotidiana das atividades.

Em especial, a aplicação de atividades que contribuem para melhoria da leitura e escrita são fundamentais para todos os níveis de instrução dos estudantes. Assim, essa aplicação contínua têm benefícios para todos os participantes do ciclo avaliativo. Sabendo disso, apresentamos neste trabalho um estudo de aplicação de métodos de \textit{Active Learning} para suporte ao critério avaliativo do professor. Como destaques dos resultados obtidos com este trabalho listamos uma série de contribuições que compõe deste estudo:

\begin{itemize}
  \item Aplicação de técnicas de \textit{Active Learning} para refinamento dos ciclos avaliativos;
  \item Dinâmica de otimização dos ciclos de clusterização e classificação para análise da distribuição das amostras e suas componentes;
  \item Proposta de amostragem guiada pela representatividade de cada amostra no \textit{cluster} que esta compõe;
  \item Modelo de linguagem formado por sequências de estruturas textuais para caracterizar padrões de resposta superiores que o nível de \textit{token};
  \item Reconhecimento de padrões que combinam a interação entre estudantes, o professor e o sistema na avaliação, detalhando o critério segundo o alinhamento ao tema;
  \item Criação de formatos próprios para relatórios e \textit{feedbacks} para descrição do vínculo entre termos e notas.
\end{itemize}

\newpage

\section{Conclusões}

O suporte computacional do processo avaliativo compõe importante parte da integração do ensino em meio digital. O meio digital visa tornar prático e rápido o processo de avaliativo, possibilitiando sua aplicação em massa. Deste modo, o professor em um mesmo ambiente consegue interagir com todos os seus alunos e acompanhar seu desempenho na disciplina. 

Por meio destas plataformas de ensino, possibilitamos o emprego de múltiplas técnicas de EDM para análise do conteúdo e, consequentemente, o aumento da capacidade avaliativa. Esse trabalho apresenta um estudo profundo do ciclo avaliativo para integração humano-máquina na correção de respostas discursivas via \textit{Active Learning}. Descrevemos o \textit{p}Nota, um sistema para construção de modelos avaliativos que interage diretamente com o tutor para compreender seu critério avaliativo sobre um conjunto de resposta. Integram o sistema vários módulos que, em sequência, realizam o reconhecimento de padrões e identificação contextual.

Deste modo, o método elabora uma forma de interagir com o professor para criação de modelos textuais para representar cada nota. O modelo alcançou níveis similares aos observados entre humanos na qualidade de atribuição de notas, com F1-ponderado médio de 78\% e ACC de 79\%. Destacamos que tais valores refletem que, das 255 atividades de classificação, 137 foram avaliadas com mais de 75\% nesta métrica. O mesmo desempenho também se reflete nas atividades de regressão, com erros menores do que observados entre humanos nos \textit{datasets} do teste. Deste modo, com esse modelo esperamos que o professor atue de forma conjunta com o sistema, corrigindo atividades e utilizando os resultados em função do desenvolvimento de seus métodos de ensino. Adicionalmente, através dos \textit{feedbacks}, esperamos compor materiais que auxiliem várias etapas do método avaliativo, incluindo a discussão de resultados em sala.

\section{Trabalhos Futuros}

Em uma perspectiva de próximos estudos em torno do modelo proposto, destacamos alguns trabalhos ainda de refinamento do modelo. O ganho analítico dos modelos de linguagem e domínio, em especial com \textit{word embeddings}, podem proporcionar novas formas de conhecimento. Mesmo atingindo resultados melhores do que os trabalhos da literatura que apresentam tais características, unindo tais modelos podemos ter um ganho importante com contextualização. Em uma outra vertente, também são válidos os esforços para entender as variações textuais encontradas em novas iterações de aplicação das atividades. Inclusive, são necessários estudos que caracterizem melhor a evolução do aprendizado.

Outra característica importante para evolução do modelo é a integração de critérios mais sofisticados na otimização dos resultados, para clusterização, classificação e regressão. Tais métodos podem ser relevantes para alcançar ainda mais qualidade na atribuição de notas, inclusive compreendendo a formação de cada conjunto de respostas. Nessa mesma linha, são fundamentais o acompanhamento do sistema em características de usabilidade. Portanto, além de conhecer a prática e os ciclos avaliativos também devemos considerar ter mais proximidade nas iterações sob a ótica do professor. Em especial, devemos contemplar o acompanhamento de escolas, turmas ou grupos de alunos sob a concepção das técnicas de ensino-aprendizagem. Nesse aspecto, esperamos melhorar a contextualização dos documentos, com o detalhamento da produção escrita e dos ciclos de aplicação das atividades.