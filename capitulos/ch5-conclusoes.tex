% ==============================================================================
% Tese Marcos A. Spalenza
% Capítulo 3 - Considerações Finais
% ==============================================================================
\chapter{Considerações Finais}
\label{cap-conclusao}

O processo de avaliação de questões discursivas compreende um longo ciclo, da formulação das atividades até a análise de desempenho. Por meio das avaliações que há o redimensionamento das práticas de ensino e aprendizagem. Assim, a dinâmica avaliativa do professor permite identificação e tratar os \textit{gaps} de aprendizagem. A avaliação também é um indicativo para possíveis refinamentos da disciplina. Conforme o desempenho de cada estudante e cada turma são identificadas as melhores formas para aprimorar as técnicas e apresentar os resultados obtidos em sala. Entretanto, uma premissa para alcançar tais resultados de produção técnica é que sejam reduzidos os esforços da aplicação das atividades discursivas nas rotinas de ensino.

Em especial, a aplicação de atividades que contribuem para melhoria da leitura e da escrita é essencial para todos os níveis de instrução dos estudantes. Assim, a continuidade do processo avaliativo tem benefícios para todos os participantes do ciclo. Cientes disso, foi apresentado neste trabalho um estudo das estruturas textuais combinados com técnicas de \textit{Active Learning} para suporte a avaliação de respostas discursivas. Como destaques dos resultados obtidos com este trabalho foram listadas uma série de contribuições que compõem este estudo:

\begin{itemize}
  \item Análise contextual do modelo linguístico através de \textit{Active Learning} com refinamento por ciclo avaliativo;
  \item Análise da distribuição de amostras combinando processos de clusterização e classificação;
  \item Otimização da seleção de hiperparâmetros, identificando a composição espacial dos \textit{clusters};
  \item Amostragem por representatividade intra-\textit{cluster} com anotação guiada do usuário para extração de relações de equivalência e divergência contextual;
  \item Construção do modelo com padrões de diferentes camadas da linguagem, para caracterizar formas distintas de resposta em nível de termos e sequências.
  \item Reconhecimento de padrões que combinam a interação entre estudantes, o professor e o sistema na avaliação, detalhando o critério segundo o alinhamento ao tema;
  \item Criação de formatos próprios para relatórios e \textit{feedbacks}, com nível de explicabilidade entre termos e notas, acompanhando didático e auditoria dos resultados.
  \item Suporte para uma série de pesquisas pedagógicas, colaborando com duas Dissertações de Mestrado do Mestrado Profissional em Química em Rede Nacional (ProfQui) do Instituto Federal do Espírito Santo.


\end{itemize}

\newpage

\section{Conclusões}

O suporte computacional do processo avaliativo compõe importante parte da integração do ensino em meio digital. O meio digital possibilita a aplicação em massa, tornando prático e rápido o ciclo de cada avaliação. Desse modo, o professor em um mesmo ambiente consegue interagir com todos os seus alunos e acompanhar seu desempenho na disciplina. 

Por meio dessas plataformas de ensino, possibilita-se o emprego de múltiplas técnicas de EDM para análise do conteúdo e, consequentemente, o aumento da capacidade avaliativa. Ao longo deste trabalho foi descrito o \textit{p}Nota. O \textit{p}Nota é um sistema para construção de modelos avaliativos que interage diretamente com o tutor para compreender seu critério avaliativo sobre um conjunto de resposta. Portanto, neste trabalho, está descrito um estudo profundo do ciclo avaliativo para integração humano-máquina na correção de respostas discursivas via \textit{Active Learning}. Como descrito neste trabalho, integram o \textit{p}Nota uma série módulos que foram adaptados para os ciclos avaliativos observados entre os professores, realizando o reconhecimento de padrões e a identificação contextual.

Como destaque aos resultados observados, temos níveis de desempenho similares aos observados entre dois humanos na atribuição de notas, com F1-ponderado médio de 78\% e ACC de 79\%. Destaca-se que tais valores refletem que, das 255 atividades de classificação, 137 foram avaliadas com mais de 75\% de ACC. O mesmo desempenho também se reflete nas atividades de regressão. Os resultados obtidos com o sistema, indicam que é possível que o mesmo faça parte da rotina pedagógica do professor. Assim, com o \textit{p}Nota o professor pode corrigir atividades e utilizar os resultados em função do desenvolvimento de seus métodos de ensino. Adicionalmente, com os \textit{feedbacks}, esperamos compor materiais que auxiliem várias etapas do método avaliativo, incluindo a discussão de resultados em sala.

\section{Trabalhos Futuros}

Em uma perspectiva de próximos estudos em torno do modelo proposto, destacam-se alguns trabalhos ainda de refinamento do modelo. O ganho analítico dos modelos de linguagem e domínio, em especial com \textit{word embeddings} e \textit{Large Language Models}, podem proporcionar ao sistema novos níveis de conhecimento. Unindo esses modelos pode-se ter um ganho importante com contextualização. Em uma outra vertente, também são válidos os esforços para entender as variações textuais encontradas em novas iterações de aplicação das atividades. Inclusive, são necessários estudos que se aprofundem no ganho de informação com a evolução do aprendizado. Assim, para além do nível linguístico, é essencial mensurar a capacidade de vincular termos e notas, aprendendo as componentes da avaliação.

Outra característica importante para evolução do modelo é a integração de critérios mais sofisticados na otimização dos resultados, para clusterização, classificação e regressão. A otimização pode ser relevante para alcançar ainda mais qualidade na atribuição de notas, extraindo detalhes da formação de cada conjunto de respostas. Nessa mesma linha, são fundamentais o acompanhamento do sistema em características de usabilidade. Portanto, além de conhecer a prática e os ciclos avaliativos também devemos considerar ter mais proximidade nas iterações sob a ótica do professor. Em especial, é necessário contemplar em detalhes o acompanhamento de escolas, turmas ou grupos de alunos sob a concepção das técnicas de ensino-aprendizagem. Nesse aspecto, é desejável a melhoria da contextualização dos documentos, com o detalhamento da produção escrita e dos ciclos de aplicação das atividades.