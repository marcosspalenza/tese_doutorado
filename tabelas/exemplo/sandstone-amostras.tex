0 & 353 & maxsim & the sandstone was probably formed in a desert or low energy environment- reddened grains suggests oxidisation of iron took place. \\ \cline{3-4}
0 & 753 & maxsim & southwest of england and wales- iron oxide from southwestern united states. quartz arenites. \\ \cline{3-4}
0 & 353 & minsim & the sandstone was probably formed in a desert or low energy environment- reddened grains suggests oxidisation of iron took place. \\ \cline{3-4}
0 & 354 & minsim & the sandstone was originally a beach or similar low energy environment- reddened grains suggests oxidisation of iron took place. \\ \cline{3-4}
0 & 753 & minsize & southwest of england and wales- iron oxide from southwestern united states. quartz arenites. \\ \cline{3-4}
0 & 354 & maxsize & the sandstone was originally a beach or similar low energy environment- reddened grains suggests oxidisation of iron took place. \\ \cline{3-4}
0 & 1098 & silhcoeff & the sandstone is transported by wind- originated in oxygen rich atmosphere is chemically weathered and transported in high energy conditions. \\ \cline{3-4}
1 & 1197 & maxsim & igneous rock-formed deep within earth and cooled slowly.mainly quartz-chemically weathered- longer aeolian processed-desert enviroment \\ \cline{3-4}
1 & 542 & maxsim & sandstone is a sedimentary rock composed mainly of sand-size mineral or rock grains. \\ \cline{3-4}
1 & 246 & minsim & it would have orriganated in a dessert because there reddened grains. it would have been deposited by slow moving water. \\ \cline{3-4}
1 & 247 & minsim & it would have orriganated in a dessert because there reddened grains. it would have been deposited by slow moving water. \\ \cline{3-4}
1 & 265 & minsize & the sandstone indicates continued hot- oxidising conditions and from a sedimentary origin. \\ \cline{3-4}
1 & 1197 & maxsize & igneous rock-formed deep within earth and cooled slowly.mainly quartz-chemically weathered- longer aeolian processed-desert enviroment \\ \cline{3-4}
1 & 266 & silhcoeff & the sandstone indicates continued hot- oxidising conditions and layered. so therefore from a sedimentary origin. \\ \cline{3-4}
2 & 845 & maxsim & well sorted and deposited by a current that gradually slowed by sediment slowing \\ \cline{3-4}
2 & 43 & maxsim & it is a desert sandstone. \\ \cline{3-4}
2 & 26 & minsim & that the sandstone originated from a desert \\ \cline{3-4}
2 & 111 & minsim & that the sandstone originated in the desert \\ \cline{3-4}
2 & 302 & minsize & weathered \\ \cline{3-4}
2 & 845 & maxsize & well sorted and deposited by a current that gradually slowed by sediment slowing \\ \cline{3-4}
2 & 1858 & silhcoeff & it is a sedimentary rock formed from grains from an arid desert \\ \cline{3-4}
2 & 1618 & silhcoeff & the sandstone grains originated in an arid desert. \\ \cline{3-4}
2 & 1485 & silhcoeff & the sandstone formed from ancient desert deposits. \\ \cline{3-4}
2 & 1174 & silhcoeff & this sandstone has actually originated from a rock in a desert. \\ \cline{3-4}
2 & 1119 & silhcoeff & the deposited sand originates from an ancient desert. \\ \cline{3-4}
2 & 52 & silhcoeff & this indicates it may be desert sandstone formed by the wind \\ \cline{3-4}
2 & 538 & silhcoeff & it originated from and area where iron was present in a river. \\ \cline{3-4}
2 & 347 & silhcoeff & the rock is from weathering which has produced a sedimentary rock. \\ \cline{3-4}
2 & 1570 & silhcoeff & the rock originated in the desert and was blown by the wind. \\ \cline{3-4}
2 & 714 & silhcoeff & it was probably transported by wind in desert conditions. \\ \cline{3-4}
2 & 1301 & silhcoeff & the rock is made from sand grains from the desert. \\ \cline{3-4}
2 & 620 & silhcoeff & the grains came from a desert sand dune. \\ \cline{3-4}
2 & 524 & silhcoeff & this rock appears to have its origin in desert sands. \\ \cline{3-4}
2 & 454 & silhcoeff & desert sand dunes sorted by wind. \\ \cline{3-4}
2 & 903 & silhcoeff & it indicates the origin is arid desert conditions. \\ \cline{3-4}
2 & 1157 & silhcoeff & it formed on land in arid conditions and was transported by wind \\ \cline{3-4}
2 & 1566 & silhcoeff & these indicators signify desert conditions in the past. \\ \cline{3-4}
2 & 1403 & silhcoeff & this is from the desert in origin it is windblown and formed on a dune. \\ \cline{3-4}
2 & 1894 & silhcoeff & the rock originates from ancient desert deposits. \\ \cline{3-4}
2 & 407 & silhcoeff & the sand was deposited as layers of an ancient desert. \\ \cline{3-4}
2 & 80 & silhcoeff & formed after high energy transportation under desert conditions \\ \cline{3-4}
2 & 931 & silhcoeff & most likely formed under highly oxidising desert conditions. \\ \cline{3-4}
2 & 685 & silhcoeff & arid dessert conditions sorted by wind. \\ \cline{3-4}
2 & 609 & silhcoeff & the rock would have originated from a desert and moved by wind \\ \cline{3-4}
2 & 890 & silhcoeff & it was a sedimentary rock that was formed under desert cinditions. \\ \cline{3-4}
2 & 1478 & silhcoeff & sedimentary rock formed in deserts and contains iron \\ \cline{3-4}
2 & 1210 & silhcoeff & this indicates desert conditions and windy conditions in the past \\ \cline{3-4}
2 & 675 & silhcoeff & formed in desert conditions in an oxygen rich atmosphere \\ \cline{3-4}
2 & 704 & silhcoeff & formed in desert conditions in oxygen rich atmosphere \\ \cline{3-4}
2 & 1717 & silhcoeff & it was sorted under stable flow conditions within a desert \\ \cline{3-4}
2 & 1646 & silhcoeff & transported by water then weathered in an oxygen rich atmosphere \\ \cline{3-4}
2 & 1808 & silhcoeff & this rock originated in a windblown desert dune. \\ \cline{3-4}
2 & 1833 & silhcoeff & the grains are formed from fragments of a pre-existing rock \\ \cline{3-4}
2 & 1811 & silhcoeff & the it probably originated from the top of a mountain and was carried by a river \\ \cline{3-4}
2 & 1812 & silhcoeff & it probably originated from the top of a mountain and was carried by a river \\ \cline{3-4}
2 & 1175 & silhcoeff & this comes from a rock in desert and the grains are oxidised. \\ \cline{3-4}
2 & 821 & silhcoeff & the sediments are from the desert and they were transported and deposited by wind \\ \cline{3-4}
2 & 1892 & silhcoeff & formed by weathering caused by the wind in a desert \\ \cline{3-4}
2 & 491 & silhcoeff & this rock was weathered in arid desert conditions \\ \cline{3-4}
2 & 922 & silhcoeff & the rock has been transported by air and originated in a desert \\ \cline{3-4}
2 & 1131 & silhcoeff & the rock originated in an arid desert climate \\ \cline{3-4}
2 & 595 & silhcoeff & it was formed in oxidising conditions- possibly in a desert environment \\ \cline{3-4}
2 & 1334 & silhcoeff & chemical weathering in hot humid africa \\ \cline{3-4}
2 & 1102 & silhcoeff & transported by water. oxidation of minerals. \\ \cline{3-4}
2 & 1285 & silhcoeff & the sandstone would have been formed in desert conditions. \\ \cline{3-4}
2 & 316 & silhcoeff & the sandstone was formed from an oxidizing environment in a desert. \\ \cline{3-4}
2 & 683 & silhcoeff & the origin of the sandstone would be from an arid desert conditon \\ \cline{3-4}
2 & 1188 & silhcoeff & the rock must have been whethered in a oxygen rich environment. \\ \cline{3-4}
2 & 797 & silhcoeff & the sandstone originated in desert conditions. \\ \cline{3-4}
2 & 1253 & silhcoeff & the sandstone originated in desert conditions. \\ \cline{3-4}
2 & 1483 & silhcoeff & the sandstone originated from desert conditions. \\ \cline{3-4}
3 & 603 & maxsim & the rounded stones indicate much erosion the redenned grains were formed at high temperatures and high pressure. \\ \cline{3-4}
3 & 1675 & maxsim & the sandstone would be from a wind blown desert deposit- the red is due to fine grain iron sulfide. \\ \cline{3-4}
3 & 603 & minsim & the rounded stones indicate much erosion the redenned grains were formed at high temperatures and high pressure. \\ \cline{3-4}
3 & 604 & minsim & the rounded stones indicate much erosion the redenned grains were formed at high temperatures and high pressure. \\ \cline{3-4}
3 & 379 & minsize & it was formed in a slow moving water system- in oxygen-rich conditions. \\ \cline{3-4}
3 & 1675 & maxsize & the sandstone would be from a wind blown desert deposit- the red is due to fine grain iron sulfide. \\ \cline{3-4}
3 & 451 & silhcoeff & this tells us that the rock was windblown- and probably originated from a sand dune in a desert. \\ \cline{3-4}
3 & 122 & silhcoeff & it originated after 2400ma- when atmospheric conditions were oxidising therefore giving the red coloured grains. \\ \cline{3-4}
3 & 916 & silhcoeff & this sandstone formed in arid and highly oxidative desert conditions- where wind blew the quartz grains. \\ \cline{3-4}
3 & 288 & silhcoeff & this rock has been a long time having much physical and chemical weathering- pitted by being transported by wind. \\ \cline{3-4}
3 & 120 & silhcoeff & it originated after 2400ma- when atmospheric conditions were oxidising- therefore giving an abundance of oxidised fe ions. \\ \cline{3-4}
3 & 478 & silhcoeff & transported by wind- probably from desert sand; contains hematite to give red colour. \\ \cline{3-4}
3 & 802 & silhcoeff & a high-energy- oxidising environment probably a desert was a feature in this sands past. \\ \cline{3-4}
4 & 589 & maxsim & formed under arid desert conditions. wind fine sand grains transported by wind. oxidising conditions- example of red-beds \\ \cline{3-4}
4 & 1304 & maxsim & desert sand- transported by wind- bounced on ground to create pitted surface which sorts them- are compressed into sandstone. \\ \cline{3-4}
4 & 86 & minsim & high energy transportation and deposition in oxidising conditions on land formed a red-bed- so is probably desert sandstone. \\ \cline{3-4}
4 & 87 & minsim & high energy transportation by wind- deposition in oxidising conditions on land forming a red-bed- so probably desert sandstone. \\ \cline{3-4}
4 & 86 & minsize & high energy transportation and deposition in oxidising conditions on land formed a red-bed- so is probably desert sandstone. \\ \cline{3-4}
4 & 589 & maxsize & formed under arid desert conditions. wind fine sand grains transported by wind. oxidising conditions- example of red-beds \\ \cline{3-4}
5 & 1596 & maxsim & involved in low tectonic activity- transported in water- rounded from erosion- red from association with iron oxides and weathering \\ \cline{3-4}
5 & 310 & maxsim & they were formed by desert conditions- wind repeated impacts caused a frosted appearance. there are iron minerals present. \\ \cline{3-4}
5 & 371 & minsim & origins are arid desert conditions with rounded pitted grains caused by wind transportation and the red surfaces by iron oxide. \\ \cline{3-4}
5 & 1608 & minsim & red haematite coating/cement from arid environment- probably a desert- and grains rounded and pitted by wind transportation \\ \cline{3-4}
5 & 326 & minsize & it has come from arid desert- the red colour shows oxidation and pitted grains shows the rock has been frosted. \\ \cline{3-4}
5 & 269 & maxsize & it originates from arid desert conditions- well -round indicates ancient sandstone- reddened from coating of iron oxide from oxidising conditions. \\ \cline{3-4}
5 & 1510 & silhcoeff & found in arid desert conditions - result of wind transportation. iron oxide coating due to oxidising conditions. \\ \cline{3-4}
 & \multicolumn{3}{r}{$ \hdots $ Foram omitidos os demais resultados (58 \textit{clusters} no total).} \\