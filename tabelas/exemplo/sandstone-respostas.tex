A249457X & 0.0 & this sand in this rock originally was located in a desert in oxidising conditions. 
 \\ \cline{2-3}

A249457X & 0.0 & this sand in this rock originally was located in a desert in oxidising conditions. 
 \\ \cline{2-3}

A249457X & 1.0 & this sand in this rock originally was located in a hot region in windy oxidising conditions. 
 \\ \cline{2-3}

A2548811 & 1.0 & it is most likely to have formed under arid desert conditions and transported by air and wind 
 \\ \cline{2-3}

T0632747 & 1.0 & sedimentary oxidised desert sandstone well weathered from strong winds 
 \\ \cline{2-3}

Y9003030 & 0.0 & from a desert and blown by the wind. once oxidising conditions where iron sulfide oxidised.maybe from humid conditions 
 \\ \cline{2-3}

A3345548 & 1.0 & reddened suggests desert. wellsorted fine pitted suggests blown and deposited by wind. wellrounded suggests possibly a dune. 
 \\ \cline{2-3}

U0991840 & 0.0 & the rock was formed in a gradually slowing current and have been exposed to oxidising conditions 
 \\ \cline{2-3}

U0991840 & 1.0 & there has been wind erosion and oxidised iron and have then been rounded and deposited in a gradual current. 
 \\ \cline{2-3}

Y8831914 & 0.0 & this rock layers together by the sand and grains of preexisting rocks of different size shape to form sandstone 
 \\ \cline{2-3}

Y8831914 & 0.0 & sandstone is a rock sedimentary which contain rocks and sand layered together. 
 \\ \cline{2-3}

Y8831914 & 0.0 & sandstone is a rock sedimentary which contain rocks and sand layered together which mean that transported by weathering. 
 \\ \cline{2-3}

A2415177 & 1.0 & weathered quartz grains from a granite containing iron carried by wind in a desert. 
 \\ \cline{2-3}

X0816556 & 0.0 & not deposited slowly deposition from air 
 \\ \cline{2-3}

X0816556 & 0.0 & deposition from air the grains contain some iron oxid 
 \\ \cline{2-3}

Y6544188 & 0.0 & the presence of reddned grains shows that sandstone derivates from sediments. sandstone is a sedimentary rock. 
 \\ \cline{2-3}

Y6544188 & 0.0 & the rock is sedimentary in origin. 
 \\ \cline{2-3}

Y6544188 & 0.0 & the sandstone was formed by deposition of flowing water. the reddened grains shows that sanstone is rich in oxygen. 
 \\ \cline{2-3}

A2482988 & 0.0 & it came from a higher class of diy store probably one of the quality garden centres squires rather than bq. 
 \\ \cline{2-3}

A2482988 & 0.0 & rolling down river of iron oxide weathering sedimentary clastic rocks full of hyrothermal fluids 
 \\ \cline{2-3}

A2482988 & 0.0 & the question is about the origins not the mode of transport as to how they got where they are. 
 \\ \cline{2-3}

A1424820 & 0.0 & this rock originated in an arid desert environment that allowed physical erosion of sediments in highly oxidising conditions. 
 \\ \cline{2-3}

A1424820 & 1.0 & this sandstone originated in a desert environment with wind blown physical weathering of mineral grains in highly oxidising conditions. 
 \\ \cline{2-3}

A2843422 & 0.0 & the sandstone would have formed and originated in desert conditions as part of desert sands. 
 \\ \cline{2-3}

A2843422 & 0.0 & the sandstone would have originated in the desert. 
 \\ \cline{2-3}

A2843422 & 1.0 & the sandstone would have originated in the desert as part of a windblown dune. 
 \\ \cline{2-3}

Y762039X & 0.0 & that the sandstone originated from a desert 
 \\ \cline{2-3}

Y762039X & 0.0 & originated from a sand desert. 
 \\ \cline{2-3}

Y762039X & 0.0 & originated from dunes in a sand desert. 
 \\ \cline{2-3}

Y3512671 & 1.0 & desert sandstone experiencing high oxidising conditions transported by wind 
 \\ \cline{2-3}

Y6523161 & 0.0 & called a conglomerate and sedimentary in construction broken by weathering erosion transport and finally deposited. 
 \\ \cline{2-3}

Y6523161 & 0.0 & were there before uncomformity occurred then weathering and erosion took place 
 \\ \cline{2-3}

Y6523161 & 1.0 & grains were wind transported and were oxidised under humid conditions 
 \\ \cline{2-3}

A3317409 & 0.0 & transported through water 
 \\ \cline{2-3}

A3317409 & 0.0 & tha sandstone is made from sediment transported through water to its resting place. 
 \\ \cline{2-3}

A3317409 & 0.0 & tha sandstone is made from sediment transported through water to its resting place. 
 \\ \cline{2-3}

Y8696943 & 0.0 & it was most likely to have formed under desert conditions 
 \\ \cline{2-3}

Y8696943 & 1.0 & originated as desert sandblown by wind and undergone oxidisation 
 \\ \cline{2-3}

Y9311424 & 0.0 & snadstone was formed on dea bed that dried 
 \\ \cline{2-3}

Y9311424 & 0.0 & sandstone was formed from sediment that entered a river and deposited 
 \\ \cline{2-3}

Y9311424 & 0.0 & sandstone was formed from sediment that entered a river and deposited they were transported by water 
 \\ \cline{2-3}

Y8964340 & 1.0 & that it may of came from a windy desert environment 
 \\ \cline{2-3}

A2336136 & 0.0 & the rock originated in a hot dry climate such as a desert. 
 \\ \cline{2-3}

A2336136 & 0.0 & it is a desert sandstone. 
 \\ \cline{2-3}

A2336136 & 1.0 & it originates in a hot dry environment weathered and transported by wind. 
 \\ \cline{2-3}

X8581258 & 1.0 & these were desert grains moved by the wind finely pitted by repeated impacts and red from oxidised iron 
 \\ \cline{2-3}

Y1699145 & 1.0 & they would be wind blown desert sand formed in oxidising times like the devonian period. 
 \\ \cline{2-3}

Y8883735 & 1.0 & the grains have been moved by wind colliding with each other to become rounded and pitted with an iron coating. 
 \\ \cline{2-3}

Y8883735 & 1.0 & very chemically weathered and moved by wind. 
 \\ \cline{2-3}

Y8883735 & 1.0 & very chemically weathered and moved by wind and exposed to oxidising conditions. 
 \\ \cline{2-3}

 & \multicolumn{2}{r}{$ \hdots $  Foram omitidos os demais resultados (1897 respostas no total).} \\