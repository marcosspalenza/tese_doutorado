% ==============================================================================
% Modelo para Tese de Doutorado (em Português)
% Prof. Vítor E. Silva Souza - Nemo / DI / UFES
%
% Baseado em abtex2-modelo-trabalho-academico.tex, v-1.9.2 laurocesar
% Copyright 2012-2014 by abnTeX2 group at http://abntex2.googlecode.com/ 
%
% This work may be distributed and/or modified under the conditions of the LaTeX 
% Project Public License, either version 1.3 of this license or (at your option) 
% any later version. The latest version of this license is in
% http://www.latex-project.org/lppl.txt.
%
% IMPORTANTE:
% Instruções encontram-se espalhadas pelo documento. Para facilitar sua leitura,
% tais instruções são precedidas por (*) -- utilize a função localizar do seu
% editor para passar por todas elas.
% ==============================================================================

% Usa o estilo abntex2, configurando detalhes de formatação e hifenização.
\documentclass[
	12pt,				% Tamanho da fonte.
	openright,			% Capítulos começam em página ímpar (insere página vazia caso preciso).
	twoside,			% Para impressão em verso e anverso. Oposto a oneside.
	a4paper,			% Tamanho do papel.
	english,			% Idioma adicional para hifenização.
	french,				% Idioma adicional para hifenização.
	spanish,			% Idioma adicional para hifenização.
	brazil				% O último idioma é o principal do documento.
	]{abntex2}



%%% Importação de pacotes. %%%
% ---- Inicio import do autor ----
\PassOptionsToPackage{table,xcdraw}{xcolor}
\usepackage{pdflscape}
\usepackage{verbatim}
\usepackage{float}
\usepackage{longtable}
\usepackage{multirow}
\usepackage{multicol}
\usepackage{csquotes}
\usepackage{subcaption}

% ---- Fim import do autor    ----

% Conserta o erro "No room for a new \count"
\usepackage{etex}
% \reserveinserts{28}

% Usa a fonte Latin Modern.
\usepackage{lmodern}

% Seleção de códigos de fonte.
\usepackage[T1]{fontenc}

% Codificação do documento em Unicode.
\usepackage[utf8]{inputenc}

% Usado pela ficha catalográfica.
\usepackage{lastpage}

% Indenta o primeiro parágrafo de cada seção.
\usepackage{indentfirst}

% Controle das cores.
\usepackage[usenames,dvipsnames]{xcolor}

% Inclusão de gráficos.
\usepackage{graphicx}

% Tabularx package: for better control of table layout.
\usepackage{tabularx}

% Inclusão de páginas em PDF diretamente no documento (para uso nos apêndices).
\usepackage{pdfpages}

% Para melhorias de justificação.
\usepackage{microtype}


% Citações padrão ABNT.
\usepackage[brazilian,hyperpageref]{backref}
\usepackage[alf]{abntex2cite}	
\renewcommand{\backrefpagesname}{Citado na(s) página(s):~}		% Usado sem a opção hyperpageref de backref.
\renewcommand{\backref}{}										% Texto padrão antes do número das páginas.
\renewcommand*{\backrefalt}[4]{									% Define os textos da citação.
	\ifcase #1
		Nenhuma citação no texto.
	\or
		Citado na página #2.
	\else
		Citado #1 vezes nas páginas #2.
	\fi}

% \rm is deprecated and should not be used in a LaTeX2e document
% http://tex.stackexchange.com/questions/151897/always-textrm-never-rm-a-counterexample
\renewcommand{\rm}{\textrm}

% Pacotes não incluídos no template abntex2. 
% Podem ser comentados caso não queira utilizá-los.

% Inclusão de símbolos não padrão.
\usepackage{amssymb}
\usepackage{eurosym}

% Para utilizar \eqref para referenciar equações.
\usepackage{amsmath}

% Permite mostrar figuras muito largas em modo paisagem com \begin{sidewaysfigure} ao invés de \begin{figure}.
\usepackage{rotating}

% Permite customizar listas enumeradas/com marcadores.
\usepackage{enumitem}

% Permite inserir hiperlinks com \url{}.
\usepackage{bigfoot}
\usepackage{hyperref}

% Permite usar o comando \hl{} para evidenciar texto com fundo amarelo. Útil para chamar atenção a itens a fazer.
\usepackage{soulutf8}

% Colorinlistoftodos package: to insert colored comments so authors can collaborate on the content.
\usepackage[colorinlistoftodos, textwidth=20mm, textsize=footnotesize]{todonotes}
\newcommand{\aluno}[1]{\todo[author=\textbf{Aluno},color=green!30,caption={},inline]{#1}}
\newcommand{\professor}[1]{\todo[author=\textbf{Professor},color=red!30,caption={},inline]{#1}}

% Permite inserir espaço em branco condicional (incluído no texto final só se necessário) em macros.
\usepackage{xspace}

% Permite incluir listagens de código com o comando \lstinputlisting{}.
\usepackage{listings}
\usepackage{caption}
\DeclareCaptionFont{white}{\color{white}}
\DeclareCaptionFormat{listing}{\colorbox{gray}{\parbox{\textwidth}{#1#2#3}}}
\captionsetup[lstlisting]{format=listing,labelfont=white,textfont=white}
\renewcommand{\lstlistingname}{Listagem}
\definecolor{mygray}{rgb}{0.5,0.5,0.5}
\lstset{
	basicstyle=\scriptsize,
	breaklines=true,
	numbers=left,
	numbersep=5pt,
	numberstyle=\tiny\color{mygray}, 
	rulecolor=\color{black},
	showstringspaces=false,
	tabsize=2,
    inputencoding=utf8,
    extendedchars=true,
    literate=%
    {é}{{\'{e}}}1
    {è}{{\`{e}}}1
    {ê}{{\^{e}}}1
    {ë}{{\¨{e}}}1
    {É}{{\'{E}}}1
    {Ê}{{\^{E}}}1
    {û}{{\^{u}}}1
    {ù}{{\`{u}}}1
    {â}{{\^{a}}}1
    {à}{{\`{a}}}1
    {á}{{\'{a}}}1
    {ã}{{\~{a}}}1
    {Á}{{\'{A}}}1
    {Â}{{\^{A}}}1
    {Ã}{{\~{A}}}1
    {ç}{{\c{c}}}1
    {Ç}{{\c{C}}}1
    {õ}{{\~{o}}}1
    {ó}{{\'{o}}}1
    {ô}{{\^{o}}}1
    {Õ}{{\~{O}}}1
    {Ó}{{\'{O}}}1
    {Ô}{{\^{O}}}1
    {î}{{\^{i}}}1
    {Î}{{\^{I}}}1
    {í}{{\'{i}}}1
    {Í}{{\~{Í}}}1
}



%%% Definição de variáveis. %%%

% (*) Substituir os textos abaixo com as informações apropriadas.
\titulo{\textit{p}Nota: Análise das Estruturas Textuais com \textit{Active Learning} para Avaliação de Respostas Discursivas}
\autor{Marcos Alécio Spalenza}
\local{Vitória, ES}
\data{2023}
\orientador{Prof. Dr. Elias de Oliveira}
\coorientador{Profª. Dra. Claudine Badue}
\instituicao{
  Universidade Federal do Espírito Santo -- UFES
  \par
  Centro Tecnológico
  \par
  Programa de Pós-Graduação em Informática}
\tipotrabalho{Tese de Doutorado}

% Preâmbulo (tipo do trabalho, objetivo, nome da instituição, área de concentração, etc.).
% (*) Verificar se está correto (ex.: substituir por Engenharia de Computação se for o caso).
\preambulo{Tese de Doutorado submetida ao Programa de Pós-Graduação em Informática da Universidade Federal do Espírito Santo, como requisito parcial para obtenção do Grau de Doutor em Ciência da Computação.}

% Macros específicas do trabalho.
% (*) Inclua aqui termos que são utilizados muitas vezes e que demandam formatação especial.
% Os exemplos abaixo incluem i* (substituindo o asterisco por uma estrela) e Java com TM em superscript.
% Use sempre \xspace para que o LaTeX inclua espaço em branco após a macro somente quando necessário.
% Macros específicas do trabalho.
% (*) Inclua aqui termos que são utilizados muitas vezes e que demandam formatação especial.
% Os exemplos abaixo incluem i* (substituindo o asterisco por uma estrela) e Java com TM em superscript.
% Use sempre \xspace para que o LaTeX inclua espaço em branco após a macro somente quando necessário.
\begin{comment}
\newcommand{\istar}{\textit{i}$^\star$\xspace}
\newcommand{\java}{Java\texttrademark\xspace}
\newcommand{\latex}{\LaTeX\xspace}
\end{comment}

%%% Configurações finais de aparência. %%%

% Altera o aspecto da cor azul.
\definecolor{blue}{RGB}{41,5,195}

% Informações do PDF.
\makeatletter
\hypersetup{
	pdftitle={\@title}, 
	pdfauthor={\@author},
	pdfsubject={\imprimirpreambulo},
	pdfcreator={LaTeX with abnTeX2},
	pdfkeywords={abnt}{latex}{abntex}{abntex2}{trabalho acadêmico}, 
	colorlinks=true,				% Colore os links (ao invés de usar caixas).
	linkcolor=blue,					% Cor dos links.
	citecolor=blue,					% Cor dos links na bibliografia.
	filecolor=magenta,				% Cor dos links de arquivo.
	urlcolor=blue,					% Cor das URLs.
	bookmarksdepth=4
}
\makeatother

% Espaçamentos entre linhas e parágrafos.
\setlength{\parindent}{1.3cm}
\setlength{\parskip}{0.2cm}



%%% Páginas iniciais do documento: capa, folha de rosto, ficha, resumo, tabelas, etc. %%%

% Compila o índice.
\makeindex

% Inicia o documento.
\begin{document}

% Retira espaço extra obsoleto entre as frases.
\frenchspacing

% Capa do trabalho.
\imprimircapa

% Folha de rosto (o * indica que haverá a ficha bibliográfica).
\imprimirfolhaderosto*


% Ficha catalográfica.
% (*) Escolher entre as versões de ficha catalográfica abaixo (comente aquela que não quiser usar).

% Versão 1: caso a biblioteca da sua universidade lhe forneça um PDF (adequar o nome do arquivo).
% \begin{fichacatalografica}
%     \includepdf{include-fichacatalografica.pdf}
% \end{fichacatalografica}

% Versão 2: caso você tenha que inserir sua própria ficha catalográfica.
% (*) Neste caso, preencher palavras-chave e adicione co-orientador (se houver).
% \begin{fichacatalografica}
%	\vspace*{\fill}
%	\hrule
%	\begin{center}
%	\begin{minipage}[c]{12.5cm}
%	
%	\imprimirautor
%	
%	\hspace{0.5cm} \imprimirtitulo  / \imprimirautor. --
%	\imprimirlocal, \imprimirdata-
%	
%	\hspace{0.5cm} \pageref{LastPage} p. : il. (algumas color.) ; 30 cm.\\
%	
%	\hspace{0.5cm} \imprimirorientadorRotulo~\imprimirorientador\\
%	
%	\hspace{0.5cm}
%	\parbox[t]{\textwidth}{\imprimirtipotrabalho~--~\imprimirinstituicao,
%	\imprimirdata.}\\
%	
%	\hspace{0.5cm}
%		1. Palavra-chave1.
%		2. Palavra-chave2.
%		I. Souza, Vítor Estêvão Silva.
%		II. Universidade Federal do Espírito Santo.
%		IV. \imprimirtitulo \\ 			
%	
%	\hspace{8.75cm} CDU 02:141:005.7\\
%	
%	\end{minipage}
%	\end{center}
%	\hrule
%\end{fichacatalografica}


% Folha de aprovação.
% (*) Escolher entre as versões de ficha catalográfica abaixo (comente aquela que não quiser usar).

% Versão 1: cópia digitalizada da folha de aprovação assinada pela banca.
% \includepdf{include-folhadeaprovacao.pdf}

% Versão 2: folha de aprovação em branco.
% (*) Ajustar a data e os nomes dos participantes da banca.
\begin{comment}
\begin{folhadeaprovacao}
  \begin{center}
    {\ABNTEXchapterfont\large\imprimirautor}
    \vspace*{\fill}\vspace*{\fill}
    \begin{center}
      \ABNTEXchapterfont\bfseries\Large\imprimirtitulo
    \end{center}
    \vspace*{\fill}
    \hspace{.45\textwidth}
    \begin{minipage}{.5\textwidth}
        \imprimirpreambulo
    \end{minipage}%
    \vspace*{\fill}
   \end{center}
   Trabalho aprovado. \imprimirlocal, 22 de março de 2017:
   \assinatura{\textbf{\imprimirorientador} \\ Orientador}
   \assinatura{\textbf{Professor} \\ Convidado 1}
   \assinatura{\textbf{Professor} \\ Convidado 2}
   \assinatura{\textbf{Professor} \\ Convidado 3}
   \assinatura{\textbf{Professor} \\ Convidado 4}
   \begin{center}
    \vspace*{0.5cm}
    {\large\imprimirlocal}
    \par
    {\large\imprimirdata}
    \vspace*{1cm}
  \end{center}  
\end{folhadeaprovacao}
\end{comment}

% Dedicatória.
% (*) Escrever dedicatória ou remover/comentar seção.
\begin{dedicatoria}
   \vspace*{\fill}
   \centering
   \noindent
   \textit{Aos meus pais, \\ Marcos e Sirlene} \vspace*{\fill}
\end{dedicatoria}


% Agradecimentos.
% (*) Escrever agradecimentos ou remover/comentar seção.
\begin{agradecimentos}
Agradeço a Deus.

Agradeço aos meus pais, Marcos e Sirlene, por todo carinho e atenção.

Agradeço ao meu irmão, Murilo, pela companhia incondicional.

Agradeço a cada um dos amigos e amigas que fiz durante todo o período de pós-graduação, por cada um dos momentos compartilhados. Sem dúvida a presença e o apoio de cada um foi essencial para finalização deste trabalho.

Minha gratidão ao professor Elias pela disponibilidade, confiança no meu trabalho e pelo apreço dado a cada resultado obtido durante todo o período do mestrado e doutorado.

Agradeço à professora Claudine pelas importantes contribuições e por cada momento de incentivo, garantindo sempre caminhos e soluções aos problemas.

Agradeço também a todos os demais professores que tornaram esta tese possível, seja durante as aulas, por meio de sugestões ou colaborações em experimentos e artigos.

Agradeço a todos os membros do LCAD e do PPGI, por terem me acolhido, pelo suporte em todas as demandas e por todos esses anos de aprendizado.

Agradeço a Fundação de Amparo à Pesquisa e Inovação do Espírito Santo (FAPES) pelo incentivo a pesquisa e desenvolvimento científico (processo 80136451).

Agradeço a \textit{NVIDIA Corporation} pela doação de uma \textit{NVIDIA TITAN V} através do \textit{NVIDIA Academic Hardware Grant Program}, contribuindo para o desenvolvimento desta e de várias pesquisas do laboratório.

Além disso, gostaria de expressar aqui minha gratidão a todos os profissionais que trabalharam dioturnamente contra a Covid-19, buscando reduzir o impacto e as consequências sobre a população.

\end{agradecimentos}


% Epígrafe.
% (*) Escrever epígrafe ou remover/comentar seção.
\begin{epigrafe}
    \vspace*{\fill}
	\begin{flushright}
		\textit{``It’s not the destination, it’s the journey. And if you guys can understand that, then what you’ll see happen is you won’t accomplish your dreams, your dreams won’t come true; something greater will.''\\
		(Kobe Bryant)}
	\end{flushright}
\end{epigrafe}


\chapter*{Publicações}
Foram desenvolvidas as seguintes publicações no período do Doutorado que, em maior ou menor grau, apresentam certa relação com o tema deste trabalho.

\noindent Publicação de trabalhos em anais de congressos:

\begin{itemize}[label={}]
\scriptsize
\item SPALENZA, M. A.; LUSQUINO-FILHO, L. A. D.; LIMA, P. M. V.; FRANÇA, F. M. G.; OLIVEIRA E. \textit{LCAD-UFES at FakeDeS 2021: Fake News Detection Using Named Entity Recognition and Part-of-Speech Sequences. In: Proceedings of the Iberian Languages Evaluation Forum}. M{\'a}laga, Spain: CEUR-WS, 2021. (IberLEF - SEPLN 2021, v. 37), p. 646-654.

\item SPALENZA, M. A.; OLIVEIRA E.; LUSQUINO-FILHO, L. A. D.; LIMA, P. M. V.; FRANÇA, F. M. G. \textit{Using NER + ML to Automatically Detect Fake News. In: Proceedings of the 20th International Conference on Intelligent Systems Design and Applications}. Online Event: Springer International Publishing, 2020. (ISDA 2020, v. 20), p. 1176-1187.

\item SPALENZA, M. A.; PIROVANI, J. P. C.; OLIVEIRA, E. \textit{Structures Discovering for Optimizing External Clustering Validation Metrics. In: Proceedings of the 19th International Conference on Intelligent Systems Design and Applications}. Auburn (WA), USA: Springer International Publishing, 2019. (ISDA 2019, v. 19), p. 150-161.

\item SPALENZA, M. A.; NOGUEIRA, M. A.; ANDRADE, L. B.; OLIVEIRA E. \textit{Uma Ferramenta para Minera{\c c}{\~a}o de Dados Educacionais: Extra{\c c}{\~a}o de Informa{\c c}{\~a}o em Ambientes Virtuais de Aprendizagem In: Computer on the Beach}. Florian{\'o}polis (SC), Brasil: Universidade do Vale do Itajaí - UNIVALI, 2018. v. 9, p. 741-750.

\end{itemize}

\noindent Participação em trabalhos publicados em anais de congressos:

\begin{itemize}[label={}]
\scriptsize
\item OLIVEIRA, E.; SPALENZA, M. A. ; PIROVANI, J. P. C. \textit{rAVA: A Robot for Virtual Support of Learning.} In: Proceedings of the 20th International Conference on Intelligent Systems Design and Applications. Online Event: Springer International Publishing, 2020. (ISDA 2020, v. 20), p. 1238-1247.

\item SILVA, W.; SPALENZA, M. A.; BOURGUET, J. R.; OLIVEIRA, E. \textit{Towards a Tailored Hybrid Recommendation-based System for Computerized Adaptive Testing through Clustering and IRT. In: Proceedings of the International Conference on Computer Supported Education}. Libon, Portugal: SCITEPRESS, 2020. (CSEDU 2020, v. 12), p. 260-268.

\item SILVA, W.; SPALENZA, M. A.; BOURGUET, J. R.; OLIVEIRA, E. \textit{Recommendation Filtering {\`a} la carte for Intelligent Tutoring Systems. In: Proceedings of the International Workshop on Algorithmic Bias in Search and Recommendation}. Online Event:  Springer International Publishing, 2020. (BIAS 2020, v. 1), p. 58-65.

\item PIROVANI, J. P. C.; ALVES, J. SPALENZA, M. A.; SILVA, W.; COLOMBO, C. S.; OLIVEIRA, E. \textit{Adapting NER (CRF+LG) for Many Textual Genres. In: Proceedings of the Iberian Languages Evaluation Forum}. Bilbao, Spain: CEUR-WS, 2019. (IberLEF - SEPLN 2019, v. 35), p. 421-433. 

\item PIROVANI, J. P. C.; SPALENZA, M. A. ; OLIVEIRA, E. \textit{Gera{\c c}{\~a}o Autom{\'a}tica de Quest{\~o}es a Partir do Reconhecimento de Entidades Nomeadas em Textos Did{\'a}ticos In: XXVIII Simp{\'o}sio Brasileiro de Inform{\'a}tica na Educa{\c c}{\~a}o}. Recife (PE), Brasil: Sociedade Brasileira de Computa{\c c}{\~a}o, 2017. (SBIE 2017, v. 28), p. 1147-1156.

\end{itemize}

\cleardoublepage

% Resumo em português.
% (*) Escrever resumo e palavras-chave.
\setlength{\absparsep}{18pt}
\begin{resumo}
O processo de avaliação é uma etapa muito importante para a verificação de aprendizagem e manutenção do andamento do ensino conforme o currículo previsto. Dentro da avaliação de aprendizagem, as questões discursivas são comumente utilizadas para desenvolver o pensamento crítico e as habilidades de escrita. Conforme é ampliado o acesso à educação, é importante que os métodos avaliativos também sejam adequados para não representarem um fator limitante. Nesse aspecto é importante ressaltar que, apesar da pequena quantidade de texto produzido, é necessário que o professor avalie cautelosamente todos os alunos para identificar possíveis problemas no aprendizado. Além disso, o tempo concorrente entre a análise de desempenho dos alunos, planejamento das aulas e atualização dos materiais impossibilita o acompanhamento detalhado do aluno em classe. Portanto, a adesão de métodos de suporte educacional é fundamental para melhorar a qualidade dos materiais e impactar diretamente no desenvolvimento do aluno. Neste trabalho, apresentamos uma ferramenta de apoio ao tutor na análise, correção e produção de \textit{feedbacks} para o método avaliativo de respostas discursivas curtas. Através de técnicas de aprendizado semi-supervisionado em \textit{Machine Learning}, o sistema auxilia o tutor na identificação principais respostas para reduzir o esforço de correção. Com os modelos avaliativos em meio computacional, o professor audita os resultados produzidos pelo sistema e acompanha seu processo de decisão. Deste modo, apresentamos a robustez do modelo avaliativo produzido pelo sistema através de diferentes \textit{datasets} da literatura. Em 200 questões com um total de 9892 respostas, alcançamos \textit{Accuracy} média de 60,62\% e \textit{F1} ponderado de 57,84\% em relação aos avaliadores humanos.

\textbf{Palavras-chaves}: Avaliação Automática de Questões Discursivas. Aprendizado Semi-Supervisionado. Sistemas de Apoio ao Tutor. Processamento de Linguagem Natural. Classificação de Texto.
\end{resumo}


% Resumo em inglês.
% (*) Escrever resumo e palavras-chave.
\begin{resumo}[Abstract]
\begin{otherlanguage*}{english}
The evaluation is a basic step that composes the learning assessment and guarantees progress through the planned curriculum. On learning assessment, the application of open-ended questions contributes to developing critical thinking and writing skills. The tutor must adapt your teaching methods at a large scale, avoiding assessment overload. Even though small documents, the answer collection produces a large corpus. Meanwhile, the tutor should analyze details inside these answers to identify potential learning gaps. Therefore, the adoption of educational supporting methods aims to improve the teacher's analytical capacity and, consequently, intensify monitoring of the student's learning. Over these studies, we present an \textit{Active Learning} model for educational document classification, specifically for Short Answer Grading problems. For this purpose, we combine clustering and classification models for pattern recognition with grammatical, morphological, semantic, syntactic, statistical, and sequential analyses for textual enrichment. The system aims to approximate the textual patterns to the tutors' evaluation criteria, reducing their effort and supporting the learning assessment. We apply our method to 65875 students' answers among 255 questions found in Short Answers Graders' literature. According to the human graders, our proposal achieves an average accuracy of 79\% and a weighted F1 of 78\%.

\textbf{Keywords}: Automatic Short Answer Grader. Active Learning. Natural Language Processing. Tutor Support Systems. Text Categorization.
\end{otherlanguage*}
\end{resumo}


% Insere lista de ilustrações.
\pdfbookmark[0]{\listfigurename}{lof}
\listoffigures*
\cleardoublepage

% Insere lista de tabelas.
\pdfbookmark[0]{\listtablename}{lot}
\listoftables*
\cleardoublepage

% Lista de abreviaturas e siglas.
% (*) Preencher com as siglas usadas ao longo do texto e seus significados.
\begin{siglas}
  \item[CAA] \textit{Computer-Assisted Assessment} (Avaliação Assistida por Computadores)
  \item[NLP] \textit{Natural Language Processing} (Processamento de Linguagem Natural)
  \item[ML] \textit{Machine Learning} (Aprendizado de Máquina)
  \item[DL] \textit{Deep Learning} (Aprendizado Profundo)
  \item[AVA] \textit{Ambiente Virtual de Aprendizagem}
  \item[SAG] \textit{Short Answer Grader} (Avaliador de Respostas Discursivas Curtas)
  \item[SAS] \textit{Short Answer Scoring} (Avaliação de Respostas Discursivas Curtas)
  \item[EDM] \textit{Educational Data Mining} (Mineração de Dados Educacionais)
  \item[TF] \textit{Term Frequency}
  \item[IDF] \textit{Inverse Document Frequency}
  \item[LSA] \textit{Latent Semantic Analysis} 
  \item[LDA] \textit{Latent Dirichlet Allocation}
  \item[CBoW] \textit{Continuous Bag-of-Words}
  \item[EaD] Ensino a Distância
  \item[MOOCs] \textit{Massive Open Online Course} (Curso Online Aberto e Massivo)
  \item[HTML] \textit{HyperText Markup Language}
  \item[POS-Tags] \textit{Part-of-Speech Tags}
  \item[NER] \textit{Named Entity Recognition}
  \item[CHS] \textit{Calinski-Harabasz Score}
  \item[DBS] \textit{Davies-Bouldin Score}
  \item[SS] \textit{Silhouette Score}
  \item[SSE] \textit{Sum of Squared Errors}
  \item[CVS] \textit{Coefficient of Variation: Cluster Sizes}
  \item[HS] \textit{Homogeneity}
  \item[CS] \textit{Completness}
  \item[CA] \textit{Clustering Accuracy}
  \item[KNN ou KNRG] \textit{K-Nearest Neighboors} ou \textit{K-Nearest Neighboors Regression}
  \item[DTR ou DTRG] \textit{Decision Tree} ou \textit{Decision Tree Regression}
  \item[SVM] \textit{Support Vector Machine}
  \item[GBC] \textit{Gradient Boosting}
  \item[RDF] \textit{Random Forest}
  \item[WSD, WSRG ou WiSARD] \textit{Wilkes, Stonham and Aleksander Recognition Device}
  \item[ACC] \textit{Accuracy}
  \item[PRE] \textit{Precision}
  \item[REC] \textit{Recall}
  \item[F1 (m, w)] \textit{F-Score (macro or weighted)}
  \item[MAE] \textit{Mean Absolute Error}
  \item[MSE] \textit{Mean Squared Error}
  \item[RMSE] \textit{Root Mean Squared Error}
  \item[LNREG] \textit{Linear Regression}
  \item[LSSR ou Lasso] \textit{Least Absolute Shrinkage and Selection Operator}
  \item[STI] Sistema Tutor Inteligente
  \item[IAT] \textit{Intelligent Assessment Technologies}
  \item[ROUGE] \textit{Recall Oriented Understudy for Gisting Evaluation}
  \item[CNN] \textit{Convolutional Neural Networks}
  \item[LSTM] \textit{Long Short-Term Memory}
  \item[GG-NAP] \textit{Neural Approximate Parsing with Generative Grading}
  \item[BERT] \textit{Bidirectional Encoder Representations from Transformers}
  \item[ELMo] \textit{Embeddings from Language Models}
  \item[GPT] \textit{Generative Pre-trained Transformer}
\end{siglas}

\newpage

% Insere o sumário.
\pdfbookmark[0]{\contentsname}{toc}
\tableofcontents*
\cleardoublepage

\newpage


%%% Início da parte de conteúdo do documento. %%%

% Marca o início dos elementos textuais.
\textual

% Inclusão dos capítulos.
% (*) Para facilitar a organização, os capítulos foram divididos em arquivo separados e colocados dentro da.
% pasta capitulos/. Caso o aluno prefira trabalhar com um só arquivo, basta substituir os comandos \include 
% pelos conteúdos dos arquivos que estão sendo incluídos, excluindo a pasta capitulos/ em seguida.
% ==============================================================================
% TCC - Nome do Aluno
% Capítulo 1 - Introdução
% ==============================================================================
\chapter{Introdução}
\label{cap-intro}

\begin{comment}
O documento é organizado em capítulos (\texttt{\textbackslash chapter\{\}}), seções (\texttt{\textbackslash section\{\}}), subseções (\texttt{\textbackslash subsection\{\}}), sub-subseções (\texttt{\textbackslash subsubsection\{\}}) e assim por diante. Atenção, porém, a não criar estruturas muito profundas (sub-sub-sub-...) pois o documento não fica bem estruturado.
\end{comment}

As avaliações de aprendizado são fundamentais para todos os níveis de ensino. É por meio do método avaliativo que o professor observa o desempenho da turma e seu progresso nos conteúdos. Com aplicações frequentes, as atividades permitem ao professor interagir com os alunos e com os materiais pedagógicos para reformulação e aperfeiçoamento da sua metodologia. Desse modo, é com o acompanhamento da disciplina e o apoio ao educando que as atividades permitem a reformulação e controle do processo de ensino-aprendizagem \cite{barreira2006}. Por meio das atividades podemos identificar o domínio dos estudantes sobre o contexto e sua capacidade de realizar inferências sobre o assunto. O papel da avaliação, portanto, é diagnosticar, apreciar e verificar a proficiência dos alunos para que o professor atue no processo de formação de modo a consolidar o aprendizado \cite{oliveira2005}.


É através do modelo de ensino-aprendizagem, que o professor observa problemas e age para contorná-los. Essa identificação de problemas e sua rápida solução torna a estrutura curricular personalizada, alinhando a turma de acordo com os objetivos da disciplina. É através das atividades, portanto, que é possível mensurar o conhecimento individual dos alunos. Um modo de aperfeiçoar a aplicação das atividades em quantidade e qualidade é dada através da mediação tecnológica. A mediação tecnológica na criação, avaliação, recomendação e visualização em dados educacionais apoia o professor na melhoria e no acompanhamento do currículo do aluno \cite{paiva2012}. É com as ferramentas de apoio, então, que o tutor pode verificar a aptidão dos estudantes, de forma individual ou coletiva, para melhorar a adaptação e a experiência com a disciplina.

Na literatura da Avaliação Assistida por Computadores (\textit{Computer Assisted Assessment} - CAA em inglês), existe uma extensa pesquisa por métodos para avaliação de questões discursivas. Sabendo que existe um critério formulado pelo professor para correção das respostas discursivas, propomos uma abordagem de reconhecimento dos padrões de textuais. Assim, neste trabalho descrevemos um modelo semi-supervisionado para reconhecimento do método avaliativo, extração de padrões textuais, classificação da base de dados e produção de \textit{feedbacks}. Considerando a liberdade textual característica das respostas discursivas, verificamos a similaridade entre respostas e os grupos de termos referenciais para atribuir notas de forma equivalente ao avaliador humano. Com os modelos de SAG, esperamos também demonstrar o método avaliativo com a criação de \textit{feedbacks}, como o quadro de \textit{rubrics} \cite{arter2006} e, consequentemente, melhorar os métodos de avaliação automática \cite{spalenza2016SBIE}.

\section{Problema} 
\label{sec-problema}

Dentro da literatura da avaliação de respostas discursivas curtas, em inglês \textit{Short Answer Grader (SAG)}, encontramos determinados problemas listados pelos autores para criação de melhores modelos avaliativos. Apesar de ser um estudo já realizado há décadas, em SAG encontramos desafios observados durante a aplicação da avaliação automática como demandas importantes e pouco estudadas até o momento. Nos primeiros sistemas, a modelagem de questões discursivas era um trabalho realizado com o texto bruto. A partir disso, a busca por equivalência entre a resposta esperada e o texto dos estudantes falhou por inúmeras vezes na padronização dos documentos e na identificação de sinônimos \cite{leffa2003}. O estudo dessa pesquisa fomentou inúmeras discussões em torno da identificação do conhecimento obtido pelas respostas escritas pelo aluno. A robustez dessa análise é parte fundamental de boa parte dos algoritmos atuais em SAG. Portanto,  para a recuperação da relação entre o conteúdo e a nota atribuída são aplicadas diversas técnicas entre Aprendizado de Máquina, Estatística, Processamento de Linguagem Natural, Reconhecimento de Padrões, dentre outras.

% (Machine Learning - ML)
% (Natural Language Processing - NLP)
% (Pattern Recognition - PR)
% (Educational Data Mining - EDM)

Em uma revisão da literatura sobre os sistemas SAG \cite{burrows2015}, os autores reúnem 37 trabalhos realizados na área. Durante essa revisão, o autor destaca o problema da  profundidade do aprendizado, tradução literal de ``\textit{depth of learning}'', separando as atividades em dois grupos: de reconhecimento e de recuperação. Tais modelos têm diferentes intúitos na aquisição de informação do aluno o que gera diferentes modos de processamento. No Brasil, conhecemos essas por questões abertas e fechadas, nomenclaturas também citadas pelos autores. Essa divisão estabelece a diferença entre as atividades que exploram apenas a necessidade de identificação e organização de conteúdo e as que dependem de construção de ideias visando respostas próprias e originais. Portanto, por definição, a chave para separar atividades que necessitem de maior ou menor conhecimento factual, ou questões abertas e fechadas, é a liberdade que o aluno possui para criação do seu conjunto de resposta.

Um problema com esse viés é reagir às questões discursivas factuais e opinativas da forma adequada \cite{bailey2008}. É esperado que o sistema lide com a liberdade de escrita do aluno recuperando o conteúdo. A forma aqui proposta para contornar esse problema de identificação do critério avaliativo parte de interações com o especialista. Essas interações são requisições de correção para buscar avaliações específicas de padrões textuais das respostas. Esse processo seleciona documentos que indiquem certo grau de distinção por grupo, resultante de uma clusterização inicial \cite{oliveira2014}. Coletamos assim a classificação dada para os itens elencados como relevantes pelo sistema em cada \textit{cluster} para continuidade do processo avaliativo.

Desta forma, a modelagem de classificadores de forma semi-supervisionada permite que encontremos grupos de respostas através da clusterização. A partir daí devemos encontrar qual é o critério avaliativo do professor, sem requisitar exemplos e chaves de resposta, dado até o momento como necessário \cite{butcher2010, mohler2011, ramachandran2015a}. Basicamente, a análise de distribuição de características e a seleção de respostas visa remontar o critério do professor, otimizando o processo de correção automática.

O problema passa da necessidade de avaliação para o aumento do número de padrões à serem avaliados pelo professor para relacionar conteúdo com classificação. Um dos principais sistemas, \textit{FreeText Author}, apresenta problemas importantes para um efetivo processo avaliativo automático \cite{butcher2010}. Os autores, em uma análise detalhada deste sistema, listaram seis problemas. O primeiro que podemos destacar é a omissão dos padrões de avaliação. O segundo, a identificação de associações entre palavras e o método avaliativo de forma inconsistente. O terceiro problema, conforme os autores, é a necessidade de identificação estrutural da sentença. A quarta dificuldade listada é o tratamento de classificações incorretas por parte do especialista. O quinto problema é o conflito de um padrão correto de resposta com uma avaliação dada como incorreta. Por fim, seguindo a linha do quarto e quinto problema citado, um sexto problema é dado pela confiabilidade do sistema como avaliador em relação a interpretação textuais inconsistentes.


É importante descrever ainda a diferença entre o conjunto de informações selecionadas para a classificação correta das respostas e a interpretação avaliativa \cite{ramachandran2015a}. A separação do cunho interpretativo do interlocutor deve ser estritamente analisada conforme o modelo avaliativo. Se os padrões de nota e de resposta estiverem em conformidade com a avaliação do especialista, a redução de dimensionalidade dada pela otimização deve indicar uma boa interpretação da relação termo-classe. Assim, encaramos a avaliação como um processo constante e passível de revisões, para que o ajuste do sistema torne o modelo extraído cada vez mais próximos das expectativas do professor. 

Por fim, podemos ainda citar dificuldade em encontrar os \textit{datasets} utilizados por trabalhos da literatura \cite{burrows2015}. É muito comum encontrar trabalhos no qual os autores coletaram dados na própria universidade e não as tornam públicas. Além disso, em SAG uma base de dados adequada deve caracterizar o processo avaliativo do professor e constar com relevantes resultados na literatura. Entretanto, o intúito deste trabalho é propor algumas soluções para tais problemas. Assim, buscamos avanços desde a descrição dos conjuntos de dados disponíveis até a apresentação de uma possível solução para a associação entre a atribuição de notas e o aspecto textual da resposta.

\begin{comment}
Por conta disso, a ferramenta proposta por neste trabalho faz análises diretas nas respostas enviadas pelos alunos e realiza pré-processamentos para tornar grupos de termos e documentos equivalentes. Alguns módulos foram adicionais para compreender melhor a estrutura frasal e compreender a escrita de cada resposta em níveis gramaticais, sintáticos e morfológicos. A avaliação dada apenas com base na sintaxe, ortografia ou organização não são suficientes para as questões discursivas curtas \cite{ramachandran2015a}.

Após os módulos linguísticos, os demais problemas serão cautelosamente trabalhados através de metodos de recuperação da informação. Assim, o 4º problema esclarece a dificuldade em separar possíveis \textit{outliers} das informações relevantes que foram apresentadas em uma única ou poucas ocasiões no banco de dados. Nosso método de contornar essa adversidade é dada pelo envio de padrões distintos para a avaliação humana, em busca de referências ao conteúdo. Desta forma, a ferramenta trabalha cada nota como um rótulo dado pelo especialista, criando um modelo mais detalhista do que a análise de uma resposta candidata elaborada pelo professor.

No 5º problema, sobre a localização de padrões de avaliação incorretamente classificados pelo especialista, há um impacto grave na adaptabilidade da máquina como avaliador. Em torno disso, interpretamos conjuntos de  avaliação para ajustar seu modelo de classificação, extraindo cada alteração como conhecimento. Assim, o sistema poderá identificar qual informação difere na resposta para o modelo avaliativo construido. Tal processo torna-o mais específico ou genérico segundo a variação da nota atribuida ao documento. O mesmo esperamos que ocorra no último problema listado pelos autores, a confiabilidade do sistema na modelagem do critério de avaliação. Com o ajuste da avaliação pelo professor durante etapas de revisão, esperamos que o sistema modifique seu modelo, refinando-o.

Em relação a issoNeste estudo esperamos realizar comparações em, ao menos, três \textit{datasets} que temos acesso e são conhecidos da literatura e dois \textit{datasets} locais. Nesses \textit{datasets} que temos disponíveis, apresentados em detalhes na Seção \ref{cap5}, encontramos uma variação no padrão de notas (contínuas ou discretas), na avaliação de um ou mais especialistas e na existência ou não de respostas candidatas. Tal variabilidade permite que diferentes testes sejam realizados para investigar o modelo avaliativo do professor, seja ele dado pela nota atribuida ou pela resposta candidata.

\end{comment}

\section{Proposta}
\label{cap1-proposta}

Neste trabalho apresentamos um método de avaliação de respostas discursivas curtas através de modelos avaliativos complexos. Para seu desenvolvimento, buscamos identificar problemas mais comuns descritos na literatura como deficiências dos sistemas \textit{Short-Answer Graders} (SAG) para apresentar uma proposta de solução. A ideia é compor um sistema para reconhecer a relação entre as respostas e as notas atribuídas de acordo com o método avaliativo do professor. Com isso, esperamos atender a demandas atuais dos trabalhos em SAG por meio de técnicas de \textit{Educational Data Mining} (EDM), \textit{Machine Learning} (ML) e \textit{Natural Language Processing} (NLP). Para a criação dos modelos selecinamos diferentes bases de dados de respostas discursivas curtas disponíveis em \textit{inglês} e \textit{português}. Dentre os \textit{datasets} observamos 3 tipos de avaliações: notas discretas, notas graduais e notas contínuas. Portanto, neste trabalho, estudamos estruturas para identificação das principais respostas do conjunto, reconhecimento do método avaliativo do professor (especialista) e elaboração \textit{feedbacks}. 

Para identificação das principais respostas apresentamos um modelo de aprendizado semi-supervisionado. No aprendizado semi-supervisionado o especialista ativamente passa o conhecimento para o algoritmo de classificação. O algoritmo, por sua vez, utiliza o as informações passadas para criar um modelo que imite o especialista na tarefa. Neste caso, o professor ensina ao sistema seu método avaliativo e, através da atribuição de notas, é formado um modelo que tenta replicar o método para as demais respostas da atividades. Cada uma das respostas enviadas para atividade é considerada uma amostra para o sistema. Dentre todas as amostras, é fundamental que o sistema aprenda cada uma das características das respostas, selecionando as principais por representatividade. Para essa seleção o sistema utiliza de técnicas de otimização e clusterização. As respostas selecionadas são denominadas de treinamento, pois serão utilizadas para produção dos modelos, enquanto as demais são o conjunto de teste.

No reconhecimento do método avaliativo do professor, modelos são criados para classificação das respostas discursivas. A categorização deve se aproximar ao máximo da tarefa realizada pelo professor, analisando detalhes parecidos na resposta. Portanto, o modelo avaliativo do sistema objetiva atender as expectativas do professor. Quanto menor a diferença entre a nota dada pelo sistema e a nota atribuída pelo professor, melhor o modelo criado. Consequentemente, os melhores modelos representam melhor a diversidade de notas e respostas com tendência menor de erros. Na gradação das notas, quanto maior a discrepância entre as notas mais críticos são os erros. Sabendo que, entre avaliadores humanos também existe esse erro. Os dados selecionados para treino do classificador ditam o conhecimento da gradação de notas distribuídas por ele. Portanto, o classificador recebe as características de cada resposta e a sua respectiva avaliação e as compara com as amostras de teste, com notas não conhecidas. Portanto, o modelo de classificação, tomado aqui como avaliador, produz as notas complementares para o conjunto de dados de teste.

Por fim, a elaboração de \textit{feedbacks} e relatórios é fundamental para o suporte ao professor. Em sala de aula, os \textit{feedbacks} são um material que detalha a avaliação para professores e alunos e descrevem o método avaliativo de forma a sanar qualquer dúvida e evidenciar qualquer problema no aprendizado. Por outro lado, na perspectiva da interação do professor com o sistema, os \textit{feedbacks} caracterizam a decisão, descrevem o modelo textual e a equivalência entre respostas. Portanto, em todos os ciclos do sistema esperamos reduzir o esforço de correção do tutor, apresentar resultados de alto nível com o modelo avaliativo e gerar materiais explicativos e completementares de qualidade.

\section{Objetivos} \label{cap1-objetivos}

O objetivo deste trabalho, portanto, é ajustar o modelo de correção criado pela máquina aos padrões estabelecidos pelo professor através da sua avaliação. Para isso, os modelos avaliativos devem compreender o método aplicado pelo professor, categorizando as respostas em classes, níveis ou intervalos contínuos de nota. Segundo a consistência de cada grupo, buscamos reduzir o esforço de correção do professor com a avaliação das respostas que apresentem apenas as principais caracteristicas textuais. Através de padrões bem definidos, esperamos reproduzir o critério avaliativo da questão justificando a classe atribuída através do seu respectivo sumário. Tal sumário, então, são os padrões de cada classe de nota partindo do agrupamento \textit{a priori} das questões. É através desse sumário por nota que recuperamos um possível critério de correção. Desta forma, através do \textit{p}Nota, esperamos que o professor esteja apto para gerenciar o seu método avaliativo em um tempo menor para concentrar-se na verificação de aprendizagem do aluno.

Portanto, temos como âmbito principal a criação de modelos para aproximar o critério avaliativo aplicado ao aluno da definição de padrões de correção e a criação de \textit{feedbacks}. Para isso, estudamos os padrões avaliativos do professor e os métodos de representação do conhecimento em base de dados de questões discursivas curtas. Para atingir o objetivo geral descrevemos os seguintes objetivos específicos:

\begin{itemize}
\item Organizar \textit{datasets} públicos e locais para comparação direta com resultados obtidos em estudos correlatos \cite{burrows2015}.
\item Estudar o impacto das técnicas de Processamento de Linguagem Natural e Recuperação da Informação para a identificação da relação termo-classe de forma léxica, morfológica, semântica, sintática, estatística ou espacial \cite{burrows2015, butcher2010}.
\item Unificar padrões de respostas dadas por professores e alunos, observando a frequência de ocorrência e co-ocorrência de termos segundo sua relevância \cite{butcher2010}.
\item Criar modelos avaliativos através do reconhecimento de padrões variáveis em categorias em dados discretos e contínuos \cite{burrows2015}.
\item Elaborar e ajustar modelos de acordo com a eficiência do sistema na recuperação da resposta atribuída pelo professor e seu modelo avaliativo \cite{burrows2015}.
\item Identificar a relação da avaliação com o comportamento textual da classe para remoção de \textit{outliers} e manter a consistência da classificação \cite{butcher2010}.
\item Apresentar avaliações adequadas ao formato de correção do professor \cite{butcher2010}.
\item Gerar \textit{feedbacks} que colaborem com o processo avaliativo, como o quadro de \textit{rubrics}, de forma a contribuir com a discussão de resultados e a representação do critério de correção \cite{oliveira2010}.
\end{itemize}

\section{Estrutura do Trabalho}

A seguir são apresentados os conteúdos dessa tese. A proposta é discutida em detalhes através de 5 capítulos. Para além da Introdução, o trabalho é composto dos seguintes capítulos:

\begin{itemize}
\item \textbf{Capítulo \ref{cap-literatura} - Revisão de Literatura:} Apresenta uma breve revisão da literatura sobre métodos de análise e avaliação de respostas discursivas curtas.

\item \textbf{Capítulo \ref{cap-metodo} - Método:} Define a estrutura do sistema \textit{p}Nota e as formas utilizadas para efetuar de maneira abrangente a análise de respostas discursivas curtas.

\item \textbf{Capítulo \ref{cap-experimentos} - Experimentos e Resultados:} Descreve por meio de oito \textit{datasets} as diferentes formas de apoio avaliativo, modelagem da relação termo-nota e a formação de \textit{feedbacks} utilizados pelo sistema.

\item \textbf{Capítulo \ref{cap-conclusao} - Conclusão:} Discute as contribuições deste trabalho, conclusões extraídas dos resultados obtidos e as perspectivas de trabalhos futuros.

\end{itemize}
% ==============================================================================
% Tese - Marcos Alécio Spalenza
% Capítulo 2 - Revisão da Literatura
% ==============================================================================
\chapter{Revisão da Literatura}
\label{cap-literatura}


A Classificação de Documentos, tradicional área de ML, pode ser subdividida segundo sua motivação ou o conteúdo do conjunto de documentos. Ela envolve treinar algoritmos de classificação com exemplos rotulados para replicar métodos de identificação de conteúdo conforme o especialista \cite{baeza2011}. Cada conjunto de documentos pode ser chamado também como \textit{dataset}, base de dados ou \textit{corpus}. A coleção destes, porém, é denominada \textit{corpora}. Portanto, para além da origem e do conteúdo dos documentos, o algoritmo deve se adaptar à especialização na triagem dos documentos de acordo com suas características.

O especialista realiza uma leitura dos documentos e identifica informações específicas que justificam a categoria atribuída. Para replicar tal tarefa, por meio da análise do conteúdo, o sistema deve identificar características que estão diretamente relacionadas à classe que será atribuída. Dependendo da característica dos documentos, o conteúdo relevante pode incluir a identificação de poucas palavras-chave até a formação de modelos linguísticos complexos \cite{jurafsky2009}. Isso acontece também com os SAG, aplicando análises complexas da relação textual para atribuição de notas \cite{paiva2012, yang2021}.

Desse modo, a atribuição de notas torna os SAG uma complexa tarefa de classificação de documentos. Para um modelo SAG é essencial a adaptação do algoritmo de acordo com o método de classificação utilizado pelo especialista. A subjetividade do critério de avaliação deve ser levada em consideração apesar do conteúdo textual \cite{pado2021}. A combinação entre o reconhecimento do modelo avaliativo e do modelo textual deve atender às expectativas do professor na avaliação \cite{condor2020}. Enquanto em parte das atividades as notas podem ser fortemente correlacionadas com a ocorrência dos termos, em outras o critério pode ter alto nível de subjetividade \cite{azad2020}. Então, para a construção de um SAG, são aspectos determinantes a análise contextual das respostas e a compreensão do formato avaliativo do professor \cite{mohler2011}.

\section{\textit{Computer-Assisted Assessment}}

A sala de aula é um ambiente rico em conteúdo. As informações produzidas são descritores para o desempenho dos estudantes em sala. A análise desses dados é essencial para acompanhar o aprendizado dos alunos, verificar a necessidade de reforço do conteúdo e monitorar o cumprimento do curricular \cite{sweta2021}. Tradicionalmente essa dinâmica faz parte dos métodos de ensino-aprendizagem empregados pelos professores, porém, superam a sua capacidade analítica \cite{madero2019}. Por conta disso, ganharam maior notoriedade e espaço prático os sistemas de EDM, amplificando a análise dos materiais produzidos em sala \cite{siemens2012, romero2010}.

Em EDM, a aquisição de conhecimento aplicado a dados educacionais visa o apoio e acompanhamento do ensino \cite{ferreira-mello2019}. A consequência da mineração de dados nesse cenário é uma expressiva redução da carga horária do professor voltada para o acompanhamento coletivo e individual \cite{sweta2021}. Essa redução ocorre com o professor passando para papéis de monitoramento e auditoria dos resultados.

Portanto, via mineração de dados, são possíveis a análise de todo o material produzido pelos alunos, a criação de \textit{feedbacks} individuais e a aplicação de reforço para determinados grupos de estudantes. Os SAG, uma das áreas de estudo em EDM, estão diretamente associados a essas três características \cite{burrows2015}. Os SAG são responsáveis pela avaliação em massa de respostas textuais curtas, replicando o critério avaliativo do professor. Os SAG fazem parte de um grupo de técnicas computacionais para apoio aos métodos avaliativos, conhecidos pelos estudos em CAA \cite{perez-marin2009}.

A verificação do aprendizado nas questões discursivas curtas contribuem para identificar se os estudantes assimilaram ou não o conteúdo ministrado em sala \cite{oliveira2013}. Além disso, a aplicação desse tipo de atividade é base para prática da escrita, busca de informações e sumarização de conteúdo. Desta forma, tais atividades realizam uma função importante para todos os níveis de ensino, principalmente durante o desenvolvimento da escrita \cite{johnstone2002}. 

Com a alta carga-horária do professor, existe baixo índice de aplicação desse tipo de questão, mesmo diante de sua relevância \cite{bilgin2017}. Assim, considerando o tempo em sala juntamente com os esforços do planejamento, a revisão e a análise das atividades acabam sendo tratadas como secundárias. O apoio computacional nessa tarefa reduz o tempo que é necessário para avaliação do conteúdo fora de sala de aula, com o professor participando parcialmente da atribuição de notas \cite{ming2005}. Nesse processo, os sistemas reproduzem o critério avaliativo do professor, com este garantindo a coerência da avaliação, fazendo o sistema seguir fielmente seu critério. Adicionalmente, as aplicações de ML nesse cenário, produzem modelos que descrevem a atribuição de notas, formando \textit{feedbacks} que podem ser aplicados diretamente em sala de aula \cite{butcher2010, bernius2022}. 

No entanto, para interpretação computacional, as questões devem ser elaboradas com objetividade \cite{bailey2008}. Nesse aspecto, dentro de um tema, deve ser possível identificar alinhamento entre as respostas, definindo se cada uma segue ou não o que compõe o critério avaliativo. Assim, as questões discursivas \cite{bezerra2008} envolvem a liberdade de escrita dos estudantes na formulação das respostas. Na Figura \ref{fig-atividades} são caracterizadas as formas de atividades segundo seu modelo de resposta \cite{spalenza2017}.

\begin{figure}[!h]
\centering
\includegraphics[width=\textwidth]{figuras/tiposAtividade}
\caption{A extração da informação e os tipos tradicionais de atividade aplicados no cotidiano de sala de aula.}
\label{fig-atividades}
\end{figure}

Como apresentado na Figura \ref{fig-atividades} o professor dispõe de alguns modelos de atividades que, refletem diferentes aspectos do aprendizado. Há quesões que são abertas e com pouca ou nenhuma restrição como as redações \cite{almeida-junior2017} ou respostas fechadas para uma única palavra, guiadas pelo enunciado. As respostas discursivas encontram-se em âmbito intermediário \cite{bailey2008}. As respostas curtas, por sua essência, visam estabelecer a relação entre o conhecimento do aluno e o conteúdo encontrado no material didático. Na Figura \ref{fig-SAG-concepts} é demonstrado o espectro de questões trabalhados por meio das respostas discursivas curtas \cite{spalenza2017}.

\begin{figure}[!h]
\centering
\includegraphics[width=\textwidth]{figuras/aprendizadoSAG}
\caption{Extração de informação em questões discursivas: entre respostas pequenas não-convergentes e a subjetividade das competências na avaliação de redações.}
\label{fig-SAG-concepts}
\end{figure}

A Figura \ref{fig-SAG-concepts} posiciona as respostas curtas enquanto um nicho das questões discursivas. O ideal é que, entre os conhecimentos, a questão deve evitar abordar temas de cunho interpretativo ou temas individuais, que tangenciam experiências específicas de cada aluno \cite{siddiqi2008}. A representação da resposta deve ser completa, dando embasamento para a correção e evitando informações restritas ou codificadas \cite{ding2020}. O enunciado deve guiar o aluno, de forma que as respostas sejam convergentes \cite{suzen2020, filighera2020}. Para isso, é fundamental que o sistema realize ao menos três etapas. A primeira etapa é o aprendizado do modelo de respostas do aluno \cite{ramachandran2015b}. A segunda etapa é reconhecer o padrão avaliativo do professor por meio do modelo de respostas \cite{funayama2020}. A terceira etapa é replicar o modelo avaliativo e elaborar \textit{feedbacks} coerentes \cite{fowler2021}.


\section{\textit{Active Learning}}

Em ML, o aprendizado ocorre com a formação de conhecimentos a partir da interpretação dos padrões \cite{bishop2006}. Esse procedimento dita a forma de aquisição de conhecimento do sistema para treinamento de modelos, buscando desempenho similar ao humano. Neste trabalho apresenta-se um método de amostragem por \textit{Active Learning} \cite{miller2020, kumar2020}, com anotação iterativa do professor em itens selecionados por meio das etapas de clusterização \cite{horbach2018}. Presentes em uma série de sistemas SAG, as técnicas de clusterização utilizam \textit{Unsupervised Learning} para avaliação com base na similaridade das amostras \cite{basu2013, zhang2016, marvaniya2018}. Porém, os trabalhos na literatura geralmente utilizam \textit{Supervised Learning}, criando modelos por meio de uma série de amostras pré-avaliadas pelo especialista.

Portanto, a grande maioria dos estudos utiliza o particionamento entre treino e teste dos dados, de acordo com o determinado em cada \textit{dataset}. Considerando cada resposta dos estudantes uma amostra, o particionamento em treino e teste reflete a divisão \textit{a priori} do conjunto de dados em um grupo para criação do modelo e outro para avaliação \cite{heilman2015}. Esse formato clássico permite ao sistema observar apenas uma parcela dos dados para reconhecimento dos padrões, realizando a inferência nas demais amostras até o momento desconhecidas \cite{bishop2006}. Assim, esses sistemas utilizam um conjunto de treino para extração do critério avaliativo, criando o modelo para replicar o método de atribuição de notas, pressupondo a equivalência deles.

O conjunto de amostras de treino e teste não tem necessariamente a mesma origem \cite{sung2019a}. O uso dos sistemas SAG pelo professor compreende sua aplicação durante a disciplina. Portanto, uma atividade avaliada com um modelo SAG pode ser utilizada para uma série de aplicações, em um diferente momento e com outro grupo de estudantes. Com uma nova iteração, outros tópicos podem ser levantados em cada resposta. A tendência nesses casos é a avaliação incorreta de parte dos dados por conta da rigidez do conjunto fixado para o treinamento.

Alguns métodos também são baseados em exemplos da resposta-alvo, denominadas respostas candidatas \cite{banjade2015, roy2016}. As respostas candidatas, são amostras elaboradas pelo professor e anotadas para representar seus padrões avaliativos. Os sistemas SAG com base nesse tipo de dado buscam, em geral, a comparação direta entre as respostas e o índice de sobreposição \cite{jimenez2013, kar2017, zhang2020}. Porém, as limitações dos dados em análise são um contraponto à liberdade textual das questões discursivas curtas. Esse tipo de treinamento gera uma tendência na avaliação, com limitações na capacidade de o modelo interpretar conteúdos adversos nas respostas dos alunos \cite{ramachandran2015a}. Além de tornar-se engessada, a resposta candidata também não garante o alinhamento para os demais documentos do \textit{dataset}.

Nesse contexto, para contornar parcialmente as limitações, ainda existem alguns métodos que utilizam mais informações sobre a atividade na etapa de treinamento, como por exemplo o enunciado, o material de apoio e o quadro de \textit{rubrics} \cite{ramachandran2015b, wang2019}. O enunciado e o material de apoio adicionam ao sistema conhecimento externo sobre o tema. Já as respostas candidatas e o quadro de \textit{rubrics} são materiais descritivos do modelo avaliativo do professor para todos, inclusive o próprio SAG \cite{mizumoto2019, marvaniya2018}. Existem ainda sistemas que precisam de mais detalhes sobre a avaliação, com a confecção de regras e filtros de conteúdo \cite{butcher2010, pribadi2017}.

Para lidar com a diversidade textual também são empregadas estratégias de \textit{Data Augmentation}. Com \textit{Data Augmentation} as amostras passadas como treinamento são combinadas para representar de forma mais complexa o modelo avaliativo. O uso do aumento de dados torna os sistemas tradicionais um pouco mais robustos a alterações e mudanças nos padrões básicos, reduzindo a ocorrência de classificações tendenciosas \cite{kumar2019, lun2020}. Assim, a quantidade de amostras para treinamento de variações para cada modelo de resposta torna-se muito superior à quantidade dada inicialmente.

Diferentemente das técnicas citadas, o método proposto de \textit{Active Learning} prioriza a seleção das principais amostras para otimização do esforço de anotação \cite{kumar2020}. A proposta combina os métodos de \textit{clusterização} \cite{spalenza2019} e classificação \cite{oliveira2014} para identificação iterativa dos diferentes tópicos abordados nas respostas. Assim, a evolução do conjunto de dados acontece durante cada uma das iterações. A clusterização, via \textit{Unsupervised Learning}, não recebe dados anotados e extrai as respostas distintas com base no nível de similaridade \cite{everitt2011}. E a classificação, via \textit{Supervised Learning}, coleta as anotações do especialista nas amostras selecionadas para treinamento do modelo \cite{bishop2006}. A partir daí, o modelo treinado replica a avaliação para as demais respostas.


\section{Processamento de Linguagem Natural}

Para criação de um modelo linguístico, os sistemas utilizam técnicas de NLP como estratégias de aquisição de informação. As primeiras técnicas de SAG da literatura e os primeiros sistemas propostos utilizavam descritores \cite{galhardi2018a}. Os descritores são características simples extraídas segundo o formato da escrita de cada documento. Em geral, são formados por características predefinidas, de acordo com a estrutura da resposta do aluno, sem levar em consideração a profundidade do conteúdo \cite{mohler2009}. Entre os descritores, os mais comuns eram a contagem de erros da linguagem, a quantidade de palavras e a frequência de certas classes gramaticais \cite{ galhardi2018b, riordan2019}. Porém, tais características predefinidas não atendem a uma grande quantidade de respostas, criando modelos linguísticos com pouca aderência ao conteúdo.

Posteriormente, surgiram estruturas para maior aquisição de informação e modelagem linguística ao observar os diferentes propósitos das questões discursivas curtas e sua aplicação multidisciplinar \cite{saha2018, kumar2019}. Nesse novo cenário, os modelos linguísticos ampliaram a aderência do sistema ao tema das atividades. Por meio do conjunto de respostas, os sistemas começaram a elaborar modelos linguísticos contextuais, até o momento suficientes para encontrar associações entre as palavras \cite{tan2020}. A partir dessas associações, os sistemas estabeleceram as primeiras conexões entre os termos de cada respostas e o critério avaliativo do professor \cite{sahu2020}.

A evolução das estratégias, agora voltadas para análise do texto por completo, adicionou mais informações aos SAG. Porém, a resposta é dada por detalhes do conjunto de respostas, sendo o todo não necessariamente relevante para avaliação. Nesse aspecto, podemos citar como adições importantes as técnicas de seleção de características, ponderação e reconhecimento de padrões \cite{banjade2016}. Para ponderação textual o modelo mais comum é o Term Frequency - Inverse Document Frequency (TF-IDF) \cite{baeza2011}. O TF-IDF é um método clássico que realiza a ponderação de acordo com a frequência dos termos, equilibrando a relevância de cada termo segundo sua ocorrência nos documentos e no \textit{dataset} \cite{sultan2016}. Entretanto, com a evolução dos métodos neurais, ficaram em evidência as \textit{word embeddings} \cite{jurafsky2009}. As \textit{embeddings} são modelos linguísticos de grande dimensionalidade adquiridos de uma coleção de documentos \cite{goldberg2017}. Esses modelos analisam a conexão semântica dado o emprego conjunto dos pares de termos em \textit{corpora} de larga escala. Assim, de forma pareada, os sistemas avaliam o nível de correspondência dos termos pelo contexto. A partir disso, os sistemas SAG avaliam a proximidade entre as respostas dos estudantes para diferentes termos, frases e contextos \cite{sung2019b, ghavidel2020, galhardi2020, haller2022}.

Por outro lado, na seleção de características e sumarizalção de conteúdo, se destacaram as técnicas como o \textit{Latent Semantic Analysis} (LSA) \cite{landauer1998} e o \textit{Latent Dirichlet Allocation} (LDA) \cite{blei2003}. O LSA é uma das mais utilizadas na literatura \cite{basu2013, sahu2020}. O uso dessa técnica compreende identificar relações semânticas dentro do conjunto de respostas \cite{mohler2009}. Assim, com o LSA, os sistemas reúnem o conteúdo que potencialmente contém maior significância no tema. O mesmo intuito é compartilhado pelo LDA. Esse algoritmo utiliza a análise probabilística para \textit{ranqueamento} dos termos encontrados no texto segundo sua identificação com grupos de documentos. No âmbito dos SAG, essa técnica de extração de tópicos, é utilizada para agrupamento pelas referências encontradas em cada grupo de nota \cite{basu2013, zhang2022}.

Entretanto, nesse nível, os modelos linguísticos criados pela frequência dos termos de cada resposta dos estudantes ainda não refletem uma análise complexa tal qual a do especialista. Portanto, na literatura existem estudos que propõem maior extração de informação textual, ainda que em textos curtos, para formação de componentes linguísticos mais robustos \cite{saha2018, zesch2018}. Assim, foram realizadas análises da estrutura textual segundo suas camadas de construção, sejam elas sintática, semântica, léxica, morfológica ou gramatical \cite{ramachandran2015b, roy2016}. Ainda nessa linha, alguns estudos também remontam o conteúdo das respostas sob a perspectiva sequencial da construção textual \cite{kumar2017}. Essas sequências subdividem cada resposta em pequenos trechos que contêm de um a \textit{n} termos para aplicar na análise de equivalência e sobreposição entre respostas \cite{jimenez2013, sakaguchi2015, sultan2016}.

Os modelos de DL também contribuíram nessa linha. Utilizando técnicas de \textit{Continuous Bag-of-Words} (CBoW) ou \textit{skip-gram} e suas derivações, estes construíram representações de padrões mais complexos da vizinhança dos \textit{tokens} \cite{mikolov2013}. Essas técnicas buscam a identificação contextual em segmentos do conteúdo com a atribuição de pesos ponderando sua relevância. Assim, as redes neurais substituíram boa parte das estratégias de quantificação de equivalência e a aplicação das métricas de sobreposição \cite{haller2022}. Através das redes, foram incorporadas melhores formas de identificar a ocorrência dos segmentos de termos. Com tais segmentos, são aplicadas enriquecimentos estruturais e semânticos na análise documental \cite{camus2020}. No entanto, uma dificuldade encontrada na construção desses SAG são os \textit{datasets} com poucas amostras para treinamento \cite{bonthu2021}. Com o avanço dos s SAG, a combinação dos aspectos de NLP que investigam a forma e a construção de cada sentença deve contemplar também tais representações de vizinhança das respostas \cite{riordan2019, kumar2019}.


\section{Avaliadores de Questões Discursivas Curtas}

Os sistemas SAG, para uma análise documental complexa, são compostos por um conjunto de métodos que incluem a criação do modelo linguístico, a organização do conhecimento e a identificação de características relevantes. Apesar disso, uma parte fundamental dos sistemas SAG são os classificadores de alta qualidade \cite{funayama2020}. Portanto, são os classificadores que destacam o conhecimento adquirido nas etapas anteriores e o aprendizado do modelo avaliativo \cite{mohler2011}.

O propósito do classificador é compreender, replicar e descrever o modelo do professor (especialista) \cite{yang2021}. Assim, é função do sistema identificar características relevantes para assimilar a forma que o professor avalia cada resposta enviada pelos estudantes \cite{jordan2012, mao2018}. Em geral, os avaliadores automáticos são divididos segundo quatro diferentes técnicas: por mapeamento de conceitos, extração de informação, análise de \textit{corpus}, algoritmos de ML \cite{burrows2015}. 

O método de mapeamento de conceitos consiste em um processo de detecção de determinado conteúdo nas respostas produzidas pelos estudantes. O reconhecimento de conteúdo, portanto, é realizado com análise de alinhamento entre termos de respostas \cite{jimenez2013, zhang2020}. Nesse método avaliativo, a principal característica é a existência dos conceitos nas respostas de maior grau de nota \cite{kar2017, chakraborty2017}. Porém, mesmo com a construção automática de padrões através da amostragem, não é garantida a consistência dos modelos produzidos \cite{azad2020}. Desse modo, tendo como principal fator a compatibilidade entre respostas, o mapeamento de conceitos tende a ser muito dependente do objetivo da questão e do conteúdo do conjunto de respostas \cite{filighera2020}.

Já a extração de informação, apresenta sistemas caracterizados por um primeiro contato com estratégias de identificação factual e contextual. Nesses modelos, existe uma evolução dos métodos para análise do conteúdo, sendo compostos por técnicas de reconhecimento de padrões e séries de expressões regulares \cite{ramachandran2015b, butcher2010}. Assim, sistemas SAG com base na extração de informação apresentam uma coleção de padrões em análise para o alinhamento entre a resposta e a expectativa do professor \cite{tan2020}. Então, o modelo de avaliação utilizado se aproxima da leitura do professor. Porém, até aqui, essas técnicas atendem apenas aos modelos predefinidos.

Os métodos baseados em \textit{corpus} distinguem-se pelo uso da análise estatística com base na frequência dos termos em cada conjunto de dados \cite{kumar2019}. Nesse método, os sistemas utilizam a linguagem para validação do alinhamento entre respostas, interpretação de variações de uso e caracterização do conteúdo \cite{ziai2012, menini2019}. Para além dos termos, essas técnicas aplicam adição de informação para maior diversidade semântica, tornando modelos mais flexíveis para análise do vocabulário do material \cite{fowler2021}.

Apesar da consistência dos modelos anteriores, em um âmbito geral existem limitações para aplicação de cada uma das técnicas em diferentes \textit{datasets} \cite{riordan2019, ding2020}. Em geral, as representações do critério avaliativo por parte do especialista por si não retratam bem o conhecimento para sua reprodução pela sistema \cite{filighera2020}. Em contraste aos modelos superficiais, as técnicas de ML foram incorporadas na análise textual para criação de modelos mais robustos, com fundamentação estatística \cite{galhardi2018b}. Assim, esses modelos visam compreender o conteúdo dos documentos, pelas diferentes componentes textuais, para realizar o reconhecimento de padrões \cite{suzen2020}. Distintamente das técnicas baseadas em regras e expressões regulares, os modelos de ML são capazes de se adaptar a diferentes temas e modelos de resposta \cite{zhang2016, saha2019, camus2020}. Assim, a capacidade de adaptação desses modelos permite a associação de padrões não convergentes. Como consequência, mesmo com amostras divergentes, existe a formação de critérios mais complexos para atribuição da nota.

Em geral, um objetivo dos sistemas SAG, descrito pela literatura, é mesclar os métodos e suas dinâmicas de aprendizado para evolução do modelo avaliativo \cite{burrows2015, zesch2018}. Dessa maneira, é essencial a construção de modelos que reproduzam com alta qualidade a atribuição de notas realizada pelo especialista \cite{jordan2012}. Apesar das dificuldades e dos detalhes subjetivos da avaliação \cite{roy2018}, o intuito é que o desenvolvimento do SAG compreenda a relação entre diferentes características de avaliação e a capacidade de atender diferentes domínios \cite{sung2019a, saha2019}. Dessa forma, esperamos o desenvolvimento de sistemas mais robustos, que compreendam o vínculo entre o critério avaliativo e a escrita dos estudantes.

% ==============================================================================
% Tese Marcos A. Spalenza
% Capítulo 3 - Proposta do Trabalho
% ==============================================================================
\chapter{Método}
\label{cap-metodo}

Neste trabalho, apresentamos o \textit{p}Nota, um SAG que aplica \textit{Active Learning} para análise da relação entre o conteúdo das respostas dos estudantes e o método avaliativo do professor. Acompanhando o desenvolvimento recente da literatura dos sistemas SAG \cite{burrows2015, bonthu2021, haller2022}, identificamos pontos sensíveis e problemas descritos nesses estudos. Utilizamos como base os fundamentos de análise documental e modelagem do método avaliativo do tutor para a criação de uma proposta SAG. Com esse direcionamento, é possível verificar os principais métodos para análise das componentes textuais para elaborar um conjunto robusto de informações sobre cada resposta. O conhecimento das respostas é vinculado ao critério de avaliação do professor. Com isso, espera-se construir modelos que maximizem os resultados na atribuição de notas, aproximando-se do formato de correções do especialista. A estrutura do \textit{p}Nota é particionada em módulos responsáveis por diferentes etapas do processo, como apresentado na Figura \ref{fig-esquema}.

\begin{figure}[!h]
\centering
\includegraphics[width=\textwidth]{figuras/estrutura-pNota.png}
\caption{Esquema do \textit{p}Nota dividido em seus quatro módulos.}
\label{fig-esquema}
\end{figure}


Desse modo, tal qual ilustrado na Figura \ref{fig-esquema}, o sistema é composto por quatro módulos. Além destes, outros três processos dependem do estado do documento na avaliação. O primeiro módulo é a \textit{Extração das Componentes Textuais}, que realiza os ciclos de coleta de dados, verificação textual, extração da informação e organização do conhecimento. Nesse módulo o sistema analisa cada resposta com aplicação de tratamentos textuais para padronização e aquisição de conhecimento. O resultado dessa etapa é um conjunto de vetores de documentos com múltiplos níveis de análise da linguagem empregada.

O segundo módulo é composto pelas técnicas de \textit{Particionamento do Conjunto de Respostas}. Nesse núcleo são empregados métodos de otimização em \textit{clustering} para uma seleção representativa do conteúdo textual. Tais técnicas são descritas em detalhes na Seção \ref{sec-amostragem}. A representação criada pelas amostras é o que define o aprendizado do sistema. Por conta disso, as amostras são escolhidas pela sua distribuição espacial \cite{salton1975, baeza2011}, buscando incluir todos os tópicos abordados no tema da questão. Aqui, aplicam-se técnicas de otimização selecionando o agrupamento com melhor desempenho nos índices qualitativos. 

A próxima etapa recebe as atividades particionadas em \textit{clusters} e com a atribuição de notas nas amostras selecionadas. Neste terceiro módulo, com a construção dos \textit{Modelos Avaliativos}, é realizada a calibração dos classificadores e a atribuição das demais notas. A calibração dos algoritmos busca refinar o critério avaliativo para compreender qual é a relação entre os termos e a nota resultante. Ao fim dessa etapa, as notas geradas colaborativamente entre professor e sistema são encaminhadas aos alunos.

Quando os resultados estão prontos, o sistema atua na etapa de \textit{Relatórios e Feedbacks}. Nesse ponto, todas as notas já foram atribuídas e é possível atuar na transparência do modelo avaliativo. Assim, com o conjunto de informações utilizadas durante os processos, são produzidos relatórios e \textit{feedbacks} que descrevem as notas atribuídas e os resultados do \textit{dataset}. Por fim, os relatórios proporcionam acesso ao formato da amostragem, distribuição de notas, análise de desempenho e descrição dos padrões de resposta. 

Antes da execução do sistema, existe a aplicação em sala de aula. Por meio dos AVA, o \textit{p}Nota busca acompanhar a evolução das salas de aula digitais, com o \textit{Ensino a Distância} (EAD) e a disseminação dos MOOCs \cite{mohapatra2017}. Para isso, utiliza-se um \textit{framework} de coleta para transferência e controle das atividades da sala virtual para processamento externo \cite{spalenza2018}. Portanto são responsabilidades da aplicação a coleta das atividades no ambiente virtual, a transferência para um servidor de processamento e o envio de resultados para o professor. Na Figura \ref{fig-framework} é apresentado o funcionamento do método de coleta de dados em diferentes plataformas de ensino.

\begin{figure}[!h]
\centering
\includegraphics[width=\textwidth]{figuras/framework-moodle.png}
\caption{\textit{Framework} utilizado para transferência de dados, interligando plataformas AVA e o servidor do \textit{p}Nota.}
\label{fig-framework}
\end{figure}

Na Figura \ref{fig-framework} é apresentada a forma empregada na extração de informação dos AVA. Com a configuração, o módulo acessa cada cliente e transfere as atividades para o servidor de processamento do \textit{p}Nota. Então, o \textit{p}Nota solicita via AVA as requisições de avaliação e, após os resultados, também envia as notas para a plataforma. Adicionalmente, os \textit{feedbacks} gerados também são encaminhados individualmente a cada aluno. Na plataforma, após a apresentação das notas, o professor também pode realizar os ajustes necessários caso o resultado não esteja totalmente alinhado ao seu critério.

O professor, controla via plataforma o processamento, sendo aberto para determinar a finalização da atividade diretamente nela. O professor é livre para realizar alterações de qualquer nota mesmo que ainda esteja em análise pelo sistema. No controle do processo avaliativo, o professor fica responsável por monitorar a atribuição de notas e ajustar os resultados propostos pelo sistema.


\section{Extração das Componentes Textuais}
\label{sec-componentes-textuais}

A primeira etapa, \textit{Extração das Componentes Textuais}, realiza o carregamento e a análise do conteúdo textual. Inicialmente é realizada a leitura dos dados, carregando o conjunto de respostas que forma a atividade. É fundamental para extração que o arquivo seja recebido da forma como foi escrito pelo aluno na plataforma. Por conta disso, na sequência é realizada uma série de pré-processamentos que efetuam a limpeza destes documentos, com padronização, segmentação, filtragem, transformação e vetorização. O resultado após essa série de processos é a informação extraída, no formato de vetores com as componentes textuais de cada documento. Na Figura \ref{fig-ect} são apresentados os processos que compõem essa etapa.

\begin{figure}[!h]
\centering
\includegraphics[width=\textwidth]{figuras/esquema-ect-pNota.png}
\caption{Detalhe do módulo de \textit{Extração das Componentes Textuais} no esquema do \textit{p}Nota.}
\label{fig-ect}
\end{figure}

Como é mostrado na Figura \ref{fig-ect}, realizamos todo o tratamento do texto nessa etapa, com coleta das informações. A padronização é composta pela coleta do conteúdo da resposta, remoção de dados não-informativos e aumento da equivalência entre as ocorrências dos termos. A segmentação efetua o particionamento das respostas em segmentos (\textit{tokens}) para verificação de cada termo. Com as frases particionadas, a filtragem seleciona as palavras com maior potencial de ter vínculo com o conteúdo. A transformação realiza análise do conteúdo para aquisição de informação em outros níveis de linguagem. Por fim, com documentos em padrões textuais realiza-se a vetorização, gerando vetores com cada \textit{token} ou série de \textit{tokens} representando o conteúdo enviado pelos alunos.

As respostas, que são recebidas em formato bruto, partem então para pré-processamentos e análises de conteúdo durante essa etapa. Aqui é fundamental a adição dos níveis de compreensão linguística, para amplificar a capacidade de aprendizado dos algoritmos nas próximas etapas. Assim, é pela forma que o texto é passado no formato vetorial \cite{baeza2011} que são identificadas respostas compatíveis até o passo da \textit{Avaliação}.


\subsection{Padronização}
\label{subsec-padronizacao}

% STD_MTH = ["accents", "punct", "spaces", "tags"]

Após o documento ser enviado pelo aluno, o conteúdo do(s) arquivo(s) está em estado bruto. No estado bruto, o conteúdo não segue padrões, em especial de codificação, espaçamento, acentuação e pontuação. É necessário portanto que o formato bruto chegue ao nível de escrita que o aluno enviou ao professor. Além disso, é fundamental remover conteúdos não interpretáveis, como caracteres não alfanuméricos e \textit{tags} (marcações). Portanto, essa etapa é composta pelos seguintes processos:

\begin{itemize}
	\item Remover acentuação;
	\item Remover caracteres não-alfanuméricos;
	\item Remover pontuação;
	\item Remover espaços extras;
	\item Remover marcações.
\end{itemize}

Após cada um dos passos, o texto do aluno está normalizado, tornando possível sua manipulação em nível de conteúdo. Pode-se inferir que a navegação no conteúdo só é possível após a remoção dos ruídos. São usados como exemplos as tags \textit{HTML} e a acentuação. Por um lado, os sinais da linguagem, como acentuação, são fundamentais para leitura e pronúncia dos termos. Mas estes são irrelevantes para identificação dos termos pelo sistema. Por outro lado, o inverso ocorre com marcações de arquivos textuais. São estruturas para leitura do sistema, mas não fazem parte do conteúdo produzido pelo estudante. Em ambos os casos, não existe qualquer relação desses dados com a semântica das respostas, e consequentemente, eles são removidos durante o processo. 

Os ruídos são muito comuns nos textos produzidos na internet, causados pela transferência de arquivos em repositórios externos, \textit{crawlers} ou \textit{web services} \cite{han2011}. Portanto, é a remoção de dados que aproxima a interpretação computacional proposta do envio do estudante na plataforma.

\subsection{Segmentação}
\label{subsec-segmentacao}

% TKN_MTH = ["simple", "word", "regex", "informal"]

Com os documentos passíveis de interpretação, e bem próximos ao que foi enviado pelo aluno, começa-se uma análise detalhada de seu conteúdo. Este é iniciado com o particionamento de cada texto em segmentos menores. A segmentação divide em menores componentes de resposta, seja por caracteres, frases ou parágrafos. Cada particionamento, no entanto, é apenas uma forma de entrada para os procedimentos realizados na sequência. Enquanto parte dos processos faz uso do texto em formato de segmentos de palavras, outros fazem análise contextual, comumente aplicada em formato de segmentos de frase.

É mais comum o formato de segmentos de palavras, denominados \textit{tokens}. Em todos os casos os segmentos são extraídos com base em uma \textit{heurística}, que delimita cada segmento. A \textit{heurística} mais comum para \textit{tokenização} é a divisão pelo espaçamento, eliminando os espaços em branco e considerando as palavras. Porém, esses métodos simples são sujeitos a muitas falhas. Nesses casos, são melhores os métodos construídos especificamente para a linguagem, considerando formas específicas de pontuação e divisões textuais. A \textit{tokenização}, então, é o método que transforma o conteúdo em uma lista de palavras.

A sequência de palavras permite que os próximos níveis trabalhem a perspectiva de cada \textit{token} dessa lista ou de sua vizinhança. Mesmo assim, é muito comum que, durante o processo, o documento seja manipulado de diferentes formas, inclusive passando várias vezes pela transformação de texto em lista de \textit{tokens} e vice-versa. Nesse formato, os \textit{tokens} permitem que haja navegação pelos termos adjacentes tal qual a análise dos termos de forma independente.


\subsection{Filtragem}
\label{subsec-filtragem}
% FLT_MTH = ["smallwords", "stopwords", "largewords", "numerals"]


A filtragem de conteúdo é uma componente muito importante desse processo. Apesar de ser uma etapa que causa perda na informação inicial dos \textit{sets} de resposta, a proposta é identificar \textit{features} que adicionam pouco ou nenhum dado relevante. É esperado, que a inerente perda de informação cause melhoria na consistência e na equivalência entre os documentos. Guiada pelo sistema, a limpeza representa itens que têm baixa correlação com o tema, não sendo componentes do núcleo das respostas. Assim, podemos incluir os seguintes componentes como parte desse processo:

\begin{itemize}
	\item Remover palavras pequenas (menores que três caracteres);
	\item Remover palavras grandes (maiores que 15 caracteres);
	\item Remover \textit{stopwords};
	\item Remover números.
\end{itemize}

Com a filtragem, busca-se uma avaliação fortemente ligada ao tema e o emprego contextualizado. É necessário remover os demais termos, que seriam de baixa significância, pouca capacidade de interpretação e menor relação com o que conteúdo. Esse é o caso das \textit{stopwords}. As \textit{stopwords} são palavras que são empregadas na linguagem como conectivos e não estão conectadas com o conteúdo passado. Elas são extremamente importantes para a nossa interpretação e reconhecimento de contexto, mas não adicionam informação quando empregadas. Assim, a lista de \textit{stopwords} é composta por palavras fundamentais para a linguagem durante a comunicação, mas sem potencial para a análise do contexto.

No caso do tamanho das palavras ainda há uma situação adicional para além da aderência ao tema. A filtragem garante que possíveis ruídos que escaparam dos processos anteriores sejam removidos. Casos específicos como caracteres isolados, links e problemas de segmentação na origem podem gerar ruídos ainda nessa parte. Inclusive, sem uma análise matemática complexa, a verificação numérica também entra em boa parte desses casos e pode ser incluída na filtragem. Mas é importante reconhecer os impactos quando os filtros são aplicados. O filtro numérico, por exemplo, afeta diretamente a capacidade de análise de conjuntos de respostas compostas por valores ou datas. 

Uma dificuldade, entretanto, é quantificar qual é o nível de filtragem desejável, balanceando a aquisição da informação. O ideal é que todos os processos de filtragem não causem impacto nos núcleos da resposta, que contêm os termos essenciais e fortemente vinculados ao tema. Nessa linha, em sua maioria, os casos de filtragem de algumas palavras específicas não impactam a forma e mantêm os termos com aderência ao tema.

Nessa etapa, os filtros de conteúdos são métodos de redução de ruído, responsáveis por discernir quais termos podem ser extraídos de cada item de resposta. Os ruídos podem causar a queda no desempenho do algoritmo da mesma forma ou pior do que a perda de informação causada na filtragem. Assim, o ruído em meio ao texto pode ser um grande problema para o sistema durante a interpretação do conhecimento. Com isso, esperamos que a filtragem auxilie os processos subsequentes com a capacidade interpretativa e relacional entre as respostas na formação do \textit{Modelo Avaliativo}.


\subsection{Transformação}
\label{subsec-transformacao}
% TRF_MTH = ["NER", "MORPH", "POS", "STEMM", "LEMMA"]

As análises de conteúdo são realizadas assim que os níveis anteriores prepararam o texto. Na transformação, a linguagem é analisada em níveis linguísticos. Neste processo, são interpretados alguns detalhes da construção textual para extração de \textit{features} via técnicas de NLP. Essas técnicas observam, entre as funções de cada palavra no texto, aspectos desde sua formação até sua função dentro da frase. Os diferentes níveis analisados nessa etapa são apresentados a seguir:

\begin{itemize}
	\item Modificação: Tipografia;
	\item Modificação: \textit{Stemming};
	\item Modificação: \textit{Lemmatization};
	\item Análise Gramatical: \textit{Part-of-Speech Tags} (POS-Tags);
	\item Análise Semântica:\textit{Named Entity Recognition} (NER);
	\item Análise Morfológica;
\end{itemize}

Cada uma das técnicas da lista aplica uma diferente transformação no texto. A primeira, bem simples, realiza a conversão do texto para uma tipografia comum, seja ela com letras maiúsculas ou minúsculas. Por outro lado, \textit{stemming} realiza conversão mais complexa, recuperando a raiz da palavra na construção da linguagem. Com \textit{stemming}, as palavras são convertidas para um núcleo comum, removendo as flexões, os prefixos e os sufixos. Por outro lado, um método equivalente é realizado com \textit{lemmatization}. Nesse outro, as palavras são convertidas para o \textit{lemma}, a palavra base, na forma com a qual é encontrada nos dicionários \cite{baeza2011}. 

Os métodos analíticos compõem ainda outras três formas mais robustas de identificação linguística. A técnica de \textit{POS-Tags} aplica a extração da função gramatical de cada palavra segundo seu emprego na frase. Em âmbito gramatical, identifica-se qual é o papel de cada palavra no contexto, entre verbos, adjetivos, pronomes, totalizando 17 categorias \cite{marneffe2021}.

Em nível semântico, aplica-se \textit{NER}, classificando o tipo de entidade nomeada de cada um dos \textit{tokens}. Com o NER, os nomes encontrados no texto são caracterizados pela classe que eles representam \cite{pirovani2019}. Entre as categorias reconhecidas há \textit{pessoa} (PER), \textit{local} (LOC), \textit{organização} (ORG) e \textit{diversos} (MISC). Isso permite que o sistema reconheça de forma simétrica diferentes menções dentro do conjunto de respostas para as principais categorias de entidades.

Por fim, o analisador morfológico identifica características da construção de cada palavra. Pela análise morfológica, as palavras são representadas pela sua flexão. Entre as flexões classificadas por cada termo estão as nominais (como gênero, número e definição) e verbais (pessoa, modo, tempo, voz). Além disso, esse mesmo processo também é responsável por algumas classificações léxicas de pronomes, adjetivos e advérbios \cite{marneffe2021}.

Cada uma dessas transformações é realizada para ampliar o conhecimento de cada \textit{token}. As análises mais complexas da linguagem e as categorizações dos termos permitem que as respostas sejam interpretadas de forma mais profunda. Essa profundidade é necessária para que, além do nível textual, a simetria das respostas seja levada em consideração. Desse modo, nesse ponto, a linguagem torna-se mais próxima da compreensão do algoritmo do que da forma original, aplicada na escrita.

A resultante desses processos é uma forte análise das componentes textuais de cada documento, buscando a identificação e a compreensão das estruturas textuais \cite{spalenza2020}. Com maior profundidade textual, espera-se tornar o sistema mais flexível para lidar com texto livre \cite{ding2020}. Assim, apesar das nuances da linguagem, o sistema consegue reconhecer e lidar com padrões que estão na composição de cada sentença \cite{filighera2020}. Então, a compreensão de diferentes níveis linguísticos é fundamental para a construção do SAG \cite{sahu2020}.


\subsection{Vetorização}
\label{subsec-vetorizacao}

A vetorização, como última etapa do pré-processamento, é responsável por extrair o modelo numérico de cada documento, permitindo mensurar a diferença ou a equivalência para os demais da coleção. Dessa forma, os documentos são representados por vetores numéricos segundo seu padrão de características. Cada uma das \textit{features} é analisada conforme sua frequência de ocorrência em cada documento do \textit{dataset}. A representação vetorial numérica de cada documento pela frequência é denominada \textit{Term Frequency} (TF). Sendo a coleção de documentos $ D = \{ d_{0}, d_{1}, d_{2}, \hdots d_{i} \} $ e as \textit{features} encontradas nos documentos $ F = \{ f_{0}, f_{1}, f_{2}, \hdots f_{j} \} $.  Portanto, para cada documento $ d $ na coleção $ D $, conta-se a frequência de cada \textit{feature} $ f_{j} $ do vocabulário $ F $. Assim, a forma vetorial do documento de índice $ i $ é dada por $ d_{i} $, sendo o vetor composto pela frequência $ n $ de cada \textit{feature} no documento $ n_{i, j} $. Então, cada documento é representado em $ D $ por sua forma vetorial $ d_{i} = [\ n_{i, 0}, n_{i, 1}, n_{i, 2} \hdots n_{i,j} ]\ $, usando TF.

Dadas as diferenças entre a frequência de cada termo em cada documento, é aplicada a ponderação para equilibrar a relação de frequência. A ponderação é denominada \textit{Inverse Document Frequency} (IDF). O \textit{Term Frequency-Inverse Document Frequency} (TF-IDF) estabelece a relação de que termos que ocorrem em muitos documentos têm menor relevância \cite{baeza2011}. A ponderação ocorre conforme a Equação \ref{eq-tfidf}.

\begin{equation}
TF-IDF = d_{i,j}* \log \frac{n_{D}}{n_{d_{j}}}
\label{eq-tfidf}
\end{equation}

O IDF é uma ponderação na frequência de cada \textit{feature} no vetor $ d_{i, j} $, segundo o total de documentos $ n_{d_{j}} $ que contém $ f_{j} $ em relação ao total de documentos da coleção $ D $. Essa ponderação reduz a diferença numérica entre uma \textit{feature} encontrada em todos os documentos para as \textit{features} que estão apenas em grupos específicos de documentos. Assim, o uso dessa forma potencialmente delimita melhores características (\textit{features}) para a construção de modelos avaliativos. A aplicação desse modelo está associada à capacidade de identificação de características com alta correlação a grupos específicos de nota.

No método de vetorização, durante a contagem de frequência de cada \textit{feature}, é priorizada a análise de vizinhança entre os termos. Preservando o aspecto textual, os \textit{n-grams} estabelecem a frequência conjunta dos termos dentro de sequências. Em vez de cada documento ser representado por um vetor simples da frequência de cada \textit{feature}, essa frequência é calculada segundo a sequências dos $ n $ termos. Sendo assim, aplicamos valores de $ n $ entre 1 a 5-\textit{grams}. São utilizadas simultaneamente as sequências de um a cinco termos, de forma a capturar o comportamento das estruturas textuais na formação dos documentos. Dessa forma, os padrões identificados em \textit{n-grams} \cite{spalenza2020} visam à associação entre \textit{features} fortemente correlacionadas nos vetores, compondo o modelo avaliativo.


\section{Particionamento do Conjunto de Respostas}
\label{sec-amostragem}

No formato vetorial, temos uma representação numérica dos documentos. Por meio dessa representação, podem ser comparadas as estruturas textuais que cada item de resposta contém. A comparabilidade permite identificar o comportamento das respostas definindo padrões. O conjunto desses padrões forma a distribuição das respostas no espaço vetorial. É possível assim, mensurar a proximidade entre as respostas, enquanto amostras, para formação de \textit{clusters} para análise. Aplicando \textit{Unsupervised Learning}, é realizado o \textit{Particionamento do Conjunto de Respostas} para identificar estruturas textuais similares e contextualizar as diferentes formas de resposta. Essa forma de extração de conhecimento por clusterização é apresentada na Figura \ref{fig-pcr}.

\begin{figure}[!h]
\centering
\includegraphics[width=\textwidth]{figuras/esquema-pcr-pNota.png}
\caption{Módulo de \textit{Particionamento do Conjunto de Respostas} no esquema do \textit{p}Nota.}
\label{fig-pcr}
\end{figure}

Na Figura \ref{fig-pcr} é apresentado um primeiro passo do método de \textit{Active Learning} empregado neste trabalho. São identificados diferentes componentes de resposta pela distribuição espacial dos vetores para anotação de amostras selecionadas do conjunto pelo especialista. A partir dessa forma de amostragem, são coletadas as notas, de acordo com o conteúdo que forma cada resposta ou grupo de respostas. Consequentemente, o processo de anotação requisitado pelo sistema vincula o conteúdo destas respostas com os padrões de avaliação do especialista. Diferentemente da maioria dos sistemas, que realizam amostragem aleatória, o \textit{p}Nota analisa as instâncias que compõem cada \textit{cluster} formado.


\subsection{Clusterização}
\label{subsec-clusterizacao}

É realizado o particionamento das respostas utilizando técnicas de clusterização. Esse processo é responsável por agrupar respostas em grupos por similaridade. O algoritmo de \textit{clusterização} utilizado é o \textit{Agglomerative} \cite{spalenza2019}, um método hierárquico de agrupamento por proximidade. O \textit{Agglomerative} compreende formar \textit{clusters} agrupando item a item até que um limiar de proximidade seja alcançado dado um $ k $ número de \textit{clusters} \cite{everitt2011}. Os grupos formados, ou \textit{clusters}, indicam algum nível de compatibilidade entre as estruturas que formam as respostas. Assim, a análise entre a equivalência e a divergência das respostas permite a contextualização da avaliação do especialista.

Para isso, precisamos que os \textit{clusters} sejam bons descritores do contexto, captando bem esse aspecto de equivalências e divergências textuais. A forma de designar se há o equilíbrio entre os \textit{clusters} foi realizada por meio do \textit{elbow method} \cite{everitt2011}. Esse método compreende testar uma sequência de parâmetros da clusterização para identificar a melhor combinação de \textit{clusters} formados segundo uma métrica de qualidade. Em geral, a métrica de qualidade é diretamente relacionada ao propósito de uso dos \textit{clusters}.

Entre as métricas estudadas estão \textit{Calinski-Harabasz Score} (CHS) \cite{calinskiharabasz1974}, \textit{Davies-Bouldin Score} (DBS) \cite{daviesbouldin1979}, \textit{Silhouette Score} (SS) \cite{rousseeuw1987}, \textit{Sum of Squared Errors} (SSE) \cite{maimon2005} e o \textit{Coeficiente de Variação} do tamanho do cluster (CVS). Essas métricas são denominadas \textit{Índices de Validação Interna} e avaliam os agrupamentos sem considerar a classe de cada amostra.

Cada índice é uma heurística utilizada para mensurar, sob diferentes perspectivas, a qualidade dos \textit{clusters} gerados em relação a outras formas de agrupamento de um mesmo \textit{dataset}. CHS mensura a razão entre a dispersão dos itens intra-\textit{cluster} e a dispersão extra-\textit{cluster}. DBS é o índice que estabelece a relação entre a média de similaridade entre as amostras do \textit{cluster} para a média de similaridade entre-\textit{clusters}. SS é a média entre as distâncias das amostras pertencentes a um \textit{cluster} em relação às amostras do \textit{cluster} mais próximo. SSE é uma métrica que avalia o erro de cada amostra que compõe um \textit{cluster} em relação ao seu centroide. O centroide é o ponto médio dos itens que constituem cada \textit{cluster}. Portanto, o centroide é uma instância representante da dispersão dos itens no \textit{cluster}, porém é um ponto artificial e não necessariamente uma amostra que o compõe. Por fim, CVS avalia o equilíbrio entre o número de amostras agrupadas em cada \textit{cluster}, considerando a diferença entre o maior grupo e o menor grupo formados.

Para a avaliação de respostas abertas, consideramos que o ideal são as análises que balanceiam os itens de cada \textit{cluster} em relação aos \textit{cluters} adjacentes. Por isso, \textit{clusters} com padrões muito específicos não devem formar bons descritores para a distribuição, mas sim para uma ou poucas amostras que compõe o grupo. Assim, para a proposta de \textit{Active Learning}, os grupos equilibrados têm maior potencial para aquisição de informação. Por outro lado, também é fundamental reconhecer a proximidade intra e inter-\textit{cluster}. Assim, foram combinados CVS e SS para avaliação amostral. Aplicamos CVS na formação dos agrupamentos enquanto o SS é observado durante a seleção amostral.


O processo de otimização, que incide sobre o \textit{elbow method}, testa os parâmetros de clusterização e visa reduzir o intervalo de busca enquanto maximiza os resultados do índice. Para isso, foram avaliados três métodos \textit{Forest Optimization}, \textit{Gaussian Process Optimization} e \textit{Dummy}. Os resultados obtidos com os dois primeiros foram equivalentes nesse contexto, sendo escolhido o \textit{Gaussian Process} para a aplicação \cite{spalenza2019}. Esse método analisa cada teste pela distribuição dos valores da métrica de qualidade como uma \textit{gaussiana}, buscando pontos de máxima da função. A resultante é dada pelo melhor valor encontrado. O parâmetro sob controle é o $ k $, número de \textit{clusters}. O intervalo de $ k $ é definido por valores de $ 2 $ até $ 2 * \sqrt{n} $, sendo $ n $ o número de amostras do \textit{dataset} \cite{han2011}. Simultaneamente, para cada combinação de $ k $ são testadas 20 métricas de distância.

\begin{itemize}
\begin{multicols}{4}
  \item braycurtis
  \item canberra
  \item chebyshev
  \item correlation
  \item cosine
  \item dice
  \item euclidean
  \item hamming
  \item haversine
  \item jaccard
  \item kulsinski
  \item mahalanobis
  \item manhattan
  \item matching
  \item minkowski
  \item rogerstanimoto
  \item russellrao
  \item sokalmichener
  \item sokalsneath
  \item yule
  \end{multicols}
\end{itemize}

O agrupamento selecionado é utilizado para amostragem em um percentual do conjunto de respostas disponíveis. Ainda avaliamos de forma qualitativa essa seleção segundo três índices \textit{Homogeneity} (HS), \textit{Completness} (CS) e \textit{Clustering Accuracy} (CA). Em uma ótica diferente da formação dos \textit{clusters}, com os índices qualitativos é mensurado o impacto de cada resultado da clusterização pela distribuição das classes. Esses são chamados \textit{Índices de Validação Externa}.

CA é o índice que avalia o desempenho da clusterização enquanto classificador por voto majoritário. Nesse cenário, cada \textit{cluster} é representado pela sua principal classe, mostrando a coesão dos grupos para sua representação de classe. Esse índice também estabelece um \textit{baseline} de desempenho de classificação. Essa métrica é simétrica a ACC, descrita na Equação \ref{eq-classification} da Seção \ref{subsec-classificacao}. HS é o índice que mensura se os \textit{clusters} são formados apenas por uma classe \cite{rosenberg2007}. CS por outro lado, avalia se todos os itens de uma classe estão presentes em um mesmo \textit{cluster} \cite{rosenberg2007}. Ambos são métricas que avaliam a entropia ($H$) dos \textit{clusters} dada a anotação das amostras, apresentadas na Equação \ref{eq-hs-cs}.

\begin{equation}
Homogeneity = 1 - \frac{H(y_{c} | \hat{y}_{c})}{H(y_{c})}
\label{eq-hs-cs}
\end{equation}

\begin{equation*}
Completness = 1 - \frac{H(y_{c} | \hat{y}_{c})}{H(\hat{y}_{c})}
\end{equation*}

A Equação \ref{eq-hs-cs} apresenta as métricas HS e CS, referências para a concentração das classes reais ($y_{c}$) dada a distribuição dos \textit{clusters} ($\hat{y}_{c}$). Assim, identifica-se o comportamento das classes de nota na distribuição pela entropia. Tais métricas permitem a identificação \textit{a posteriori} dos resultados mais coesos de clusterização. A concentração de classe por \textit{cluster}, permite uma melhor amostragem sendo possível amplificar os resultados obtidos na etapa de classificação.


\subsection{Seleção de Amostras}
\label{subsec-selecao-amostras}

Com a formação dos \textit{clusters}, identificam-se as principais respostas de cada agrupamento para inferência do modelo avaliativo do professor. A amostragem é realizada com a coleta de um percentual dos itens mais representativos que compõem o \textit{dataset}. Essa coleta analisa padrões de documentos de cada \textit{cluster}, a fim de compreender como é dada a avaliação do especialista para os diferentes padrões de resposta. As amostras são selecionadas conforme critérios específicos, descrevendo componentes do \textit{cluster}. 

A nossa amostragem segue alguns critérios. Os critérios definem a sequência de seleção para atingir o percentual escolhido para anotação. No primeiro grupo de amostras são selecionados os pares de amostras que apresentam maior e menor similaridade de cada \textit{cluster}. O segundo grupo é composto por amostras com mais e menos características. O terceiro grupo é formado de duas formas: pelo coeficiente \textit{silhouette} \cite{rousseeuw1987} de cada amostra ou pela seleção aleatória.

Nesse último grupo a análise de dispersão calcula o coeficiente de \textit{silhouette} da amostra. Tal qual o SS, esse índice determina a razão entre a distância da amostra para os demais itens do grupo em relação aos itens do \textit{cluster} mais próximo. Dessa forma, por meio desse método é incrementado o número de amostras por dispersão até alcançar o percentual de amostragem selecionado. Uma outra opção de seleção é a escolha de amostras pelo balanceamento do tamanho dos \textit{clusters}. Nesse caso, o método determina que um item seja aleatoriamente selecionado, ponderado de acordo com a quantidade de itens que compõem cada grupo. Descartando as duas opções anteriores, as amostras são selecionadas aleatoriamente entre todo o \textit{dataset}.

Terminando esse procedimento de seleção, as amostras são enviadas para atribuição de notas pelo professor. Na plataforma de correção que o professor adotou, ele realiza a atribuição de notas para cada item sugerido pelo sistema após a amostragem. Finalizado esse processo, com o conjunto de respostas representativas e suas respectivas notas atribuídas pelo professor, começa-se a análise de padrões para inferência das notas para as demais respostas.

\section{Modelo Avaliativo}
\label{sec-avaliacao}

O passo posterior à atribuição de notas, etapa com participação do professor, é a criação dos modelos computacionais. O \textit{Modelo Avaliativo}, é a etapa do \textit{p}Nota que desenvolve o SAG para replicar a forma de avaliação do professor. Assim, após a atribuição de notas, são criados os padrões que vinculam o texto e a avaliação.  O sistema, portanto, deve se aproximar da forma como o professor gera avaliações. A plataforma utilizada para reconhecimento de padrões de nota é apresentada na Figura \ref{fig-ma}.

\begin{figure}[!h]
\centering
\includegraphics[width=\textwidth]{figuras/esquema-ma-pNota.png}
\caption{Etapa de construção do \textit{Modelo Avaliativo} no esquema do \textit{p}Nota.}
\label{fig-ma}
\end{figure}

Ao receber todas as notas para as amostras, o processo começa a construção dos modelos, tal qual ilustrado na Figura \ref{fig-ma}. A criação de um modelo SAG complexo compreende identificar detalhes correspondentes entre as respostas. O \textit{p}Nota analisa a correspondência entre notas e \textit{features} textuais para analisar equivalência. É possível considerar que padrões equivalentes de uma mesma nota têm alta probabilidade de serem relacionados ao que o professor considerou para avaliação. Cada atividade tem especificamente uma forma de avaliação e um padrão avaliativo alinhado com a prática do professor. Portanto, a identificação do critério avaliativo não é trivial. Por isso, a técnica de \textit{Active Learning} proposta neste estudo é fundamental para compreender o modelo e a forma que o professor trabalha durante a avaliação.

O \textit{Modelo Avaliativo} gerado pelo \textit{p}Nota é responsável por atender a expectativa de nota do professor. Sua função é vincular o padrão avaliativo com o padrão textual (\textit{features}) do conjunto de resposta. Então, a função da técnica de \textit{Active Learning} proposta é treinar classificadores contextuais, transformando os métodos tradicionais de ML em avaliadores especializados. Por meio do conhecimento de uma série de níveis de linguagem e da otimização da seleção de amostras busca-se reduzir os problemas que são incorretamente atribuídos apenas à técnica aplicada na atribuição de notas. Desta forma, espera-se que os \textit{Modelos Avaliativos} lidem com a variação linguística, a individualidade dos \textit{outliers} e o desbalanceamento dos níveis de nota.


\subsection{Classificação}
\label{subsec-classificacao}

O processo de classificação é utilizado em dois tipos distintos de avaliação: com notas ordinais e discretas. As notas ordinais permitem estabelecer ordem de escala numérica enquanto as discretas são textuais e requerem interpretação. No entanto, aos classificadores empregados, trabalha-se com a relação de diferença entre os níveis para aprendizado de convergência e divergência. Por um lado, a convergência indica equivalência entre os padrões de avaliação e texto encontrados nas amostras. Por outro lado, a divergência indica os padrões incompatíveis de texto por nota e entre níveis de nota, degradando sua influência avaliativa. As técnicas empregadas, em sua essência, devem assimilar o que compõe um nível de nota (equivalência) e o que é informação irrelevante ou auxiliar (divergência). Para estudar esses aspectos são aplicadas cinco diferentes formas de reconhecimento de padrões por meio dos algoritmos: \textit{K-Nearest Neighbors}, \textit{Decision Tree}, \textit{Support Vector Machine}, \textit{Gradient Boosting}, \textit{Random Forest} e \textit{WiSARD}.

O \textit{K-Nearest Neighbors} (KNN) é o algoritmo de classificação pela análise da vizinhança amostral. No KNN, cada amostra é categorizada pela distribuição local dos seus $ k $ vizinhos. A atribuição do rótulo é por voto majoritário, atribuindo o mesmo valor à amostra não anotada. Diferentemente deste, o algoritmo \textit{Decision Tree} (DTR) estabelece a equivalência entre amostras, sob uma perspectiva das características que as compõem. O DTR associa os grupos anotados com a mesma classe pelos limiar das características, gerando regras de decisão. As regras, elaboradas automaticamente, delimitam as principais \textit{features} segundo os valores de tendência de classe. O processo de classificação, então, acontece com a comparação de cada um dos itens dentre a cadeia de decisões na estrutura de árvore.

Outro tradicional algoritmo, o \textit{Support Vector Machine} (SVM) estabelece uma forma distinta de observar os dados. Os dados, em grupos por categoria, formam um \textit{kernel}. O \textit{kernel}, diferentemente do DTR, cria modelo espacial que delimita a diferença entre categorias. Então, cada amostra, é identificada segundo sua posição em relação ao limiar de características dado o modelo representante da classe. De forma similar é aplicado o algoritmo \textit{Wilkes, Stonham and Aleksander Recognition Device} (WSD) \cite{aleksander1984, wisard2020}, conhecido como \textit{WiSARD}\footnote{wisardpkg - https://github.com/IAZero/wisardpkg}. O algoritmo produz um modelo binário com o registro de padrões de características. Cada padrão é reconhecido em análise sequencial de um intervalo de bits predefinido. O modelo binário criado é comparado com as respostas não avaliadas, categorizando-as pela similaridade entre padrões. Especificamente para esse algoritmo, a conversão dos vetores TF-IDF em seu formato binário foi dada com 1 \textit{bit} por característica, de acordo com a esparsidade observada em dados textuais \cite{manning1999}. Assim, dado como pré-requisito de sua aplicação, o padrão submetido é dado pela existência (valor 1) ou não (valor 0) de cada característica na resposta.

Adicionalmente, dois modelos de \textit{ensemble} foram aplicados. Os \textit{ensembles} são técnicas que combinam vários classificadores mais simples para determinar áreas de decisão mais robustas. Os classificadores simples são denominados \textit{weak learners}, em busca de detalhes na avaliação entre termos e classes. Nesse aspecto, \textit{Random Forest} (RDF) é um algoritmo que combina o método tradicional \textit{Decision Tree} com \textit{subsets} de amostras. Desta forma, o RDF combina análises parciais do conjunto de dados para definir regras de decisão mais complexas sobre a distribuição de amostras. De modo similar, o \textit{Gradient Boosting} (GBC) combina uma série de \textit{Regression Trees} para otimização diferencial da função de perda (\textit{loss}). Nessa linha, o GBC observa o gradiente da função de perda com \textit{Logistic Regression}. Nesse aspecto, com uma série de amostragens, a técnica procura minorar o erro de classificação obtido com a calibração do modelo segundo uma sequência de \textit{subsets}. 

A combinação com modelos tradicionais e técnicas de \textit{ensemble} visa potencializar a capacidade analítica do método. Com diferentes formatos de dados, a proposta deste trabalho testa diferentes modelos procurando o que melhor se adapta ao padrão avaliativo do professor. Nesse aspecto, o método de classificação é escolhido de acordo com a similaridade entre o modelo automático com o critério do professor \cite{pado2021}. Para avaliar esse aspecto utiliza-se o coeficiente \textit{kappa} quadrático \cite{cohen1960}. As amostras são separadas em dois grupos acordo com o \textit{Stratified K-Fold}, para mensurar a capacidade de cada algoritmo na categorização das amostras. Dentro do próprio conjunto utilizado para treinamento dos modelos SAG, é possível avaliar a paridade dos resultados com o avaliador humano \cite{artstein2008}.

Na sequência, a qualidade de cada um é avaliada com quatro métricas. A \textit{Accuracy} (ACC), ou acurácia, mensura a equivalência percentual entre as avaliações. A \textit{Precision} (PRE), ou precisão, quantifica a atribuição correta de rótulos em razão da quantidade de atribuições incoerentes da mesma categoria. De forma similar, a \textit{Recall} (REC), ou revocação, quantifica a atribuição correta de rótulos em razão dos itens de determinada classe que foram classificados de forma incorreta. Por fim, F1 é o balanceamento entre PRE e REC, observando simultaneamente os erros de e para cada classe. A Equação \ref{eq-classification} apresenta a fórmula de cada uma das métricas citadas para avaliação qualitativa dos algoritmos de classificação testados.

\begin{equation}
Accuracy = \frac{TP+TN}{TP+TN+FP+FN}
\label{eq-classification}
\end{equation}

\begin{equation*}
Precision = \frac{TP}{TP+FP}
\end{equation*}

\begin{equation*}
Recall = \frac{TP}{TP+FN}
\end{equation*}

\begin{equation*}
F{1} = \frac{2*Precision*Recall}{Precision+Recall}
\end{equation*}

Na Equação \ref{eq-classification}, é possível observar as fórmulas para mensurar a qualidade dos classificadores. Nelas $ T $ refere-se aos casos verdadeiros e $ F $ aos falsos. Da mesma forma, $ P $ refere-se aos casos positivos e $ N $ aos negativos \cite{manning2008}. Porém, tradicionalmente a avaliação tem nuances que se extendem para além das marcações entre certo (ou verdadeiro) e errado (ou falso). Assim, as métricas são balanceadas conforme o número de classes, como determinado na Equação \ref{eq-classification-nlabels}. 

\begin{equation}
Accuracy = \frac{1}{n_\text{amostras}} \sum_{i=0}^{n_\text{amostras}-1} 1(\hat{y}_i = y_i)
\label{eq-classification-nlabels}
\end{equation}

\begin{equation*}
Precision_{macro} = \frac{1}{\left|n_{classes}\right|} \sum_{c \in n_{classes}} Precision(y_c, \hat{y}_c)
\end{equation*}

\begin{equation*}
Recall_{macro} = \frac{1}{\left|n_{classes}\right|} \sum_{c \in n_{classes}} Recall(y_c, \hat{y}_c)
\end{equation*}

\begin{equation*}
F{1}_{macro} = \frac{1}{\left|n_{classes}\right|} \sum_{c \in n_{classes}} F{1}_\beta(y_c, \hat{y}_c)
\end{equation*}

\begin{equation*}
F{1}_{ponderado} = \frac{1}{\sum_{c \in n_{classes}} \left|y_c\right|} \sum_{c \in n_{classes}} \left|y_c\right| F{1}(y_c, \hat{y}_c)
\end{equation*}

A Equação \ref{eq-classification-nlabels} mostra as métricas qualitativas aplicadas em avaliações com múltiplos níveis de nota, sendo $y$ o valor de nota atribuído para cada amostra ou grupo de amostras ($c$) \cite{manning2008}. Por definição, os SAGs são majoritariamente criados para avaliações com mais de uma classe de nota. É usada para mensurar o desempenho a média (macro) da atribuição de notas. Porém, pelo já esperado desbalanceamento entre notas, também avalia-se o F1 ponderado pela quantidade de amostras por classe. Comparando estatisticamente os desempenhos, via \textit{kappa}, a expectativa é selecionar o que tem notas mais adequadas ao modelo avaliativo do professor. Mas, o melhor modelo é o que efetivamente apresenta maior ganho de qualidade nessas métricas quando comparado com a avaliação final do professor.


\subsection{Regressão}
\label{subsec-regressao}

Outra forma de atribuição de nota é a não-categórica. Nesses casos, são chamadas de notas contínuas, pois apresentam um intervalo de notas possíveis mas sem níveis específicos. São aplicados os métodos de regressão, estimando a nota pela similaridade entre respostas. Nesse formato ainda se enquadram as noções de \textit{equivalência} e \textit{divergência} entre as \textit{features} das respostas na avaliação. Os cinco métodos de regressão aplicados são \textit{Regressão Linear}, \textit{Lasso}, \textit{K-Nearest Neighbors}, \textit{Decision Tree} e \textit{WiSARD}.

A Regressão Linear (LNRG) é um algoritmo que avalia a tendência linear das amostras segundo sua distribuição. Essa tendência linear busca, no espaço n-dimensional das características, definir os coeficientes do hiperplano que minimizam o resíduo entre as amostras. É importante para o algoritmo determinar uma função de tendência dos dados. Minimizar o erro pelos coeficientes da função reflete na simplificação do conjunto de dados. Contudo é determinante que o modelo não apresente \textit{overfitting} e um baixo desempenho com o viés dos dados de treinamento. Por outro lado, como espera-se do algoritmo, a aquisição de informação deve extrair um modelo que minimamente descreva os dados conhecidos, evitando a ocorrência de \textit{underfitting}. Assim, o modelo simplificado deve ser direcionado ao desempenho linear e não apenas à associação forte com o conjunto de treinamento. Também é utilizada uma variante do LNRG tradicional, denominada \textit{Least Absolute Shrinkage and Selection Operator - Lasso} (LSSR), que utiliza a normalização dos dados com a função $ L1 $, reduzindo a complexidade do modelo de dados e prevenindo o \textit{overfitting}.

Os demais três modelos, são similares aos modelos utilizados na classificação. O \textit{K-Nearest Neighbors} (KNRG), assim como o algoritmo de classificação, observa a distribuição dos dados e define o valor resultante de acordo com a vizinhança. Assim, o resultado de cada amostra de valor desconhecido é a interpolação entre os valores das $ K $ amostras mais próximas conhecidas. De forma semelhante, \textit{Decision Tree} (DTRG) observa características semelhantes entre amostras e, por equivalência, divide em subgrupos. A subdivisão dos itens na árvore e o particionamento em subgrupos delimita regiões específicas com resultantes correspondentes por aproximação. Dessa maneira, após o particionamento das regiões amostrais em zonas de decisão, o valor dado para todas as amostras ali categorizadas é a média conhecida do subgrupo de treinamento. De forma similar funciona a WiSARD (WSRG), organizando registradores com as notas das respostas similares atribuindo o valor médio do registrador para respostas de padrão equivalente.

Para seleção do regressor mais adequado utiliza-se a correlação de \textit{Pearson}. Pela correlação mensura-se a compatibilidade dos avaliadores como pares, do sistema e do professor. Nessa visão, maiores índices de correlação indicam distribuições equivalentes de distribuição de notas \cite{morettin2010}. Isso implica modelos avaliativos mais equivalentes ao método avaliativo do professor. Já a avaliação é dada pelo resíduo entre as duas notas, considerando como ideal o modelo que apresenta uma série de notas próximas do que foi atribuído pelo professor. Assim, para mensurar a diferença entre a expectativa do professor e a nota resultante do sistema são utilizadas três métricas: o \textit{Mean Absolute Error} (MAE), o \textit{Mean Squared Error} (MSE) e o \textit{Root Mean Squared Error} (RMSE)

O MAE, erro médio absoluto, mensura a resíduo absoluto entre a nota predita e a nota dada pelo professor. Em outras palavras, o MAE avalia as diferenças em módulo entre os valores obtidos, segundo o alinhamento de cada predição com a expectativa do professor. Enquanto isso, MSE ou erro médio quadrático, é uma medida do resíduo entre os valores com penalização dos erros absolutos. Assim, por meio do MSE erros maiores têm maior impacto no sistema quando comparados com erros de menor grau. Por fim, o RMSE ou raiz do erro médio quadrático, é a raiz quadrada do valor obtido no MSE, normalizando o erro obtido nessa métrica em relação à avaliação do professor. A Equação \ref{eq-regressao} apresenta a fórmula de cada uma das métricas utilizadas para avaliação dos métodos de regressão citados.

\begin{equation}
MAE = \sum_{i=0}^{D}|y_i-p|
\label{eq-regressao}
\end{equation}

\begin{equation*}
MSE = \sum_{i=0}^{D}(y_i-p)^2
\end{equation*}

\begin{equation*}
RMSE = \sqrt{\sum_{i=0}^{D}(y_i-p)^2}
\end{equation*}

Na Equação \ref{eq-regressao} são apresentadas as fórmulas de avaliar o erro do modelo criado conforme a expectativa de nota. Assim, em cada fórmula as amostra $ i $ da coleção são comparadas com as notas atribuídas do avaliador humano (professor) $ y $ e pelo sistema $ p $. O melhor modelo avaliativo é o que apresenta menor nível de erro em relação à atribuição de notas do professor. Apesar de serem comuns os erros entre modelos computacionais e a expectativa do especialista, é crucial para um bom avaliador automático a proximidade entre os modelos. Nos sistemas SAG, foram observados durante a correção entre professores até 0,66 pontos de divergência em notas de zero a cinco pontos \cite{mohler2011}. Em uma escala de zero a dez pontos, representaria 1,32 pontos de divergência entre avaliadores humanos. Assim, é esperado que o sistema minimize os erros em relação ao professor, reduzindo a divergência para a nota do professor.


\section{Relatórios e \textit{Feedbacks}}
\label{sec-relatorios}

Com as notas atribuídas pelo sistema, ocorre a criação de relatórios e \textit{feedbacks}. Eles contribuem para descrição do modelo de Inteligência Artificial aplicado na atribuição de notas. Em especial, os \textit{feedbacks} devem destacar quais são as \textit{features} relevantes levadas em conta na criação do modelo avaliativo. Nessa mesma linha, os relatórios são a forma de descrever os processos realizados para todos os participantes. Esta etapa, aplicada na descrição dos processos internos do \textit{p}Nota, é ilustrada na Figura \ref{fig-rf}. 

\begin{figure}[!h]
\centering
\includegraphics[width=\textwidth]{figuras/esquema-rf-pNota.png}
\caption{Módulo de \textit{Relatórios e Feedbacks} no esquema do \textit{p}Nota.}
\label{fig-rf}
\end{figure}


Conforme a etapa apresentada na Figura \ref{fig-rf}, os relatórios precedem o envio dos resultados para o AVA, de forma a descrever o \textit{Modelo Avaliativo}. Com fundamento na relação entre termos e notas, a caracterização das respostas é determinante para conectar usuários com os métodos aplicados pelo \textit{p}Nota. Assim, os relatórios e os \textit{feedbacks} devem descrever em detalhes a forma de avaliação aplicada para instruir os usuários. Um exemplo completo dos relatórios é apresentado no Apêndice \ref{exemplo-pNota}.

Os procedimentos de descrição do \textit{p}Nota incluem, desde características superficiais, conectadas às calibrações e às etapas do sistema, até características mais profundas, que contextualizam a interpretação do sistema sobre a atividade. É possível citar como exemplos do primeiro os \textit{clusters} formados, amostras selecionadas para anotação ou características mais frequentes. No segundo, por outro lado, são apresentados em linguagem natural os resultados e a proximidade entre as notas.


\subsection{Relatório dos Processos}
\label{subsec-relatorio-processos}

Os relatórios buscam passar por cada etapa do \textit{p}Nota para mostrar como foram as interações com o professor e os resultados obtidos. Um desses é o relatório de esforço de anotação e treinamento do algoritmo. Retomando o exemplo do Capítulo \ref{cap1-intro}, na Tabela \ref{tab-ptasag-train-46} é apresentado tal relatório para a atividade 46 do \textit{dataset PTASAG}.

\begin{table}[!b]
\centering
\caption{Particionamento das amostras em treino e teste na atividade exemplo \textit{PTASAG Atividade 46}.}
\label{tab-ptasag-train-46}
\begin{tabular}{|c c c c|} \hline
\multicolumn{3}{|l}{Dataset} & Amostras\\ 

\multicolumn{3}{|l}{PTASAG : Atividade 46} & 655 \\ \hline 

Treino (Un.) & Treino (\%)  & Teste (Un.) & Teste (\%) \\ \hline 

524 & 80.0 & 131 & 20.0 \\ 

\hline \hline
\end{tabular}
\end{table}


Na Tabela \ref{tab-ptasag-train-46} é apresentado um exemplo de relatório utilizado para explicar o que foi realizado em um dos processos. O mesmo informa qual foi o particionamento de amostras utilizado e qual foi o esforço de correção do professor. Porém, o nível descritivo deve ser maior quando caracterizamos aspectos avaliativos, tornando cada vez mais transparente a avaliação. Durante a elaboração destes, identifica-se uma certa dificuldade de interpretação das métricas categóricas em relação ao observado com métricas contínuas. Por conta disso, são estabelecidos três níveis de desempenho para as métricas percentuais: \textit{Avançado}, \textit{Adequado} e \textit{Insuficiente} \cite{nascimento2020}.

\begin{itemize}
	\item Intervalo de 75\% - 100\%: Nível \textcolor{green}{Avançado};
	\item Intervalo de 35\% - 75\%: Nível \textcolor{yellow}{Adequado};
	\item Intervalo de 0\% - 35\%: Nível \textcolor{red}{Insuficiente}.
\end{itemize}


Os níveis são similares aos que o professor utiliza para determinar o conteúdo assimilado pelos alunos durante a avaliação \cite{nascimento2020}. Em nível \textcolor{red}{Insuficiente} a relação entre as notas finais divulgadas pelo professor e as notas do sistema apresenta índices abaixo do esperado. Em nível \textcolor{yellow}{Adequado} as notas apresentam alinhamento com as que foram atribuídas pelo professor. E em nível \textcolor{green}{Avançado}, os resultados do modelo avaliativo foram próximos aos divulgados pelo especialista, identificando bem seu método avaliativo. Assim, foi necessário trazer para a realidade em sala de aula os resultados da classificação, tal qual já é a realidade quando é aplicado o nível de erro entre avaliadores \cite{almeida-junior2017}. Na Figura \ref{fig-ptasag-performance-46} é apresentado o desempenho de categorização conforme esses três níveis.

\begin{figure}[!t]
 \centering
 \includegraphics[width=0.8\textwidth]{figuras/exemplo/exemplo-ptasag-rdf.png}
 \caption{Resultados de desempenho do exemplo \textit{PTASAG Atividade 46}.}
 \label{fig-ptasag-performance-46}
\end{figure}

Na Figura \ref{fig-ptasag-performance-46}, há um resultado de alto desempenho de classificação usando o classificador RDF. Com os níveis e as cores, os resultados e as métricas de avaliação tornam-se um pouco mais simples para interpretação e análise dos professores, sendo assim, uma ferramenta útil para comparação e uso na sua rotina de validação do sistema.


\subsection{\textit{Feedbacks} Contextuais}
\label{subsec-feedbacks-contextuais}

Os \textit{feedbacks} contextuais são métodos que descrevem como foi o comportamento da avaliação segundo o conteúdo. Esses métodos aplicam-se diretamente à realidade da disciplina e visam caracterizar o vínculo textual de cada nível de nota. O objetivo é levar para as salas de aula um material que apoia os estudantes na compreensão da disciplina, dando suporte ao método do professor.

O primeiro modelo realiza a aplicação de cores nas respostas, identificando quais são as palavras mais correlacionadas com cada nota. Essa técnica é realizada com otimização por Algoritmo Genético \cite{spalenza2016a} ou com Lime\footnote{Lime - https://github.com/marcotcr/lime}. O Lime é uma ferramenta de visualização que descreve o processo de classificação de acordo com os padrões do conteúdo \cite{ribeiro2016}. Em ambos, a ideia é atribuir coloração por nota e mostrar os termos mais correlacionados com cada uma das classes. Assim, associa-se a menção de cada termo com a nota recebida pela resposta e seu alinhamento, definindo \textit{status} negativo ou positivo. Na Figura \ref{fig-highlight-46}, identificam-se termos que ampliam a relação da resposta com a nota 3 atribuída.

\begin{figure}[!h]
 \centering
 \includegraphics[width=.75\textwidth]{figuras/exemplo/highlight.jpeg}
 \caption{Destaques nos principais termos da resposta do estudante \#1995 do \textit{PTASAG Atividade 46}.}
 \label{fig-highlight-46}
\end{figure}

Como é exposto na Figura \ref{fig-highlight-46}, os termos \textit{coração}, \textit{sangue} e \textit{pressão} são destaques dessa resposta. Porém, os termos poderiam ser encontrados nas demais respostas. Por isso, foi identificado que o contexto apresenta 63\% de correlação com demais menções que receberam nota 3. Um segundo modelo, também em níveis textuais, extrai o conteúdo chave, de acordo com os tópicos mencionados por nível de nota. Nesse nível aplica-se LDA \cite{hoffman2013} no reconhecimento da composição de nota, ou seja os termos que em tese são essenciais para receber cada uma das classes. O LDA é uma técnica que aplica estatística descritiva para equivalência parcial dos dados. Nesse caso, a identificação da compatibilidade vetorial entre os textos que compõem um determinado nível de nota \cite{sahu2020}. Portanto, os dois são complementares. Enquanto o primeiro observa cada resposta pela perspectiva da avaliação, o segundo realiza o oposto. 

Apesar do acompanhamento da dinâmica do sistema no processo avaliativo, é complexo ao sistema identificar padrões coerentes de resposta. Para isso, utiliza-se o quadro de \textit{rubrics} para representar o modelo avaliativo elaborado pelo sistema em conjunto com o professor. O quadro de \textit{rubrics} é um modelo de caracterização do processo avaliativo conforme o modelo de resposta esperado para cada nota. Após o processo avaliativo, este torna-se um descritor, determinando na perspectiva dos estudantes quais foram as principais características elencadas para cada nota. Na Tabela \ref{tab-rubrics-exemplo} há um exemplo do quadro de \textit{rubrics} para a nota 3 da \textit{Atividade 46}.

\begin{table}[!h]
\centering
\caption{Tabela de \textit{rubrics} para as duas notas encontradas na atividade exemplo e as respostas mais alinhadas com as palavras selecionadas pelo LDA.}
\label{tab-rubrics-exemplo}
\footnotesize
\begin{tabular}{ p{2cm} | p{14cm}}
\multicolumn{2}{l}{\textbf{PTASAG : Atividade 46}} \\ \hline
\multicolumn{2}{c}{\textbf{Nota: 3}} \\ \hline 
\multicolumn{2}{l}{\textit{T{\'o}picos: arterias coracao corpo levam pressao rico sangue veias}} \\ \hline
 \# & Exemplos \\ \hline
19 & Veias \textit{levam} o \textit{sangue} para o \textit{coracao} e as \textit{arterias} \textit{levam} o \textit{sangue} do \textit{coracao} As \textit{veias} sao mais finas e as \textit{arterias} sao grossas e resistentes\\ \hline
78 & As \textit{veias} \textit{levam} \textit{sangue} ate o \textit{coracao} elas nao aguentam muita \textit{pressao} As \textit{arterias} \textit{levam} o \textit{sangue} do \textit{coracao} para o resto do \textit{corpo} pois aguentam maior \textit{pressao} e sao maiores\\ \hline
242 & As \textit{veias} \textit{levam} \textit{sangue} do \textit{corpo} para o \textit{coracao} onde ele possa ser bombeado novamente para o \textit{corpo} As \textit{arterias} saem do \textit{coracao} tornando se cada vez mais finas esses vasos \textit{levam} o \textit{sangue} nutrientes e oxigenio do \textit{coracao} para os tecidos\\ \hline
328 & As \textit{arterias} transportam o \textit{sangue} que sai do \textit{coracao} inicialmente \textit{rico} em O2 as diversas partes do \textit{corpo} as \textit{veias} recolhem esse \textit{sangue} \textit{rico} em CO2 e \textit{levam} de volta para o \textit{coracao} As \textit{arterias} possuem paredes mais grossas e as \textit{veias} possuem valvulas que impedem o \textit{sangue} de voltar\\ \hline
372 & As \textit{veias} \textit{levam} \textit{sangue} do \textit{corpo} ao \textit{coracao} e as \textit{arterias} do \textit{coracao} ao \textit{corpo} Alem disso as \textit{arterias} sao mais grossas que as \textit{veias} para suportar \textit{pressao}\\ \hline
444 & As \textit{veias} realizam o transporte do \textit{sangue} venoso \textit{rico} em CO2 no sentido do \textit{corpo} para o \textit{coracao} Ja as \textit{arterias} carregam o \textit{sangue} \textit{rico} em O2 do \textit{coracao} para o \textit{corpo} Alem disso as \textit{arterias} sao mais grossas que as \textit{veias} pois tem que aguentar a \textit{pressao} exercido pelos batimentos cardiacos que bombam o \textit{sangue}\\ \hline
456 & As \textit{arterias} sao vasos sanguineos responsaveis por conduzir o \textit{sangue} para fora do \textit{coracao} carregando \textit{sangue} artereal \textit{rico} em O2 As \textit{veias} sao responsaveis por conduzir o \textit{sangue} proveniente dos tecidos para o \textit{coracao} onde e \textit{rico} em CO2 e carrega \textit{sangue} venoso\\ \hline
479 & As \textit{veias} \textit{levam} o \textit{sangue} dos tecidos para o \textit{coracao} e possuem baixa \textit{pressao} ja as \textit{arterias} possuem alta \textit{pressao} e \textit{levam} \textit{sangue} do \textit{coracao} para o \textit{corpo}\\ \hline
\hline
\end{tabular}
\end{table}

Na Tabela \ref{tab-rubrics-exemplo} foram coletados os tópicos mais relevantes em cada uma das categorias para destacar os principais termos das respostas. No exemplo a resposta nota 3 fica evidente. Esta deve citar a relação entre o \textit{corpo} e o \textit{coração}, com a \textit{pressão} \textit{arterial} levando o sangue, e retornando pelas \textit{veias}. Então, a função do \textit{rubrics} é definir os padrões de resposta do sistema e, adicionalmente, criar relatórios que expliquem o formato avaliativo em apoio ao professor.

% ==============================================================================
% TCC - Nome do Aluno
% Capítulo 3 - Avaliação do Trabalho
% ==============================================================================
\chapter{Experimentos e Resultados}
\label{cap-experimentos}

Esse capítulo apresenta três séries de experimentos. A primeira apresenta a parte fundamental do aprendizado semi-supervisionado do sistema \textit{p}Nota, utilizando \textit{clusterização} para a identificação dos principais itens de resposta em cada base de dados. A segunda apresenta os métodos de classificação, a qualidade do aprendizado do sistema na predição de notas e sua adequação ao modelo esperado pelo tutor. Por fim, o terceiro módulo reflete como os modelos de resposta são formados pelo sistema e apresentados como feedback aos alunos e professores. Os experimentos foram realizados utilizando conjuntos de dados da literatura que apresentam diferentes características.

\section{Base de Dados}
Oito bases de dados foram selecionadas de acordo com a literatura, em português e inglês. Cada base de dados foi utilizada conforme as suas características. As bases de dados foram organizadas segundo o formato da nota, entre ordinais, discretas e contínua \cite{morettin2010}.

Em bases de dados com notas \textit{ordinais} o método avaliativo do tutor é dado de forma textual e categórica. A representação do rótulo não estabelece escalas para o sistema, não sendo possível mensurar a diferenças na escala \textit{a priori}. O modelo formado deve compreender as estruturas textuais de forma simbólica, caracterizando a essência de cada nível. Portanto, o classificador deve ser robusto para aprender a relevância das respostas pela equivalência de palavras-chave. Basicamente, é fundamental para o classificador produzir um modelo com as informações essenciais para a resposta receber tal categoria e reproduzir o modelo.

Por outro lado, outra situação acontece com bases de dados avaliados com notas contínuas. As notas \textit{contínuas} não apresentam níveis, mas sim intervalos numéricos. As respostas recebem notas de acordo com o intervalo avaliativo. Apesar de numérico, o fato da variável não definir uma categoria que represente a divergência entre respostas dificulta o aprendizado do modelo avaliativo. Ao sistema, isso torna subjetiva a espectativa de resposta subjetivo. Assim, esse tipo de atividade é avaliada por interpolação. Nesse caso, o sistema realiza uma regressão de acordo com os pontos conhecidos, gerando a nota pela referência ao grau de similaridade para as demais respostas.

Por fim, a avaliação \textit{discreta} numérica é a mais comum. Esse modelo favorece também os sistemas computacionais na criação da representação de resposta por categoria de nota. Ao tempo que a categoria induz a equivalência de todas as respostas ao qual foi associada. Assim, o sistema consegue mensurar equivalência e divergências pelos indícios de proximidade entre respostas avaliadas já conhecidas para além da mesma categoria. O desafio do sistema com este tipo de nota é criar um bom modelo de classificação que aprenda essa relação dupla. Para além da categoria das respostas, o sistema passa a ter que interpretar as informações fundamentais de cada classe e a escala de divergência para as demais categorias. A Tabela \ref{tab-datasets} apresenta os detalhes de cada \textit{dataset}, incluindo o número de questões, o total de respostas, o modelo avaliativo aplicado e a linguagem.

\begin{center}
\begin{table}[!h]
\begin{tabular}{r |c c c c} 
 \hline
 Base de Dados & Quest{\~o}es & Respostas & Modelo Avaliativo & Linguagem \\ \hline
 SEMEVAL2013 Beetle & 47 & 4380 & ordinal & Ingl{\^e}s \\
 SEMEVAL2013 SciEntsBank & 143 & 5509 & ordinal & Ingl{\^e}s \\
 Kaggle ASAP-SAS & 10 & 17043 & discreto & Ingl{\^e}s \\
 Powergrading & 10 & 6980 & discreto & Ingl{\^e}s \\
 UK Open University & 20 & 23790 & discreto & Ingl{\^e}s \\
 University of North Texas & 87 & 2610 & cont{\'i}nuo & Ingl{\^e}s \\
 Kaggle PTASAG & 15 & 7473 & discreto & Portugu{\^e}s \\
 Projeto Feira Liter{\'a}ria & 10 & 700 & discreto & Portugu{\^e}s \\
 VestUFES & 5 & 460 & cont{\'i}nuo & Portugu{\^e}s \\
 \hline
 \hline
\end{tabular}
\caption{Bases de dados e suas principais caracter{\'i}sticas.}
\label{tab-datasets}
\end{table}
\end{center}

A Tabela \ref{tab-datasets} descreve os oito \textit{datasets} utilizados nos experimentos deste capítulo. Através das características apresentadas, sabendo que cada \textit{dataset} contém uma quantidade regular de respotas, observamos a grande diversidade de quantidade de respostas por questão. Com questões de 30 até mais de 1800 respostas. No total, esse \textit{corpora} apresenta um total de 337 questões e 61.268 respostas. Cada base de dados e sua descrição completa é apresentada a seguir:


\subsection{Base de Dados \textit{Beetle} do \textit{SEMEVAL'2013 : Task 7} \textit{(Inglês)}}
\label{beetle-db}

\textit{Beetle} \cite{dzikovska2012} é um dos \textit{datasets} utilizados durante o \textit{International Workshop on Semantic Evaluation - SEMEVAL'2013}. O \textit{SEMEVAL} seleciona anualmente uma série de desafios em análise semântica e apresenta no formato de competição. O \textit{corpus Beetle} foi selecionado para a \textit{Task 7: The Joint Student Response Analysis and 8th Recognizing Textual Entailment Challenge} \cite{dzikovska2013}. Portanto, a competição consistia em duas propostas. A primeira é a análise e avaliação das respostas obtidas e a segunda o reconhecimento da relação textual entre as respostas coletadas e a expectativa de resposta do professor.

Esse \textit{dataset} consiste em uma coleção de interações entre estudantes e o sistema \textit{Beetle II}. Beetle II é um Sistema Tutor Inteligente (STI) para aprendizado de conhecimentos básicos em Eletricidade e Eletrônica do Ensino Médio. Os alunos foram acompanhados durante 3 a 5 horas para preparar materiais, construir e observar circuitos no simulador e interagir com o STI. Esse sistema faz questões aos alunos, avalia as respostas e envia \textit{feedbacks} via \textit{chat}. Na construção deste \textit{dataset} foram acompanhados 73 estudantes voluntários da \textit{Southeastern University} dos Estados Unidos.

Foram aplicadas questões categorizadas em dois tipos factuais e explicativas. As questões factuais requerem que o aluno nomeie diretamente determinados objetos ou propriedades. Equanto isso, as questões explicativas demandam que o aluno desenvolva a resposta em uma ou duas frases. Para a formação do \textit{dataset} foram adicionadas apenas as atividades do segundo tipo, pois representam maior complexidade para sistemas computacionais. No total foram selecionadas 47 questões com 4380 respostas. A avaliação foi feita conforme o domínio demonstrado sobre o assunto em cinco categorias: \textit{correct}, \textit{partially-correct-incomplete}, \textit{contradictory}, \textit{irrelevant} e \textit{non-domain}. Durante a anotação o coeficiente \textit{Kappa} obtido foi de 69\% de concordância.


\subsection{Base de Dados \textit{SciEntsBank} do \textit{SEMEVAL'2013 : Task 7} \textit{(Inglês)}}
\label{scientsbank-db}

O \textit{corpus Science Entailments Bank (SciEntsBank)} \cite{dzikovska2012} é um dos \textit{datasets} utilizados durante o \textit{International Workshop on Semantic Evaluation - SEMEVAL'2013} \cite{dzikovska2013}, com foco na avaliação de sistemas conforme a sua capacidade de análise e exploração semântica da linguagem. É uma base de dados formadas pela avaliação de questões da disciplina de Ciências. Na avaliação 16 assuntos distintos são abordados entre ciências físicas, ciências da terra, ciências da vida, ciências do espaço, pensamento científico e tecnologia. 

As questões são parte da \textit{Berkeley Lawrence Hall of Science Assessing Science Knowledge (ASK)} com avaliações padronizadas de acordo com o material de apoio \textit{Full Option Science System (FOSS)}. Participaram estudantes dos Estados Unidos de terceira a sexta série, coletando em torno de 16 mil respostas. Porém, dentre as questões de preenchimento, objetivas e discursivas, foram utilizadas apenas as discursivas, que requisitavam explicações dos alunos segundo o tema. As respostas foram graaduadas em cinco notas ordinais: \textit{correct}, \textit{partially-correct-incomplete}, \textit{contradictory}, \textit{irrelevant} e \textit{non-domain}. Portanto, o \textit{SciEntsBank} consiste em um conjunto com 143 questões selecionadas e 5509 respostas. No processo de avaliação foi observado o coeficiente \textit{Kappa} com 72.8\% de concordância.

\begin{comment}
[('contradictory', 557), ('correct', 2241), ('irrelevant', 1248), ('non_domain', 26), ('partially_correct_incomplete', 1437)] 5509 SCIENTSBANK

[('contradictory', 1160), ('correct', 1841), ('irrelevant', 130), ('non_domain', 218), ('partially_correct_incomplete', 1031)] 4380 BEETLE 
\end{comment}

\subsection{Base de Dados do Concurso ASAP-SAS no \textit{Kaggle} \textit{(Inglês)}}
\label{kaggle-db}

A base de dados \textit{ASAP - SAS}, \textit{Automated Student Assessment Prize - Short Answer Scoring} é uma competição proposta pela \textit{Hewllet Foundation} na plataforma \textit{Kaggle} \footnote{The Hewlett Foundation - Short Answer Scoring: https://www.kaggle.com/c/asap-sas}. A \textit{ASAP} consistiu em três fases:

\begin{itemize}
\item Fase 1:  Demonstração em respostas longas (redações); 
\item Fase 2:  Demonstração em respostas curtas (discursivas);
\item Fase 3:  Demonstração simbólica matemática/lógica (gráficos e diagramas).
\end{itemize}

O objetivo da competição foi descobrir novos sistemas de apoio ao desenvolvimento de escolas e professores. Especificamente, as três fases destacam a atividade lenta e de alto custo de avaliar manualmente testes, mesmo que com padrões bem definidos. Uma consequência disso é a redução do uso de questões discursivas nas escolas, dando preferência para as questões objetivas para evitar a sobrecarga de trabalho. Isso evidencia uma gradativa redução da capacidade dos professores em incentivar o pensamento crítico e as habilidades de escrita. Portanto, os sistemas de apoio, são uma possível solução para suportar os métodos de correção, avaliação e feedback ao conteúdo textual dos alunos.

Neste contexto, a competição apresentou 10 questões multidiciplinares, de ciências à artes. Estão distribuidas 17043 respostas de alunos dentre essas atividades. Para chegar nessa quantidade, foram selecionadas por volta de 1700 respostas dentre 3000 respostas em cada atividade. Cada resposta tem aproximadamente 50 palavras. A primeira avaliação foi dada pelo primeiro especialista como nota final e a segunda nota foi atribuída apenas para demonstrar o nível de confiança da primeira nota. A avaliação apresentada por dois especialistas apresentou concordância de 90\% no coeficiente \textit{Kappa}.


\subsection{Base de Dados \textit{Powergrading} \textit{(Inglês)}}
\label{powergrading-db}
Elaborado através do \textit{United States Citizenship Exam} (USCIS) em 2012, a base de dados \textit{Powergrading} contém 10 questões e 6980 respostas \cite{basu2013}. Desenvolvida originalmente para destacar a possibilidade de avaliação massiva, o \textit{dataset} selecionou 698 respostas para cada uma das questões. As respostas foram geradas com \textit{Amazon Mechanical Turk}, serviço remoto de análise manual de conteúdo para anotação da \textit{Amazon}. Foi coletada por um grupo de pesquisa da \textit{Microsoft}\footnote{Powergrading: https://www.microsoft.com/en-us/download/details.aspx?id=52397} e cada questão acompanha um modelo de resposta e as respostas recebidas para cada questão.  Além disso, acompanha o \textit{dataset} outras 10 questões não avaliadas.

Foram requisitadas respostas com poucas palavras, atingindo no máximo uma ou duas sentenças. Por conta disso, os resultados são respostas muito curtas com 4 palavras em média. Em geral, por conta da convergência, vários padrões de resposta se repetem \cite{riordan2017}. Com avaliações binárias, 1 para resposta correta e 0 para incorreta, cada resposta apresenta avaliações de três diferentes tutores. Apesar de alguns trabalhos assumirem um dos avaliadores como padrão, utilizamos como modelo de avaliação a resultante da comparação entre os três. Apesar de não ter complexidade linguística, ocorreu contradição entre os avaliadores em 470 respostas. Em valores percentuais, isso representa 7\% do total de respostas do conjunto de dados.

\begin{comment}
Respostas não coincidentes em cada questão
q1 - 1 q2 - 12 q3 - 145 q4 - 45 q5 - 18 q6 - 52 q7 - 20 q8 - 17 q13 - 108 q20 - 52
\end{comment}

\subsection{Base de Dados da \textit{UK Open University} \textit{(Inglês)}}
\label{openunv-db}

A base de dados da \textit{UK Open University} é um conjunto de questões coletadas na disciplina de introdução à ciências, denominada \textit{S103 - Discovering Science} \cite{jordan2012}. O foco do conjunto de atividades são abordagens em questões factuais bem concisas, não excedendo 20 palavras. Os alunos receberam as atividades através do ambiente da \textit{Intelligent Assessment Technologies (IAT)}, o \textit{FreeText Author}. O \textit{FreeText Author} foi utilizado como um método de \textit{CAA} de modo interativo e com resultado automático analizando a resposta do aluno segundo os padrões de resposta conhecidos. O sistema permitiu uma sequência de envios e apresentava comentários da resposta como \textit{feedback} para os alunos. Dependendo da complexidade da resposta, o tempo de retorno dos resultado varia muito entre alguns poucos minutos até mais do que um dia.

Dentre as 20 questões, esse \textit{dataset} apresenta diferentes quantidades de respostas entre 511 e 1897. A avaliação é discreta e binária, definindo cada resposta como correta ou incorreta. Não existe notas intermediárias, representando diretamente se o aluno atendeu ou não os requisitos da resposta.


\subsection{Base de dados da \textit{University of North Texas} \textit{(Inglês)}}
\label{ntexasunv-db}

O \textit{dataset} da \textit{University of North Texas - UNT} \cite{mohler2011}, conhecido como \textit{Texas dataset} \footnote{Texas Dataset: https://web.eecs.umich.edu/{\textasciitilde}mihalcea/downloads.html}, é uma coleção de questões discursivas extraída no curso de Ciência da Computação. Composta por 80 atividades únicas, esse conjunto é composto por dez listas de exercícios com até sete questões e dois testes com dez questões cada. Foram aplicados em um ambiente virtual de aprendizagem durante a disciplina de Estrutura de Dados para 31 alunos. No total o \textit{dataset} é composto por 2273 respostas de alunos dentre as 80 atividades.

A avaliação foi feita com cinco notas discretas, de 5 equivalente a resposta perfeita até 0 completamente incorreta. Foram avaliadas por dois avaliadores independentes, estudantes do curso de Ciência da Computação. Para os autores, o modelo seguido pelo sistema deve ser a resultante da média ente os avaliadores, em intervalo contínuo. Dentre as notas atribuídas, 57,7\% das respostas receberam a mesma nota. Enquanto isso, um nível de diferença entre as notas representou 22,9\% do total de respostas. Foi constatado também que, dentre as diferenças na avaliação, o avaliador 1 atribuia maiores notas 76\% das vezes.

\subsection{Base de dados \textit{PTASAG} no \textit{Kaggle} \textit{(Português)}}
\label{ptasag-db}

A \textit{PTASAG - Portuguese Automatic Short Answer Grading Data} é uma base de dados brasileira apresentada por \cite{galhardi2018b} e disponibilizada na plataforma \textit{Kaggle} \footnote{PT-ASAG-2018: https://www.kaggle.com/lucasbgalhardi/pt-asag-2018}. Foi coletada pela Universidade Federal do Pampa - Unipampa em conjunto com cinco professores de biologia do Ensino Fundamental. Foram criadas 15 atividades com base no sistema Auto-Avaliador CIR. Em biologia, os tópicos abordados foram sobre o corpo humano. Cada questão acompanha uma lista de conceitos, as respostas avaliadas e as respostas candidatas criadas pelos professores como referência. Foram criadas entre duas e quatro respostas candidatas contendo entre três e seis palavras-chave.

As atividades foram aplicadas ao Ensino Fundamental para 326 estudantes de 12 a 14 anos do 8\textsuperscript{\b{o}} e 9\textsuperscript{\b{o}} ano. Somados a estes, também foram aplicados a 333 alunos do Ensino Médio de 14 a 17 anos. As respostas foram avalidas por 14 estudantes de uma turma do último ano, considerando uma escala de notas de 0 a 3:

\begin{itemize}
\item Nota 0: Majoritariamente incorreta, fora de tópico ou sem sentido;
\item Nota 1: Incorreta ou incompleta mas com trechos corretos;
\item Nota 2: Correta mas com importantes trechos faltantes;
\item Nota 3: Majoritariamente correta apresentando os principais pontos.
\end{itemize}

No total, participaram 659 estudantes com um total de 7473 respostas. Cada uma das 15 questões apresenta entre 348 e 615 respostas. Apenas 4 questões foram avalidas por mais de um avaliador para verificar a concordância entre avaliadores. O coefficiente \textit{Kappa} observado foi de, em média, 53.25\%.


\subsection{Base de Dados do Projeto Feira Literária das Ciências Exatas \textit{(Português)}}
\label{findes-db}

É um conjunto de dados coletados durante o Projeto Feira Literária das Ciências Exatas \cite{nascimento2020}. As questões foram obtidas durante uma Atividade Experimental Problematizada por meio de um livro paradidático, ou seja, cujo objetivo primário não é o apoio didático. O livro escolhido foi \textit{A Formula Secreta} de David Shephard. 

Conforme o livro, os professores formularam 10 atividades e ministraram para 70 alunos do 5º ano de Ensino Fundamental. Essas atividades ministradas descreviam situações práticas de Química básica. No total, o conjunto de dados conta com 10 questões, 700 respostas e suas respectivas avaliações.


\subsection{Base de Dados do Vestibular UFES \textit{(Português)}}
\label{vest-ufes-db}

A base de dados VestUFES \cite{pissinati2014} é uma amostra das questões discursivas de Português do vestibular da UFES em 2012. A amostra selecionada contém 460 respostas divididas igualmente entre as 5 questões de língua portuguesa, também referentes a respostas dadas por 92 diferentes alunos.

Cada resposta foi avaliada por dois avaliadores. De acordo com o vestibular da universidade, os a avaliadores atribuíram notas entre 0 e 2 pontos em cada questão, totalizando 10 na soma da prova. Caso houvesse divergências de mais de 1 ponto entre as correções um terceiro avaliador era acionado para reavaliar a coerência das notas. A nota das respostas do \textit{dataset} foram redimensionadas pelo autor para o intervalo de 0 a 10 pontos. Na nova escala, as diferenças observadas entre os avaliadores foi de, em média, 1,38 pontos com desvio padrão de 1,75.


\begin{comment}
\subsection{Base de Dados das Disciplinas UFES \textit{(Português)} \label{disciplinas-ufes-db}
Essa base de dados foi coletada de algumas disciplinas ministradas na Universidade Federal do Espírito Santo - UFES entre 2015 e 2016 através do Moodle do Laboratório Computação de Alto Desempenho - LCAD. Entre as disciplinas estão Metodologia e Técnicas de Pesquisa Científica, Filosofia e Tecnologia da Informação II. Totalizando 45 atividades, neste \textit{dataset} foram recebidas 1162 submissões com média 25,82 e desvio padrão de 13,54 respostas por atividade.

O diferencial dessa base de dados é a mudança de características nas respostas encontradas nas múltiplas disciplinas, professores e alunos. Observam-se alterações consideráveis quanto ao tamanho, número de grupos, critério avaliativo ou objetividade. 
\end{comment}


\section{Experimentos}
\label{sec-experimentos}

A série de experimentos realizados têm como intúito demonstrar o uso do \textit{p}Nota, em conjunto com o professor, de forma a criar modelos com bom desempenho na atribução de notas. Além disso, através dos \textit{datasets} conhecidos na literatura, podemos comparar a proposta de análise de estruturas textuais com os principais trabalhos publicados. Em uma primeira comparação avaliaremos a qualidade dos \textit{clusters} formados.

\subsection{Resultados de \textit{Clusterização}}
\label{sec-res-clustering}

Em uma primeira análise, observamos os resultados dos \textit{clusters} formados para os conjuntos de atividades através do processo de otimização \cite{spalenza2019}. Analisamos os resultados segundo três métricas de informação e o grau de similaridade entre grupos. Para isso, utilizamos as métricas ARI, NMI e CA, descritas na Seção \ref{subsec-clusterizacao}. Os resultados médios obtidos para os conjuntos de dados são apresentados na Tabela \ref{tab-clstr-index}.

\begin{table}[!h]
\begin{center}
\begin{tabular}{r | c c c}
 \hline
 Base de Dados & ARI & NMI & CA \\ \hline 
SEMEVAL2013 Beetle & 0,0590 & 0,1291 & 0,5521 \\
SEMEVAL2013 SciEntsBank & 0,0246 & 0,1046 & 0,5747 \\
Kaggle ASAP-SAS & -0,0123 & 0,0190 & 0,5225 \\
Powergrading & 0,1009 & 0,1000 & 0,9025 \\
UK Open University  & 0,0424 & 0,0564 & 0,7604 \\
University of North Texas & 0,0540 & 0,1910 & 0,5869 \\
Kaggle PTASAG & 0,0577 & 0,0838 & 0,4863 \\
Projeto Feira Liter{\'a}ria & 0,0301 & 0,1088 & 0,5529 \\
VestUFES & & & \\
 \hline
\end{tabular}
\end{center}
\caption{Bases de dados e índices qualitativos de \textit{clusterização}.}
\label{tab-clstr-index}
\end{table}

A Tabela \ref{tab-clstr-index} demonstra que, apesar da baixa relação entre \textit{clusters} e sua representação direta como rótulos, temos excelentes resultados quanto à classificação. Isso indica que o sistema apresenta uma capacidade de particionamento dos dados em núcleos distintos de uma mesma classe. Na prática, uma categoria é representada por mais que um \textit{cluster}, seja formado por \textit{outliers} (até 10 amostras), grupos com pares de categorias ou grupos com padrões consistentes. Um quarto tipo seriam os grupos com padrões aleatórios agrupados, em geral não desejável. Neste último caso, os resultados não contribuem muito para a passagem de informações entre as etapas de amostragem por \textit{clusterização} e classificação. Com alinhamento direto ao método de \textit{clusterização} proposto pelos autores, o \textit{dataset Powergrading} apresentou alto desempenho logo de partida com nosso método. Por outro lado, com grande gradação de notas, os \textit{datasets} do \textit{SEMEVAL Beetle} e \textit{SciEntsBank} indicam grande complexidade quanto a homogeneidade dos grupos. Para destacar essa distinção entre os grupos e o conteúdo, apresentamos a diferença entre \textit{centróides} na Figura \ref{fig-hmPowergrading}.


\begin{figure}[!h]
\begin{minipage}[t]{.5\textwidth}
\centering
\includegraphics[width=\textwidth]{figuras/hmq2.png} 
\end{minipage}%
\begin{minipage}[t]{.5\textwidth}
\centering
\includegraphics[width=\textwidth]{figuras/hmq4.png}
\end{minipage}%
\hfill
\begin{minipage}[t]{.5\textwidth}
\centering
\includegraphics[width=\textwidth]{figuras/hmq19.png}
\end{minipage}%
\begin{minipage}[t]{.5\textwidth}
\centering
\includegraphics[width=\textwidth]{figuras/hmq20.png}
\end{minipage}
\caption{Similaridade entre \textit{centróides} para as atividades q2, q4 e q19 e q20 em \textit{Powergrading}.}
\label{fig-hmPowergrading}
\end{figure}

Como observamos nas imagens apresentadas na Figura \ref{fig-hmPowergrading}, existe a formação de grupos muito coesos. Essa coesão é explícita pela divergência do item médio (\textit{centróide}) os demais grupos. Como vemos através da matriz, em geral os índices de similaridade são menores que 0,25. Além disso, as atividades q2, q4, q19 e q20 apresentam respectivamente CA de 0,878, 0,805, 1,000 e 0,973. Isso demonstra a formação grupos divergentes mas com forte relação com a classe de nota. Para extração de padrões isso significa resultados de alta qualidade diretamente nos clusters formados. De certa forma este caso é incomum, ao qual esperamos lidar até com resultados de menor consistência nos \textit{clusters}. Denominamos agrupamentos consistentes aqueles que estão diretamente alinhados com a categoria atribuída. Entretanto, o cenário encontrado neste \textit{dataset} é o melhor possível, com padrões de resposta bem dispostos em \textit{clusters}. O caso oposto é dado pela diferença entre \textit{centróides} na Figura \ref{fig-hmSciEntsBank}.


\begin{figure}[!h]
\begin{minipage}[t]{.5\textwidth}
\centering
\includegraphics[width=\textwidth]{figuras/hmEM16b.png} 
\end{minipage}%
\begin{minipage}[t]{.5\textwidth}
\centering
\includegraphics[width=\textwidth]{figuras/hmEM35.png}
\end{minipage}%
\hfill
\begin{minipage}[t]{.5\textwidth}
\centering
\includegraphics[width=\textwidth]{figuras/hmEM45c.png}
\end{minipage}%
\begin{minipage}[t]{.5\textwidth}
\centering
\includegraphics[width=\textwidth]{figuras/hmFN17a.png}
\end{minipage}
\caption{Similaridade entre \textit{centróides} para as atividades EM-16b, EM-35, EM-45c e FN-17a em \textit{SciEntsBank}.}
\label{fig-hmSciEntsBank}
\end{figure}

Como observamos na Figura \ref{fig-hmSciEntsBank}, as atividades EM-16b, EM-35, EM-45c e FN-17a apresentam regiões de maior similaridade. Diferentemente do que foi notado na atividade anterior, neste temos algumas áreas coincidentes, acima de 0,35 destacadas em azul. As áreas em azul indicam um grau grande de similaridade entre centróides dos grupos e, por consequência, são compatíveis em certo nível. Por conta disso, podemos considerar que tais grupos podem ter diferentes perspectivas de uma mesma classe ou mesclar diferentes classes. Por consequência, em ambos os casos, torna-se maior a responsabilidade do classificador \textit{a posteriori}, para designar padrões refinados para tais \textit{clusters}. Por outro lado, os resultados em CA para as atividades EM-16b, EM-35, EM-45c e FN-17a são de 0,425, 0,625, 0,825 e 0,700 respectivamente. Notamos que as quatro atividades representam diferentes níveis em CA, inclusive com avaliações não coincidentes. Alinhado a isso, é fundamental que o classificador seja responsável por refinar o reconhecimento de padrões estabelecido na \textit{clusterização}, elevando o nível dos resultados alcançados.

\subsection{Resultados de \textit{Classificação}}
\label{sec-res-classificacao}

Após resultados bem sucedidos na \textit{clusterização}, com CA por voto majoritário com médias superiores à 50\%, realizamos a análise dos resultados de classificação. Tais experimentos caracterizam-se pela simetria com a maioria das publicações e desafios lançados em avaliação de respostas discursivas \cite{burrows2015}. Nestes encontramos um particionamento de 75\% de respostas selecionadas para treinamento e validação dos modelos e 25\% para avaliação do desempenho das propostas. No caso deste trabalho, as partições iniciais dos \textit{datasets} foram desconsideradas por conta da seleção de amostras de modo semi-supervisionado. Portanto, neste trabalho a amostragem dos 75\% foi realizada conforme a distribuição dos itens nos \textit{clusters} formados, como relatado em detalhes na Seção \ref{sec-amostragem}. Portanto, foram mantidos tais percentuais como limiar superior dos sistemas. Vale ressaltar que esse processo ainda representa uma redução considerável do trabalho de correção nos mesmos 25\%, apesar da maioria dos dados ser designada para avaliação do professor.

Seguindo a característica da avaliação, vamos apresentar os resultados obtidos segundo a forma avaliativa adotada pelos professores. De forma mais complexa, temos os conjuntos de dados \textit{Beetle} e \textit{SciEntsBank} avaliados em 5 categorias textuais (ordinais) que indicam a completude da resposta conforme a expectativa do professor. Nesse aspecto, os \textit{datasets} do evento \textit{SEMEVAL' 2013}, apresentam 3 níveis de desafios. O primeiro nível é a avaliação de respostas não conhecidas, selecionadas aleatóriamente no conjunto de respostas (\textit{Unseen Answers}). O segundo nível compreende a correção de respostas em questões desconhecidas, ainda em um determinado domínio (\textit{Unseen Questions}). E, por fim, o terceiro nível está relacionado a análise de respostas em um domínio desconhecido (\textit{Unseen Domain}). Assim como a maioria dos sistemas SAG, o desafio que se enquadra no tópico aqui abordado é o primeiro (\textit{Unseen Answers}), avaliando conjuntos de respostas dentro de um mesmo tópico. 

Com destaque para o desbalanceamento dos dados \cite{dzikovska2013}, ambos os \textit{datasets} foram anotados em 5 categorias: \textit{correct}, \textit{partially-correct-incomplete}, \textit{contradictory}, \textit{irrelevant} e \textit{non-domain}. Evidenciamos ainda a complexidade, inclusive semântica, para separar as três categorias inferiores, \textit{contradictory}, \textit{irrelevant} e \textit{non-domain}. Utilizando os 6 classificadores descritos na Seção \ref{subsec-classificacao}, apresentamos os resultados obtidos na Tabela \ref{tab-SEMEVAL75}.

\begin{table}[!h]
\begin{center}
\begin{tabular}{l r r r r r r r}
    \hline
    \multicolumn{7}{l}{\textbf{Beetle}} &  (5 Categorias) \\ \hline
     & \multicolumn{7}{c}{M{\'e}tricas} \\

     & ACC & PRE & REC & F1(m) & F1(w) & Kappa(ln) & Kappa(qu) \\ \cline{2-8}
DTR & 56,26\% & 31,22\% & 31,27\% & 30,43\% & 55,32\% & 0,1026 & 0,0802 \\
GBC & 59,39\% & \textbf{36,56\%} & 35,08\% & 34,64\% & \textbf{59,28\%} & \textbf{0,1841} & \textbf{0,1990} \\
KNN & 55,79\% & 32,95\% & 35,09\% & 32,54\% & 53,73\% & 0,1167 & 0,1350 \\
RDF & 58,85\% & 36,53\% & 36,31\% & \textbf{34,95\%} & 56,03\% & 0,1362 & 0,1402 \\
SVM & 58,05\% & 31,50\% & 35,01\% & 31,35\% & 52,36\% & 0,0850 & 0,1046 \\
WSD & \textbf{59,52\%} & 35,62\% & \textbf{36,47\%} & 34,66\% & 56,18\% & 0,1391 & 0,1440 \\

    \hline
    \\
    \\
    \hline
    \multicolumn{7}{l}{\textbf{SciEntsBank}} &  (5 Categorias) \\ \hline
     & \multicolumn{7}{c}{M{\'e}tricas} \\

     & ACC & PRE & REC & F1(m) & F1(w) & Kappa(ln) & Kappa(qu) \\ \cline{2-8}
DTR & 44,45\% & \textbf{31,34\%} & 32,54\% & 30,45\% & 43,76\% & 0,1678 & 0,1433 \\
GBC & \textbf{45,85\%} & 31,23\% & \textbf{33,43\%} & \textbf{30,71\%} & \textbf{43,96\%} & \textbf{0,1720} & \textbf{0,1454} \\
KNN & 43,22\% & 27,19\% & 31,97\% & 27,78\% & 39,17\% & 0,0995 & 0,1043 \\
RDF & 43,78\% & 25,91\% & 31,26\% & 26,99\% & 38,98\% & 0,0994 & 0,0874 \\
SVM & 41,93\% & 21,36\% & 30,27\% & 23,53\% & 34,16\% & 0,0361 & 0,0345 \\
WSD & 41,89\% & 25,68\% & 29,09\% & 25,73\% & 38,09\% & 0,0607 & 0,0309 \\

    \hline
    \hline
\end{tabular}
\end{center}
\caption{Resultados dos seis classificadores testados nos \textit{datasets} do \textit{SEMEVAL' 2013}.}
\label{tab-SEMEVAL75}
\end{table}

A Tabela \ref{tab-SEMEVAL75} caracteriza o desempenho do sistema com os seis classificadores testados. Fica evidente que a performance do sistema é superior no \textit{Beetle} em relação ao \textit{SciEntsBank}. Na primeira base de dados, há um equilíbrio entre os classificadores testados, com GBC, WSD, DTR e RDF alcançando F1-ponderado entre 55\% e 60\%. Os classificadores KNN e SVM apresentam uma ligeira queda neste índice, com 53,73\% e 52,36\% respectivamente. Enquanto isso, na segunda, temos resultados consistentes do GBC e DTR, com F1-ponderado de 43\% e F1-macro acima de 30\%. Os demais classificadores apresentaram F1-ponderado abaixo de 40\%. Deste modo, os valores em análise destacam características da formação dos conjuntos de dados. O \textit{dataset Beetle} apresenta cerca de 84 respostas por questão. Enquanto isso, o \textit{SciEntsBank} contém apenas 37 respostas em média por questão, ficando evidente a diferença entre eles. Os resultados obtidos também estão ilustrados na Figura \ref{fig-semeval75}.

\begin{figure}[!h]
\centering
\includegraphics[width=\textwidth]{figuras/Beetle-75.png}

\includegraphics[width=\textwidth]{figuras/SciEntsBank-75.png}
\caption{Resultados obtidos nos \textit{datasets Beetle} e \textit{SciEntsBank} pelos classificadores em comparação com os principais encontrados na literatura.}
\label{fig-semeval75}
\end{figure}

Na Figura \ref{fig-semeval75}, os gráficos destacam os valores obtidos nas métricas F1-macro e F1-ponderado conforme os principais resultados da literatura \cite{dzikovska2013, ramachandran2015a,sahu2020}. Conforme os trabalhos da literatura, evidenciamos a distância dos resultados obtidos em relação ao que foi apresentado como o \textit{estado da arte} neste cenário. Apesar disso, ambos os resultados são superiores ao \textit{baseline} proposto. Nesta perspectiva, abordagens com a criação de modelos de resposta, utilizando informações adicionais sobre o tema ou expressões regulares, sobressaem com melhor desempenho mas, em geral, demandam maior esforço do professor \cite{ramachandran2015a, sahu2020}. No entanto, isto indica que tais estratégias seguem um único viés de resposta, tornando-se menos efetivas ao lidar com variações linguísticas \cite{filighera2020}.

Ainda, temos dentre as 4380 respostas do \textit{Beetle} 1841 anotadas como \textit{correct}, 1160 como \textit{contradictory} e 1031 como \textit{partially-correct-incomplete}. Por outro lado, apenas 218 foram avaliadas como \textit{non-domain} e 130 como \textit{irrelevant}. Considerando ainda a distribuição de classes, a situação é agravada em relação ao \textit{SciEntsBank}. Dentre as 5509 respostas, 2241 foram dadas como \textit{correct}, 1437 como \textit{partially-correct-incomplete}, 1248 como \textit{irrelevant} e 557 como \textit{contradictory}. Só constam neste \textit{dataset} 26 respostas anotadas como \textit{non-domain} dentre as 143 questões. Notoriamente, são poucas amostras para algumas categorias que se destacam quando vemos que, em média, o primeiro \textit{dataset} apresenta 93 respostas por questão enquanto o segundo apresenta apenas 38 respostas. Portando, apesar da complexidade de avaliar tal questão, os resultados são positivos, aprimorando resultados esperados conforme a distribuição de \textit{clusters}.

Em outra perspectiva, com dados discretos, analisamos os resultados dos \textit{datasets Open University}, \textit{Powergrading} e \textit{Kaggle ASAP-SAS}. Inicialmente abordaremos o \textit{dataset} da \textit{Open University}, por conta do formato de anotação. Neste conjunto de dados a cada resposta foi designada a nota 0 ou nota 1, como respostas corretas ou incorretas. Bem distinta dos \textit{datasets Beetle} e \textit{SciEntsBank}, este conjunto contém mais de 23 mil respostas e, em média, 1190 respostas para cada questão. Isso impacta diretamente na construção de modelos de resposta, com uma variedade de padrões para uma mesma classe, sendo possível a identificação de núcleos de resposta bem consistentes segundo a simetria da classe. Os resultados apresentados na Tabela \ref{tab-OU75} refletem justamente este aspecto.

\begin{table}[!h]
\begin{center}
\begin{tabular}{l r r r r r r r}
    \hline
    \multicolumn{7}{l}{\textbf{Open University}} &  (2 Categorias) \\ \hline
     & \multicolumn{7}{c}{M{\'e}tricas} \\ \cline{2-8}

     & ACC & PRE & REC & F1(m) & F1(w) & Kappa(ln) & Kappa(qu) \\ \cline{2-8}
DTR & 95,39\% & 90,37\% & \textbf{91,96\%} & 90,83\% & 94,86\% & 0,8053 & 0,8053 \\
GBC & \textbf{96,15\%} & \textbf{93,69\%} & 91,68\% & \textbf{92,01\%} & \textbf{95,51\%} & \textbf{0,8317} & \textbf{0,8317} \\
KNN & 89,41\% & 84,82\% & 82,60\% & 82,60\% & 88,52\% & 0,6264 & 0,6264 \\
RDF & 92,69\% & 89,41\% & 89,04\% & 86,86\% & 92,92\% & 0,7344 & 0,7344 \\
SVM & 86,46\% & 79,15\% & 77,07\% & 74,81\% & 82,98\% & 0,5093 & 0,5093 \\
WSD & 80,62\% & 74,53\% & 71,29\% & 70,16\% & 78,43\% & 0,3788 & 0,3788 \\

    \hline
    \hline
\end{tabular}
\end{center}
\caption{Resultados de classificação para o \textit{dataset OpenUniversity}.}
\label{tab-OU75}
\end{table}

É evidente, através dos resultados em destaque na Tabela \ref{tab-OU75}, que a grande quantidade de padrões de nota aumentam consideravelmente a qualidade de classificação. Com ACC de 96,15\%, o algoritmo GBC captura detalhes nas respostas e formas de avaliação muito próximas do modelo do professor. Por conta disso, podemos dizer que a captura das estruturas de linguagem analisadas foram de alta qualidade. Com isso, o sistema um classificador retorna um excelente modelo de avaliação, compatível com as expectativas do professor. De acordo com as classes, os classificadores GBC, RDF e DT apresentam F1-ponderado acima de 90\%. A Figura \ref{fig-OU75} apresenta os resultados obtidos neste \textit{dataset} em relação aos descritos pelos autores em sua publicação \cite{butcher2010}.

\begin{figure}[!h]
\centering
\includegraphics[width=\textwidth]{figuras/OU-75.png}
\caption{Comparação do sistema dos autores do \textit{dataset Open University} em relação ao \textit{p}Nota.}
\label{fig-OU75}
\end{figure}

A Figura \ref{fig-OU75} caracteriza a proximidade do modelo GBC do \textit{p}Nota com o \textit{IAT}, sistema aplicado na \textit{Open University}. Como o artigo destaca, o sistema tem conhecimento sobre o conteúdo e regras de associação de respostas com o tema para a produção de \textit{feedbacks} direcionados. Portanto, tal trabalho contém uma base de conhecimento sobre o tema para além do que é formado pelas atividades. Entretanto, apenas com os exemplos anotados, o \textit{p}Nota produz um modelo consistente, com desempenho equivalente aos modelos voltados ao tema.

Anotado de forma similar ao \textit{Open University}, o \textit{dataset Powergrading} também contém respostas avaliadas de forma binária (correto 1 ou incorreto 0). Porém, o \textit{Powergrading} é avaliado por três avaliadores. Com as notas binárias, o objetivo é atender o resultado coincidente entre os avaliadores humanos. O desempenho obtido pelo \textit{p}Nota neste \textit{dataset} é caracterizado na Tabela \ref{tab-PG75}.

\begin{table}[!h]
\begin{center}
\begin{tabular}{l r r r r r r r}
    \hline
    \multicolumn{7}{l}{\textbf{Powergrading}} &  (2 Categorias) \\ \hline
     & \multicolumn{7}{c}{M{\'e}tricas} \\ \cline{2-8}

     & ACC & PRE & REC & F1(m) & F1(w) & Kappa(ln) & Kappa(qu) \\ \cline{2-8}
DTR & 99,37\% & 92,62\% & 91,91\% & 92,19\% & 99,27\% & 0,8444 & 0,8444 \\
GBC & \textbf{99,49\%} & 93,77\% & \textbf{91,97\%} & \textbf{92,53\%} & \textbf{99,37\%} & \textbf{0,8515} & \textbf{0,8515} \\
KNN & 99,31\% & 89,66\% & 90,00\% & 89,82\% & 98,98\% & 0,8000 & 0,8000 \\
RDF & 99,37\% & \textbf{94,68\%} & 90,50\% & 90,75\% & 99,11\% & 0,8173 & 0,8173 \\
SVM & 99,37\% & \textbf{94,68\%} & 90,50\% & 90,75\% & 99,11\% & 0,8173 & 0,8173 \\
WSD & 99,03\% & 90,39\% & 90,32\% & 90,33\% & 98,90\% & 0,8074 & 0,8074 \\

    \hline
    \hline
\end{tabular}
\end{center}
\caption{Resultados de classificação para o \textit{dataset Powergrading}.}
\label{tab-PG75}
\end{table}

A Tabela \ref{tab-PG75} aponta o desemepenho dos classificadores no \textit{dataset Powergrading}, inclusive superiores a média de 90\% em CA (voto majoritário). Nesse aspecto, todos os 6 classificadores apresentam ACC e F1-ponderado acima de 99\%, reflexo direto da homogeneidade dos \textit{clusters}formados na etapa anterior \cite{basu2013}. Em aspecto similar ao observado no \textit{dataset Open University}, com muitas amostras os modelos das classes se tornam muito consistentes. Os resultados observados ficam evidentes na Figura \ref{fig-PG75}.

\begin{figure}[!h]
\centering
\includegraphics[width=\textwidth]{figuras/Powergrading-75}
\caption{Resultados dos classificadores com dados do \textit{dataset Powergrading}.}
\label{fig-PG75}
\end{figure}

Podemos, através da Figura \ref{fig-PG75}, destacar a relação da \textit{clusterização} na amostragem para o classificador. Apresentando textos restritos ao tema, sucintos e em \textit{clusters} homogêneos, são formados classificadores com alto desempenho. Deste modo, com modelos bem definidos nas etapas de \textit{clusterização} e classificação produzimos avaliadores SAG muito similares ao tutor. Por outro lado, em alguns \textit{datasets} a diversidade textual, a característica das notas atribuídas, os modelos linguísticos e a expectativa de resposta influenciam muito na capacidade do avaliador automático. Todas estas características são evidenciadas com conteúdos mais complexos, como a base de dados da competição do \textit{Kaggle ASAP-SAS}.



\begin{comment}

Edição a partir daqui
\end{comment}


Em outra perspectiva, temos o \textit{dataset} da \textit{University of North Texas} da avaliação com poucos dados. Neste conjunto, cada questão contém 30 respostas. Todas as perguntas são referentes a disciplina de Estrutura de Dados do curso de Ciência da Computação. Assim, como os demais procedimentos, sem usar respostas candidatas, utilizamos a amostragem através dos grupos resultantes da \textit{clusterização}. Foi avaliada a atribuição de notas para três anotações do conjunto de dados: \textit{Avaliador1}, \textit{Avaliador2} e a \textit{Média}. Os resultados obtidos com os algoritmos de regressão LINR, LSSR, KNRG, DTRG, WSRG na atribuição de notas entre 0 e 5 são apresentados na Tabela \ref{tab-UNT75}.

\begin{table}[!h]
\begin{center}
\begin{tabular}{p{5cm} r r r }
    \hline
    \multicolumn{4}{l}{\textbf{University of North Texas} (Notas 0 - 5)} \\ \hline
     & \multicolumn{3}{c}{M{\'e}tricas} \\

    & \multicolumn{3}{l}{Avaliador1} \\ \cline{2-4}
 & MAE & MSE & RMSE \\
LINR & 1,0066 & \textbf{2,5069} & \textbf{1,1955} \\
LSSR & 1,3273 & 3,1713 & 1,4712 \\
KNRG & 0,9366 & 2,9032 & 1,2557 \\
DTRG & \textbf{0,9233} & 3,7338 & 1,4482 \\
WSRG & 1,2832 & 3,0113 & 1,4240 \\
\\
& \multicolumn{3}{l}{Avaliador2} \\ \cline{2-4}
 & MAE & MSE & RMSE \\
LINR & \textbf{0,4752} & \textbf{0,6099} & \textbf{0,6119} \\
LSSR & 0,6502 & 0,8605 & 0,7640 \\
KNRG & 0,4917 & 0,7550 & 0,6658 \\
DTRG & 0,5121 & 1,2002 & 0,7856 \\
WSRG & 0,6523 & 0,8839 & 0,7680 \\
\\
& \multicolumn{3}{l}{Média} \\ \cline{2-4}
 & MAE & MSE & RMSE \\
LINR & \textbf{0,5058} & \textbf{0,5476} & \textbf{0,6199} \\
LSSR & 0,7299 & 0,8464 & 0,8170 \\
KNRG & 0,5055 & 0,6804 & 0,6765 \\
DTRG & 0,5811 & 1,1244 & 0,8372 \\
WSRG & 0,7024 & 0,8088 & 0,7920 \\

    \hline
    \hline
\end{tabular}
\end{center}
\caption{Índices de erro para cada algoritmos de regressão resultantes de cada um dos três cenários de avaliação do \textit{dataset} da \textit{University of North Texas}.}
\label{tab-UNT75}
\end{table}

A Tabela \ref{tab-UNT75} detalha o desempenho de cada algoritmo de regressão para as métricas MAE, MSE e RMSE. Nos resultados, destacamos a divergência entre avaliadores, com o \textit{Avaliador 1} sendo mais irregular na análise do sistema. Por outro lado, observando os classificadores, ressaltamos a capacidade de reconhecimento de padrões mais efetiva aos modelos do \textit{Avaliador 2} e da \textit{Média}. A diferença entre os melhores resultados se destacam, para o primeiro avaliador obtemos RMSE de 1,1930 pontos. Enquanto isso, para o segundo obtemos RMSE de 0,6099 pontos, uma queda de 0,5831 pontos. Neste conjunto de dados, conforme a literatura, o RMSE foi utilizado como padrão para comparações com os demais trabalhos, buscando minimizar o erro obtido de acordo com a nota média. As Figuras \ref{fig-UNT-Err} e \ref{fig-UNT-RMSE} detalham os erros encontrados para as três métricas por cada um dos algoritmos e a comparação com o RMSE dos trabalhos da literatura.

\begin{figure}[!h]
\begin{minipage}[t]{.5\textwidth}
 \centering
 \includegraphics[width=\textwidth]{figuras/UNT-MAE.png}
\end{minipage}
\hfill
\begin{minipage}[t]{.5\textwidth}
 \centering
 \includegraphics[width=\textwidth]{figuras/UNT-MSE.png}
\end{minipage}
\caption{Índices de MAE e MSE para os algoritmos testados em cada um dos três cenários de avaliação do \textit{dataset} da \textit{University of North Texas}.}
\label{fig-UNT-Err}
\end{figure}

\begin{figure}[!h]
\centering
\includegraphics[width=\textwidth]{figuras/UNT-RMSE}
\caption{Comparação entre o índice de RMSE obtidos pelo sistema e modelos propostos na literatura.}
\label{fig-UNT-RMSE}
\end{figure}

Como as Figuras \ref{fig-UNT-Err} e \ref{fig-UNT-RMSE} ilustram, existe uma grande diferença entre os avaliadores. O erro observado entre humanos em RMSE é de 0,66 pontos \cite{mohler2011}. Em uma comparação com demais métodos de avaliação, podemos destacar os excelentes resultados obtidos com o algoritmo de regressão linear simples LINR, como o melhor em todos os três cenários. Entretanto, dentre os algoritmos de regressão testados observamos consistência entre resultados, com pequenas variações entre os algoritmos. Enquanto na literatura o menor resultado observado é de 0,793 em RMSE, alcançamos resultados de 0,617 com LINR, seguido pelo KNRG com 0,677 pontos. Dada a avaliação com base no alinhamento com a resposta candidata elaborada pelo professor, os resultados obtidos reforçam que o uso da amostragem na produção do critério avaliativo pode ser muito efetivo. Assim, o estudo da distribuição de amostras torna o sistema conhecedor da dinâmica de avaliação realizando, neste caso, a interpolação em um espectro de notas conhecido. Esse nível, apesar da amostragem pré-estabelecida, caracteriza-se por um RMSE bem abaixo do observado até o momento na literatura \cite{ramachandran2015b, kumar2017, sahu2020}.





Adicionalmente, utilizamos dados \textit{Projeto Feira Literária} para ilustrar os resultados em dados nacionais. Em português, tais dados são um comparativo direto aos resultados obtidos em dados do exterior. Este dado foi coletado em conjunto com os autores para descrição da aplicação do \textit{p}Nota e seu uso por professores \cite{nascimento2020}. Este caracteriza-se pela presença de erros de escrita e conteúdos fora de tópico, sendo fatores avaliados negativamente pelo tutor. A Tabela \ref{tab-findes75} caracteriza o desempenho de cada um dos seis classificadores e seus resultados alcançados na avaliação das questões do projeto.

\begin{table}[!h]
\begin{center}
\begin{tabular}{l r r r r r r r}
    \hline
    \multicolumn{7}{l}{\textbf{Projeto Feira Liter{\'a}ria}} & (4 Categorias) \\ \hline
     & \multicolumn{7}{c}{M{\'e}tricas} \\ \cline{2-8}

    \input{tabelas/tab-findes75.tex}

    \hline
    \hline
\end{tabular}
\end{center}
\caption{Resultados de classificação para o \textit{Projeto Feira Literária}.}
\label{tab-findes75}
\end{table}

Conforme a Tabela \ref{tab-findes75}, observamos que dos seis classificadores testados, três apresentam resultados de boa qualidade. Apesar dos desafios do conjunto de dados, os algoritmos RDF, WSD e GBC apresentam resultados superiores a 60\% em ACC e F1-ponderado. Com 70 respostas e a pluralidade de estruturas textuais encontradas, destacamos a similaridade entre o desafio de correção dos datasets \textit{Beetle} e \textit{SciEntsBank} no ensino de ciências. Porém, podemos ressaltar como uma diferença melhores índices de PRE, REC e F1-macro. Podemos observar através da Figura \ref{fig-findes75} a proximidade destes índices com o grau de ACC alcançado.

\begin{figure}[!h]
\centering
\includegraphics[width=\textwidth]{figuras/Findes-75}
\caption{Resultados alcançados para os classificadores com dados em português do \textit{Projeto Feira Literária}.}
\label{fig-findes75}
\end{figure}

Nessa perspectiva, a Figura \ref{fig-findes75} representa um equilíbrio dos classificadores na produção de modelos por nota e a avaliação em geral de cada categoria. Quatro classificadores, RDF, WSD, GBC e DTR, apresentam alto desempenho, enquanto KNN e SVM destoam com resultados inferiores a 50\% em F1-ponderado. Portanto, os modelos avaliativos gerados demonstram complexidade mas em geral têm bons resultados.


\subsection{Resultados de \textit{Amostragem}}
\label{sec-res-amostragem}


\section{Discussão de Resultados}
\label{sec-discussao}

Os sistemas SAG têm características interessantes e de grande complexidade. Ao tempo que é um desafio lidar com diferentes tamanhos e padrões de resposta, ainda existem grande impacto o critério de avaliação. O desbalanceamento entre classes, os múltiplos padrões de resposta e a baixa quantidade de exemplos tornam complexa a criação de modelos avaliativos robustos.

De qualquer forma, por conta da dinâmica de avaliação própria de cada questão, vemos dentro de cada base de dados múltiplas perspectivas. Enquanto parte destoa com dificuldade no processo avaliativo, outras questões apresentam alto desempenho avaliativo com qualidade superior até a avaliação entre humanos. Um destaque no método avaliativo do \textit{p}Nota quando comparado à demais abordagens da literatura é a capacidade de processamento em várias linguagens. Nesse aspecto, a adaptabilidade do sistema com a análise personalizada da estrutura linguística caracteriza-se como um diferencial das abordagens mais recentes. Em sua aplicação, o sistema demanda entre 5 minutos e 6 horas de processamento conforme o número de características e amostras do conjunto de dados, desconsiderando o processo de anotação do professor. Logo, é fundamental adequar o sistema ao uso cotidiano do tutor e seu ritmo de correção, tendo em vista a entrega rápida de resultados em sala.

Na criação do modelo linguístico, os sistemas trabalham o alinhamento entre o conjunto de respostas e as respostas candidatas. Entretanto, a proposta apresentada neste trabalho busca a evolução do modelo criado iterativamente com a avaliação do professor. Para além da análise textual o sistema prioriza a conexão entre conteúdo e critério avaliativo. Deste modo, as nuances textuais interpretadas pelo tutor durante a avaliação são destacadas com o modelo estabelecido no modelo de atribuição de notas. Superdimensionados para a avaliação de uma determinada disciplina ou tema, os modelos rígidos divergem bastante da aplicação dos sistemas SAG no cotidiano do professor. Assim, o tutor espera que o sistema seja capaz de reduzir o esforço de correção, dando suporte ao seu método assim que requisitado. Portanto, independente do cenário ao qual é aplicado, o sistema SAG deve lidar com as respostas buscando minorar a demanda de verificação do conteúdo caracterizando as demandas de ensino-aprendizagem.

Além disso, ainda é importante salientar que, apesar de serem comuns os modelos rígidos, direcionados a domínios específicos ou dependentes de regras, o modelo proposto neste trabalho foi o mesmo aplicado para todas as questões. Por conta disso, inclui-se na qualidade dos resultados obtidos a capacidade do próprio sistema na adaptação ao tema e ao modelo de avaliação. Nesta linha, poucos parâmetros são passados, tornando o sistema uma cadeia de decisões sobre o processo. A lista completa de parâmetros está no Capítulo \ref{manual-pNota}.

Portanto, a escolha dos modelos segundo seu alinhamento com a avaliação humana acrescenta fluidez no processo de correção. Tornamos o modelo flexível para que, cada modelo seja utilizado nas situações ao qual melhor se adequa. O nível de adequação entre o modelo e a expectativa do professor é dada através das medidas de correlação de \textit{Pearson} ou coeficiente \textit{Kappa}. A tendência, então, é que os métodos de classificação ou de regressão selecionados sejam os que tem desempenho superior, com o professor auditando os resultados realizando apenas pequenos ajustes.
% ==============================================================================
% TCC - Nome do Aluno
% Capítulo 3 - Considerações Finais
% ==============================================================================
\chapter{Considerações Finais}
\label{cap-conclusoes}

\section{Trabalhos Futuros}

\section{Conclusões}


%%% Páginas finais do documento: bibliografia e anexos. %%%

% Finaliza a parte no bookmark do PDF para que se inicie o bookmark na raiz e adiciona espaço de parte no sumário.
\phantompart

% Marca o início dos elementos pós-textuais.
\postextual

% Referências bibliográficas
\bibliography{bibliografia}


% Apêndices.
\begin{apendicesenv}

% Imprime uma página indicando o início dos apêndices.
\partapendices


%
% Apêndices
%
\chapter{Relatorios \textit{p}Nota}
\label{exemplo-pNota}

O \textit{p}Nota tem uma série de relatórios que são utilizados para demonstrar o \textit{status} da avaliação e capacitar a auditoria do processo. Nesse constam os seguintes itens:

\begin{itemize}
\item Lista de Requisição de Amostras
\item Distribuição de \textit{Clusters} de Documentos
\item Distribuição de Treino
\item Distribuição de Características
\item Frequência dos Termos
\item Rubricas por Nota
\item Resultados de Classificação
\item Gráficos de Desempenho
\item Matriz de Confusão
\item Árvore de Decisão
\end{itemize}

Para ilustrar os resultados obtidos com o \textit{p}Nota vamos utilizar um pequeno exemplo. O exemplo é dado pela \textit{Atividade 08b}, extraído do \textit{dataset} do Projeto Feira Literária das Ciências Exatas. O enunciado da questão é dado a seguir:

\begin{quote}
\interlinepenalty=10000
\textit{Em eventos e festas, é comum as pessoas colocarem sal grosso sobre blocos de gelo para manter a temperatura. Por que essa ação mantém o gelo?.}
\end{quote}

A seguir é apresentado o arquivo completo de relatórios e \textit{feedbacks} gerados durante o processo.

\includepdf[pages={1-}]{relatorio-pNota.pdf}

\end{apendicesenv}


% Índice remissivo.
\phantompart
\printindex

% Fim do documento.
\end{document}